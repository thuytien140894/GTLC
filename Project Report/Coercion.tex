\subsection{Coercions}
Herman et al. [4] represent casts as combinable coercions to reduce 
space comsumption of gradual typing while preserving its semantics. 
\subsection{Coercions}
Herman et al. [4] represent casts as combinable coercions to reduce 
space comsumption of gradual typing while preserving its semantics. 
\subsection{Coercions}
Herman et al. [4] represent casts as combinable coercions to reduce 
space comsumption of gradual typing while preserving its semantics. 
\input{figures/Coercion}
The syntax and type system for the coercion calculus are defined in 
Figure 4 [4]. Note that we add blame labels to certain coercions to 
enable blame tracking in our gradual type system. The judgment $\vdash c: S 
\rightsquigarrow T$ indicates that coercion $c$ casts a value of type 
$S$ to type $T$. The identity $I$ is a coercion of any type to itself. 
$D!$ injects a type into ? while $D?^l$ projects ? into a precise type. 
Since an injection is always safe, we do not need to blame-track it. 
A coercion of two inconsistent types results in $\texttt{Fail}^l$. 

As discussed in the previous section, UD blame assignment arises  
from the coercion calculus because a coercion 
between a function type and ? always goes through the 
dynamic function type $? \rightarrow \: ?$. Thus,
the coercion $? \rightsquigarrow (S \rightarrow T)$ is an 
abstraction of two coercions: $? \rightsquigarrow 
(? \rightarrow \: ?)$ first to make sure ? is a function type 
and then $(? \rightarrow \: ?) \rightsquigarrow 
(S \rightarrow T)$. Similarly, the coercion $(S \rightarrow T) \rightsquigarrow \: ?$ 
performs the coercion $(S \rightarrow T) \rightsquigarrow (? \rightarrow \: ?)$ 
followed by $(? \rightarrow \: ?) \rightsquigarrow \: ?$ to 
generalize a function dynamic type.  
The coercion calculus uses the indirection $\texttt{Fun}?^l$ and $\texttt{Fun}!$ 
to represent the injection $(? \rightarrow \: ?) \rightsquigarrow \: ?$ and 
projection $? \rightsquigarrow (? \rightarrow \: ?)$. The same concept is 
applied to reference coercions. The injection $\texttt{Ref}!$ and 
projection $\texttt{Ref}?^l$ are auxiliary to the coercions $\texttt{Ref} \: T 
\rightsquigarrow \: ?$ and $? \rightsquigarrow \texttt{Ref} \: T$.

Multiple coercions can behave as one. The function coercion $\texttt{Fun} \: c \: d$ 
applies coercion $c$ to the argument and $d$ to the result. Coercing 
two function types $(S_1 \rightarrow S_2) \rightsquigarrow (T_1 \rightarrow T_2)$ 
coerces $T_1$ to $S_1$ and $S_2$ to $T_2$. Likewise, the reference coercion 
$\texttt{Ref} \: S \rightsquigarrow \texttt{Ref} \: T$ results in 
$\texttt{Ref} \: c \: d$, where $c$ coerces all writes from 
$T$ to $S$ and $d$ coerces all reads from $S$ to $T$. Finally, the coercion 
composition $c;d$ applies each coercion from left to right.

\subsection{Reduction Rules}
The bounded space consumption of coercions derives from their reducibility. 
The core set of normalization rules is given in Figure 5. These rules model 
after those of Herman et al. [4] with some additions from Siek and Garcia [5]. 
The added rules are needed for blame assignment and its confluence. First, 
we omit the rule 
\begin{equation*}
    c;\texttt{Fail}^l = \texttt{Fail}^l
\end{equation*}
to ensure that blame should always come from the closest coercion to an 
expression, unless it is an injection. For instance, $\langle \texttt{Nat}?^{l_1};\texttt{Fail}^{l_2} 
\rangle \langle \texttt{Bool}! \rangle \texttt{true} \longrightarrow 
\langle \texttt{Bool}!;\texttt{Nat}?^{l_1};\texttt{Fail}^{l_2} \rangle \texttt{true}
\longrightarrow \langle \texttt{Fail}^{l_1};\texttt{Fail}^{l_2} \rangle \texttt{true}
\longrightarrow \texttt{Fail}^{l_1}$. Here the blame is assigned to the 
most inner coercion $l_1$, instead of $l_2$ 
if $\texttt{Fail}^{l_2}$ is allowed to propagate from the right. 

Coercions can be reassociated to enable further reduction. A coercion $c$ that 
cannot be further reduced is normalized, represented as $\tilde{c}$. Figure 
5 shows a list of normal coercions. Note that a composition of normalized 
coercion is only normal if neither is the identity $I$. Also, we omit 
the normalized coercion $\texttt{Fail}^l$, which is handled separately 
during evaluation. 

\input{figures/Reduction}

\subsection{Eager Error Detection}
Let us revisit the expression described in the \textit{Introduction}:
\begin{equation*}
    \langle \texttt{Bool} \rightarrow \texttt{Nat} \Leftarrow  
    \: ? \rightarrow \: ? \rangle ^l 
    \langle ? \rightarrow \: ? \Leftarrow 
    \texttt{Nat} \rightarrow \texttt{Nat} \rangle
    (\lambda x \! : \! \texttt{Nat} . \; x)
\end{equation*}
Observe that this cast would fail because type $\texttt{Bool} \rightarrow 
\texttt{Nat}$ is inconsistent with type $\texttt{Nat} \rightarrow 
\texttt{Nat}$. However, this error would not be detected in the traditional 
cast checking system until $(\lambda x \! : \! \texttt{Nat} . \; x)$ is 
applied to a boolean. On the other hand, our gradual type system would not 
perform any function application on this expression until its coercions are 
normalized. Using one of the core reduction rules, we obtain:
\begin{equation*}
    \langle \texttt{Fun} \: \texttt{Fail}^l \: I \rangle
    (\lambda x \! : \! \texttt{Nat} . \; x)
\end{equation*}
where $\texttt{Fail}^l$ results from the attempt to coerce $\texttt{Nat} \rightarrow 
\texttt{Nat}$ to $\texttt{Bool} \rightarrow \texttt{Nat}$. As a result, 
the evaluation process should effectively terminate to a cast error before 
performing any more unnecessary operations. In order to achieve this 
early error detection, we extend the coercion normalization rules with 
four eager patterns (see Figure 5) to propagate cast errors. These added rules are similar to 
those given by Herman et al. [4] but require the coercion in the argument 
position to be normalized when $\texttt{Fail}^l$ is in the result [8]. 
This restriction makes sure that a failed argument coercion takes the blame when 
the result coercion fails too. Failed reference coercions follow an 
analogous pattern: blame only goes to a failed ``read'' coercion if 
the ``write'' coercion is normal and not a failure.

We have covered the design specifications for gradual typing based on coercions 
with UD blame assignment. In the next section, we will apply this design choice 
to implement a $\lambda ^? _{\rightarrow}$ interpreter in Haskell. 
The syntax and type system for the coercion calculus are defined in 
Figure 4 [4]. Note that we add blame labels to certain coercions to 
enable blame tracking in our gradual type system. The judgment $\vdash c: S 
\rightsquigarrow T$ indicates that coercion $c$ casts a value of type 
$S$ to type $T$. The identity $I$ is a coercion of any type to itself. 
$D!$ injects a type into ? while $D?^l$ projects ? into a precise type. 
Since an injection is always safe, we do not need to blame-track it. 
A coercion of two inconsistent types results in $\texttt{Fail}^l$. 

As discussed in the previous section, UD blame assignment arises  
from the coercion calculus because a coercion 
between a function type and ? always goes through the 
dynamic function type $? \rightarrow \: ?$. Thus,
the coercion $? \rightsquigarrow (S \rightarrow T)$ is an 
abstraction of two coercions: $? \rightsquigarrow 
(? \rightarrow \: ?)$ first to make sure ? is a function type 
and then $(? \rightarrow \: ?) \rightsquigarrow 
(S \rightarrow T)$. Similarly, the coercion $(S \rightarrow T) \rightsquigarrow \: ?$ 
performs the coercion $(S \rightarrow T) \rightsquigarrow (? \rightarrow \: ?)$ 
followed by $(? \rightarrow \: ?) \rightsquigarrow \: ?$ to 
generalize a function dynamic type.  
The coercion calculus uses the indirection $\texttt{Fun}?^l$ and $\texttt{Fun}!$ 
to represent the injection $(? \rightarrow \: ?) \rightsquigarrow \: ?$ and 
projection $? \rightsquigarrow (? \rightarrow \: ?)$. The same concept is 
applied to reference coercions. The injection $\texttt{Ref}!$ and 
projection $\texttt{Ref}?^l$ are auxiliary to the coercions $\texttt{Ref} \: T 
\rightsquigarrow \: ?$ and $? \rightsquigarrow \texttt{Ref} \: T$.

Multiple coercions can behave as one. The function coercion $\texttt{Fun} \: c \: d$ 
applies coercion $c$ to the argument and $d$ to the result. Coercing 
two function types $(S_1 \rightarrow S_2) \rightsquigarrow (T_1 \rightarrow T_2)$ 
coerces $T_1$ to $S_1$ and $S_2$ to $T_2$. Likewise, the reference coercion 
$\texttt{Ref} \: S \rightsquigarrow \texttt{Ref} \: T$ results in 
$\texttt{Ref} \: c \: d$, where $c$ coerces all writes from 
$T$ to $S$ and $d$ coerces all reads from $S$ to $T$. Finally, the coercion 
composition $c;d$ applies each coercion from left to right.

\subsection{Reduction Rules}
The bounded space consumption of coercions derives from their reducibility. 
The core set of normalization rules is given in Figure 5. These rules model 
after those of Herman et al. [4] with some additions from Siek and Garcia [5]. 
The added rules are needed for blame assignment and its confluence. First, 
we omit the rule 
\begin{equation*}
    c;\texttt{Fail}^l = \texttt{Fail}^l
\end{equation*}
to ensure that blame should always come from the closest coercion to an 
expression, unless it is an injection. For instance, $\langle \texttt{Nat}?^{l_1};\texttt{Fail}^{l_2} 
\rangle \langle \texttt{Bool}! \rangle \texttt{true} \longrightarrow 
\langle \texttt{Bool}!;\texttt{Nat}?^{l_1};\texttt{Fail}^{l_2} \rangle \texttt{true}
\longrightarrow \langle \texttt{Fail}^{l_1};\texttt{Fail}^{l_2} \rangle \texttt{true}
\longrightarrow \texttt{Fail}^{l_1}$. Here the blame is assigned to the 
most inner coercion $l_1$, instead of $l_2$ 
if $\texttt{Fail}^{l_2}$ is allowed to propagate from the right. 

Coercions can be reassociated to enable further reduction. A coercion $c$ that 
cannot be further reduced is normalized, represented as $\tilde{c}$. Figure 
5 shows a list of normal coercions. Note that a composition of normalized 
coercion is only normal if neither is the identity $I$. Also, we omit 
the normalized coercion $\texttt{Fail}^l$, which is handled separately 
during evaluation. 

\begin{figure}[ht]
    \caption{Coercion Normalization}
    \hrule
    \vspace{4mm}
    \qquad \text{\underline{N}ormal Coercions:}
    \begin{alignat*}{3}
        &(\textit{normal coercions}) \qquad 
        & \tilde{c} ::= \: & I \; | \;  
              B?^l \; | \;
              B! \; | \; 
              \hat{c};B! \; | \; 
              B?^l;\hat{c} \; | \; 
              B?^l; \texttt{Fail}^l \; | \; \\
        &&    & \texttt{Fun} \: \hat{c} \: \hat{d} \; | \;
              \hat{c};\texttt{Fun}! \; | \; 
              \texttt{Fun}?^l;\hat{c} \; | \; 
              \texttt{Fun} \: \hat{c} \: \hat{d}; \texttt{Fail}^l \; | \; \\
        &&    & \texttt{Ref} \: \hat{c} \: \hat{d} \; | \; 
              \texttt{Ref} \: \hat{c} \: \hat{d}; \texttt{Fail}^l \; | \;
              \hat{c};\texttt{Ref}! \; | \;
              \texttt{Ref}?^l;\hat{c} \\
        &(\textit{regular coercions}) \qquad
        & \hat{c} ::= \: & \tilde{c} \: \: \text{where} \: \: \tilde{c} \neq I
    \end{alignat*}
    \qquad \text{\underline{C}ore Reduction:}
    \begin{alignat*}{3} 
        I;c & = c \hspace{1cm} & \texttt{Fun} \: I \: I & = I \\
        c;I & = c \hspace{1cm} & \texttt{Ref} \: I \: I & = I \\
        D!;D?^l & = I \hspace{1cm} & c_1;(\tilde{c_2;c_3}) & = (c_1;c_2);c_3 \\
        D!;\texttt{Fail}^l & = \texttt{Fail}^l \hspace{1cm} & (\tilde{c_1;c_2});c_3 & = c_1;(c_2;c_3) \\
        \texttt{Fail}^l;c & = \texttt{Fail}^l \hspace{1cm} & (\texttt{Fun} \: \tilde{c_1} \: \tilde{c_2});(\texttt{Fun} \: \tilde{d_1} \: \tilde{d_2}) & = 
        \texttt{Fun} \: (\tilde{d_1};\tilde{c_1}) \: (\tilde{c_2};\tilde{d_2}) \\
        D!;D'?^l & = \texttt{Fail}^l \hspace{1cm} & (\texttt{Ref} \: \tilde{c_1} \: \tilde{c_2});(\texttt{Ref} \: \tilde{d_1} \: \tilde{d_2}) & = 
        \texttt{Ref} \: (\tilde{d_1};\tilde{c_1}) \: (\tilde{c_2};\tilde{d_2})
    \end{alignat*}
    \qquad \text{\underline{E}ager Reduction:}
    \begin{alignat*}{3} 
        \texttt{Fun} \: \texttt{Fail}^l \: d & = \texttt{Fail}^l \hspace{1cm} & 
        \texttt{Ref} \: \texttt{Fail}^l \: d & = \texttt{Fail}^l \\
        \texttt{Fun} \: \tilde{c} \: \texttt{Fail}^l & = \texttt{Fail}^l \hspace{1cm} & 
        \texttt{Ref} \: \tilde{c} \: \texttt{Fail}^l & = \texttt{Fail}^l 
    \end{alignat*}
    \hrule
\end{figure} 

\subsection{Eager Error Detection}
Let us revisit the expression described in the \textit{Introduction}:
\begin{equation*}
    \langle \texttt{Bool} \rightarrow \texttt{Nat} \Leftarrow  
    \: ? \rightarrow \: ? \rangle ^l 
    \langle ? \rightarrow \: ? \Leftarrow 
    \texttt{Nat} \rightarrow \texttt{Nat} \rangle
    (\lambda x \! : \! \texttt{Nat} . \; x)
\end{equation*}
Observe that this cast would fail because type $\texttt{Bool} \rightarrow 
\texttt{Nat}$ is inconsistent with type $\texttt{Nat} \rightarrow 
\texttt{Nat}$. However, this error would not be detected in the traditional 
cast checking system until $(\lambda x \! : \! \texttt{Nat} . \; x)$ is 
applied to a boolean. On the other hand, our gradual type system would not 
perform any function application on this expression until its coercions are 
normalized. Using one of the core reduction rules, we obtain:
\begin{equation*}
    \langle \texttt{Fun} \: \texttt{Fail}^l \: I \rangle
    (\lambda x \! : \! \texttt{Nat} . \; x)
\end{equation*}
where $\texttt{Fail}^l$ results from the attempt to coerce $\texttt{Nat} \rightarrow 
\texttt{Nat}$ to $\texttt{Bool} \rightarrow \texttt{Nat}$. As a result, 
the evaluation process should effectively terminate to a cast error before 
performing any more unnecessary operations. In order to achieve this 
early error detection, we extend the coercion normalization rules with 
four eager patterns (see Figure 5) to propagate cast errors. These added rules are similar to 
those given by Herman et al. [4] but require the coercion in the argument 
position to be normalized when $\texttt{Fail}^l$ is in the result [8]. 
This restriction makes sure that a failed argument coercion takes the blame when 
the result coercion fails too. Failed reference coercions follow an 
analogous pattern: blame only goes to a failed ``read'' coercion if 
the ``write'' coercion is normal and not a failure.

We have covered the design specifications for gradual typing based on coercions 
with UD blame assignment. In the next section, we will apply this design choice 
to implement a $\lambda ^? _{\rightarrow}$ interpreter in Haskell. 
The syntax and type system for the coercion calculus are defined in 
Figure 4 [4]. Note that we add blame labels to certain coercions to 
enable blame tracking in our gradual type system. The judgment $\vdash c: S 
\rightsquigarrow T$ indicates that coercion $c$ casts a value of type 
$S$ to type $T$. The identity $I$ is a coercion of any type to itself. 
$D!$ injects a type into ? while $D?^l$ projects ? into a precise type. 
Since an injection is always safe, we do not need to blame-track it. 
A coercion of two inconsistent types results in $\texttt{Fail}^l$. 

As discussed in the previous section, UD blame assignment arises  
from the coercion calculus because a coercion 
between a function type and ? always goes through the 
dynamic function type $? \rightarrow \: ?$. Thus,
the coercion $? \rightsquigarrow (S \rightarrow T)$ is an 
abstraction of two coercions: $? \rightsquigarrow 
(? \rightarrow \: ?)$ first to make sure ? is a function type 
and then $(? \rightarrow \: ?) \rightsquigarrow 
(S \rightarrow T)$. Similarly, the coercion $(S \rightarrow T) \rightsquigarrow \: ?$ 
performs the coercion $(S \rightarrow T) \rightsquigarrow (? \rightarrow \: ?)$ 
followed by $(? \rightarrow \: ?) \rightsquigarrow \: ?$ to 
generalize a function dynamic type.  
The coercion calculus uses the indirection $\texttt{Fun}?^l$ and $\texttt{Fun}!$ 
to represent the injection $(? \rightarrow \: ?) \rightsquigarrow \: ?$ and 
projection $? \rightsquigarrow (? \rightarrow \: ?)$. The same concept is 
applied to reference coercions. The injection $\texttt{Ref}!$ and 
projection $\texttt{Ref}?^l$ are auxiliary to the coercions $\texttt{Ref} \: T 
\rightsquigarrow \: ?$ and $? \rightsquigarrow \texttt{Ref} \: T$.

Multiple coercions can behave as one. The function coercion $\texttt{Fun} \: c \: d$ 
applies coercion $c$ to the argument and $d$ to the result. Coercing 
two function types $(S_1 \rightarrow S_2) \rightsquigarrow (T_1 \rightarrow T_2)$ 
coerces $T_1$ to $S_1$ and $S_2$ to $T_2$. Likewise, the reference coercion 
$\texttt{Ref} \: S \rightsquigarrow \texttt{Ref} \: T$ results in 
$\texttt{Ref} \: c \: d$, where $c$ coerces all writes from 
$T$ to $S$ and $d$ coerces all reads from $S$ to $T$. Finally, the coercion 
composition $c;d$ applies each coercion from left to right.

\subsection{Reduction Rules}
The bounded space consumption of coercions derives from their reducibility. 
The core set of normalization rules is given in Figure 5. These rules model 
after those of Herman et al. [4] with some additions from Siek and Garcia [5]. 
The added rules are needed for blame assignment and its confluence. First, 
we omit the rule 
\begin{equation*}
    c;\texttt{Fail}^l = \texttt{Fail}^l
\end{equation*}
to ensure that blame should always come from the closest coercion to an 
expression, unless it is an injection. For instance, $\langle \texttt{Nat}?^{l_1};\texttt{Fail}^{l_2} 
\rangle \langle \texttt{Bool}! \rangle \texttt{true} \longrightarrow 
\langle \texttt{Bool}!;\texttt{Nat}?^{l_1};\texttt{Fail}^{l_2} \rangle \texttt{true}
\longrightarrow \langle \texttt{Fail}^{l_1};\texttt{Fail}^{l_2} \rangle \texttt{true}
\longrightarrow \texttt{Fail}^{l_1}$. Here the blame is assigned to the 
most inner coercion $l_1$, instead of $l_2$ 
if $\texttt{Fail}^{l_2}$ is allowed to propagate from the right. 

Coercions can be reassociated to enable further reduction. A coercion $c$ that 
cannot be further reduced is normalized, represented as $\tilde{c}$. Figure 
5 shows a list of normal coercions. Note that a composition of normalized 
coercion is only normal if neither is the identity $I$. Also, we omit 
the normalized coercion $\texttt{Fail}^l$, which is handled separately 
during evaluation. 

\begin{figure}[ht]
    \caption{Coercion Normalization}
    \hrule
    \vspace{4mm}
    \qquad \text{\underline{N}ormal Coercions:}
    \begin{alignat*}{3}
        &(\textit{normal coercions}) \qquad 
        & \tilde{c} ::= \: & I \; | \;  
              B?^l \; | \;
              B! \; | \; 
              \hat{c};B! \; | \; 
              B?^l;\hat{c} \; | \; 
              B?^l; \texttt{Fail}^l \; | \; \\
        &&    & \texttt{Fun} \: \hat{c} \: \hat{d} \; | \;
              \hat{c};\texttt{Fun}! \; | \; 
              \texttt{Fun}?^l;\hat{c} \; | \; 
              \texttt{Fun} \: \hat{c} \: \hat{d}; \texttt{Fail}^l \; | \; \\
        &&    & \texttt{Ref} \: \hat{c} \: \hat{d} \; | \; 
              \texttt{Ref} \: \hat{c} \: \hat{d}; \texttt{Fail}^l \; | \;
              \hat{c};\texttt{Ref}! \; | \;
              \texttt{Ref}?^l;\hat{c} \\
        &(\textit{regular coercions}) \qquad
        & \hat{c} ::= \: & \tilde{c} \: \: \text{where} \: \: \tilde{c} \neq I
    \end{alignat*}
    \qquad \text{\underline{C}ore Reduction:}
    \begin{alignat*}{3} 
        I;c & = c \hspace{1cm} & \texttt{Fun} \: I \: I & = I \\
        c;I & = c \hspace{1cm} & \texttt{Ref} \: I \: I & = I \\
        D!;D?^l & = I \hspace{1cm} & c_1;(\tilde{c_2;c_3}) & = (c_1;c_2);c_3 \\
        D!;\texttt{Fail}^l & = \texttt{Fail}^l \hspace{1cm} & (\tilde{c_1;c_2});c_3 & = c_1;(c_2;c_3) \\
        \texttt{Fail}^l;c & = \texttt{Fail}^l \hspace{1cm} & (\texttt{Fun} \: \tilde{c_1} \: \tilde{c_2});(\texttt{Fun} \: \tilde{d_1} \: \tilde{d_2}) & = 
        \texttt{Fun} \: (\tilde{d_1};\tilde{c_1}) \: (\tilde{c_2};\tilde{d_2}) \\
        D!;D'?^l & = \texttt{Fail}^l \hspace{1cm} & (\texttt{Ref} \: \tilde{c_1} \: \tilde{c_2});(\texttt{Ref} \: \tilde{d_1} \: \tilde{d_2}) & = 
        \texttt{Ref} \: (\tilde{d_1};\tilde{c_1}) \: (\tilde{c_2};\tilde{d_2})
    \end{alignat*}
    \qquad \text{\underline{E}ager Reduction:}
    \begin{alignat*}{3} 
        \texttt{Fun} \: \texttt{Fail}^l \: d & = \texttt{Fail}^l \hspace{1cm} & 
        \texttt{Ref} \: \texttt{Fail}^l \: d & = \texttt{Fail}^l \\
        \texttt{Fun} \: \tilde{c} \: \texttt{Fail}^l & = \texttt{Fail}^l \hspace{1cm} & 
        \texttt{Ref} \: \tilde{c} \: \texttt{Fail}^l & = \texttt{Fail}^l 
    \end{alignat*}
    \hrule
\end{figure} 

\subsection{Eager Error Detection}
Let us revisit the expression described in the \textit{Introduction}:
\begin{equation*}
    \langle \texttt{Bool} \rightarrow \texttt{Nat} \Leftarrow  
    \: ? \rightarrow \: ? \rangle ^l 
    \langle ? \rightarrow \: ? \Leftarrow 
    \texttt{Nat} \rightarrow \texttt{Nat} \rangle
    (\lambda x \! : \! \texttt{Nat} . \; x)
\end{equation*}
Observe that this cast would fail because type $\texttt{Bool} \rightarrow 
\texttt{Nat}$ is inconsistent with type $\texttt{Nat} \rightarrow 
\texttt{Nat}$. However, this error would not be detected in the traditional 
cast checking system until $(\lambda x \! : \! \texttt{Nat} . \; x)$ is 
applied to a boolean. On the other hand, our gradual type system would not 
perform any function application on this expression until its coercions are 
normalized. Using one of the core reduction rules, we obtain:
\begin{equation*}
    \langle \texttt{Fun} \: \texttt{Fail}^l \: I \rangle
    (\lambda x \! : \! \texttt{Nat} . \; x)
\end{equation*}
where $\texttt{Fail}^l$ results from the attempt to coerce $\texttt{Nat} \rightarrow 
\texttt{Nat}$ to $\texttt{Bool} \rightarrow \texttt{Nat}$. As a result, 
the evaluation process should effectively terminate to a cast error before 
performing any more unnecessary operations. In order to achieve this 
early error detection, we extend the coercion normalization rules with 
four eager patterns (see Figure 5) to propagate cast errors. These added rules are similar to 
those given by Herman et al. [4] but require the coercion in the argument 
position to be normalized when $\texttt{Fail}^l$ is in the result [8]. 
This restriction makes sure that a failed argument coercion takes the blame when 
the result coercion fails too. Failed reference coercions follow an 
analogous pattern: blame only goes to a failed ``read'' coercion if 
the ``write'' coercion is normal and not a failure.

We have covered the design specifications for gradual typing based on coercions 
with UD blame assignment. In the next section, we will apply this design choice 
to implement a $\lambda ^? _{\rightarrow}$ interpreter in Haskell. 
\subsection{Coercions}
Herman et al. [4] represent casts as combinable coercions to reduce 
space comsumption of gradual typing while preserving its semantics. 
The syntax and type system for the coercion calculus are defined in 
Figure 4 [4]. Note that we add blame labels to certain coercions to 
enable blame tracking in our gradual type system. The judgment $\vdash c: S 
\rightsquigarrow T$ indicates that coercion $c$ casts a value of type 
$S$ to type $T$. The identity $I$ is a coercion of any type to itself. 
$D!$ injects a type into ? while $D?^l$ projects ? into a precise type. 
Since an injection is always safe, we do not need to blame-track it. 
A coercion of two inconsistent types results in $\texttt{Fail}^l$. 

As discussed in the previous section, UD blame assignment arises  
from the coercion calculus because a coercion 
between a function type and ? always goes through the 
dynamic function type $? \rightarrow \: ?$. Thus,
the coercion $? \rightsquigarrow (S \rightarrow T)$ is an 
abstraction of two coercions: $? \rightsquigarrow 
(? \rightarrow \: ?)$ first to make sure that ? is a function type 
and then $(? \rightarrow \: ?) \rightsquigarrow 
(S \rightarrow T)$. Similarly, the coercion $(S \rightarrow T) \rightsquigarrow \: ?$ 
performs the coercion $(S \rightarrow T) \rightsquigarrow (? \rightarrow \: ?)$ 
followed by $(? \rightarrow \: ?) \rightsquigarrow \: ?$ to 
generalize a function dynamic type.  
The coercion calculus uses the tags $\texttt{Fun}!$ and $\texttt{Fun}?^l$ 
to represent the injection $(? \rightarrow \: ?) \rightsquigarrow \: ?$ and 
projection $? \rightsquigarrow (? \rightarrow \: ?)$. The same concept is 
applied to reference coercions. The injection $\texttt{Ref}!$ and 
projection $\texttt{Ref}?^l$ are auxiliary to the coercions $\texttt{Ref} \: T 
\rightsquigarrow \: ?$ and $? \rightsquigarrow \texttt{Ref} \: T$.

Multiple coercions can behave as one. The function coercion $\texttt{Fun} \: c \: d$ 
applies coercion $c$ to the argument and $d$ to the result. Coercing 
two function types $(S_1 \rightarrow S_2) \rightsquigarrow (T_1 \rightarrow T_2)$ 
coerces $T_1$ to $S_1$ and $S_2$ to $T_2$. Likewise, the reference coercion 
$\texttt{Ref} \: S \rightsquigarrow \texttt{Ref} \: T$ results in 
$\texttt{Ref} \: c \: d$, where $c$ coerces all writes from 
$T$ to $S$ and $d$ coerces all reads from $S$ to $T$. Finally, the coercion 
composition $c;d$ applies each coercion from left to right.

\subsection{Reduction Rules}
The bounded space consumption of coercions derives from their reducibility. 
The core set of normalization rules is given in Figure 5. These rules model 
after those found in Herman et al. [4] with some additions from Siek and Garcia [7]. 
The added rules are needed for blame assignment and its confluence. First, 
we omit the rule 
\begin{equation*}
    c;\texttt{Fail}^l = \texttt{Fail}^l
\end{equation*}
to ensure that blame should always come from the closest coercion to an 
expression, unless it is an injection. For instance, $\langle \texttt{Nat}?^{l_1};\texttt{Fail}^{l_2} 
\rangle \langle \texttt{Bool}! \rangle \texttt{true} \longrightarrow 
\langle \texttt{Bool}!;\texttt{Nat}?^{l_1};\texttt{Fail}^{l_2} \rangle \texttt{true}
\longrightarrow \langle \texttt{Fail}^{l_1};\texttt{Fail}^{l_2} \rangle \texttt{true}
\longrightarrow \texttt{Fail}^{l_1}$. Here the blame is assigned to the 
most inner coercion $l_1$, instead of $l_2$ 
if $\texttt{Fail}^{l_2}$ is allowed to propagate from the right. 

Coercions can be reassociated to enable further reduction. A coercion $c$ that 
cannot be further reduced is normalized, represented as $\tilde{c}$. Figure 
5 shows a list of normal coercions [7]. Note that a composition of normalized 
coercions is only normal if none of them are the identity $I$. Also, we omit 
the normalized coercion $\texttt{Fail}^l$, which is handled separately 
during evaluation. 

\begin{figure}[ht]
    \caption{Coercion Normalization}
    \hrule
    \vspace{4mm}
    \qquad \text{\underline{N}ormal Coercions:}
    \begin{alignat*}{3}
        &(\textit{normal coercions}) \qquad 
        & \tilde{c} ::= \: & I \; | \;  
              B?^l \; | \;
              B! \; | \; 
              \hat{c};B! \; | \; 
              B?^l;\hat{c} \; | \; 
              B?^l; \texttt{Fail}^l \; | \; \\
        &&    & \texttt{Fun} \: \hat{c} \: \hat{d} \; | \;
              \hat{c};\texttt{Fun}! \; | \; 
              \texttt{Fun}?^l;\hat{c} \; | \; 
              \texttt{Fun} \: \hat{c} \: \hat{d}; \texttt{Fail}^l \; | \; \\
        &&    & \texttt{Ref} \: \hat{c} \: \hat{d} \; | \; 
              \texttt{Ref} \: \hat{c} \: \hat{d}; \texttt{Fail}^l \; | \;
              \hat{c};\texttt{Ref}! \; | \;
              \texttt{Ref}?^l;\hat{c} \\
        &(\textit{regular coercions}) \qquad
        & \hat{c} ::= \: & \tilde{c} \: \: \text{where} \: \: \tilde{c} \neq I
    \end{alignat*}
    \qquad \text{\underline{C}ore Reduction:}
    \begin{alignat*}{3} 
        I;c & = c \hspace{1cm} & \texttt{Fun} \: I \: I & = I \\
        c;I & = c \hspace{1cm} & \texttt{Ref} \: I \: I & = I \\
        D!;D?^l & = I \hspace{1cm} & c_1;(\tilde{c_2;c_3}) & = (c_1;c_2);c_3 \\
        D!;\texttt{Fail}^l & = \texttt{Fail}^l \hspace{1cm} & (\tilde{c_1;c_2});c_3 & = c_1;(c_2;c_3) \\
        \texttt{Fail}^l;c & = \texttt{Fail}^l \hspace{1cm} & (\texttt{Fun} \: \tilde{c_1} \: \tilde{c_2});(\texttt{Fun} \: \tilde{d_1} \: \tilde{d_2}) & = 
        \texttt{Fun} \: (\tilde{d_1};\tilde{c_1}) \: (\tilde{c_2};\tilde{d_2}) \\
        D!;D'?^l & = \texttt{Fail}^l \hspace{1cm} & (\texttt{Ref} \: \tilde{c_1} \: \tilde{c_2});(\texttt{Ref} \: \tilde{d_1} \: \tilde{d_2}) & = 
        \texttt{Ref} \: (\tilde{d_1};\tilde{c_1}) \: (\tilde{c_2};\tilde{d_2})
    \end{alignat*}
    \qquad \text{\underline{E}ager Reduction:}
    \begin{alignat*}{3} 
        \texttt{Fun} \: \texttt{Fail}^l \: d & = \texttt{Fail}^l \hspace{1cm} & 
        \texttt{Ref} \: \texttt{Fail}^l \: d & = \texttt{Fail}^l \\
        \texttt{Fun} \: \tilde{c} \: \texttt{Fail}^l & = \texttt{Fail}^l \hspace{1cm} & 
        \texttt{Ref} \: \tilde{c} \: \texttt{Fail}^l & = \texttt{Fail}^l 
    \end{alignat*}
    \hrule
\end{figure} 

\subsection{Eager Error Detection}
Let us revisit expression (1) with added labels for casts:
\begin{equation*}
    \langle \texttt{Bool} \rightarrow \texttt{Nat} \Leftarrow  
    \: ? \rightarrow \: ? \rangle ^l 
    \langle ? \rightarrow \: ? \Leftarrow 
    \texttt{Nat} \rightarrow \texttt{Nat} \rangle
    (\lambda x \! : \! \texttt{Nat} . \; x)
\end{equation*}
Observe that this cast would fail because type $\texttt{Bool} \rightarrow 
\texttt{Nat}$ is inconsistent with type $\texttt{Nat} \rightarrow 
\texttt{Nat}$. However, this error would not be detected in the traditional 
cast checking system until $(\lambda x \! : \! \texttt{Nat} . \; x)$ is 
applied to a boolean. On the other hand, our gradual type system would not 
perform any function application on this expression until its coercions are 
normalized. Using one of the core reduction rules, we obtain:
\begin{equation*}
    \langle \texttt{Fun} \: \texttt{Fail}^l \: I \rangle
    (\lambda x \! : \! \texttt{Nat} . \; x)
\end{equation*}
where $\texttt{Fail}^l$ results from the attempt to coerce the argument type 
from \texttt{Nat} to \texttt{Bool}. As a result, 
the evaluation process should effectively terminate to a cast error before 
performing any more unnecessary operations. In order to achieve this 
early error detection, we extend the coercion normalization rules with 
four eager patterns (see Figure 5) to propagate cast errors. These added rules are similar to 
those given in Herman et al. [4] but require the coercion in the argument 
position to be normalized when the output coercion evaluates to $\texttt{Fail}^l$ 
in a function coercion [8]. 
This restriction makes sure that a failed argument coercion takes the blame when 
the output coercion fails too. Failed reference coercions follow an 
analogous pattern: blame only goes to a failed ``read'' coercion if 
the ``write'' coercion is normal and not a failure.

We have covered the design specifications for gradual typing based on coercions 
with UD blame assignment. In the next section, we will apply this design choice 
to implement a $\lambda ^? _{\rightarrow}$ interpreter in Haskell. 