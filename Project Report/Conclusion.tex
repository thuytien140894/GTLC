Our interpretation of $\lambda ^? _{\rightarrow}$ closely 
mimics the formal definitions of gradual typing based on coercions introduced in 
Herman et al. [4], with some modifications to the syntax 
and coercion reduction rules found in Siek et al. [8] to 
support UD blame tracking. The use of coercions instead of 
traditional cast wrappers helps improve the space efficiency  
of casts by eagerly combining adjacent casts during 
evaluation. As a result, the coercion calculus enables 
early error detection whenever coercions fail to combine. 
We also add blame tracking to the gradual system  
to inform the programmer of the origin of a cast error. 
We focus the interpretation of $\lambda ^? _{\rightarrow}$ on its 
type checking and evaluation procedures. The typechecker 
uses the consistency rules to insert necessary casts for 
dynamically-typed parts of a program. The evaluator is then 
responsible for reducing the modified program to a value 
or a cast error. 

Our current interpreter only accepts one statement. We would 
like to extend the abstract syntax tree to allow a program made of 
a sequence of statements. As a result, a cast error message 
should also include a line number to better identify the 
failed expression. Another possible improvement would be to 
explore how to incorporate subtyping 
into the gradual type system through the addition of records. 
The subtype relation in Figure 3 only 
serves to verify the soundness of UD blame assignment. It 
it not actually part of the implementation. 
