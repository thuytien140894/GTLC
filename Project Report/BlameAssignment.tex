Casts can be illegal, safe or unsafe. Casts are immediately rejected 
when the source and target types are inconsistent. A safe cast 
$\langle T \Leftarrow S \rangle$ respects 
the subtype relation $S <: T$, and therefore should not be blamed. 
On the other hand, any downcast from a supertype is considered unsafe. 
Our gradual type system keeps track of unsafe casts and identifies the 
one that is responsible for a cast error. 

As briefly mentioned in the \textit{Introduction}, there are two blame 
tracking strategies depending on how a function type is coerced 
to ?. The coercion calculus presented in Herman et al. [4] 
incorporates the function dynamic type $ ? \rightarrow \: ? $ 
to describe more specifically a function type with dynamic argument and return 
types. Accordingly, any coercion from a precise function type $S \rightarrow T$ 
to a general dynamic type ? has to check first if type ? 
is a function type. Consider the following evaluation for an expression 
with labeled casts:
\begin{align*}
    & (\langle ? \rightarrow \texttt{Nat} \Leftarrow 
    \: ? \rangle ^{l_1}
    \langle ? \Leftarrow 
    \texttt{Nat} \rightarrow \texttt{Nat} \rangle ^{l_2}
    (\lambda x \! : \! \texttt{Nat} . \; x)) 
    \langle ? \Leftarrow \texttt{Bool} \rangle ^{l_3} \texttt{true} \\
    \longrightarrow \:
    & (\langle ? \rightarrow \texttt{Nat} \Leftarrow 
    \: ? \rightarrow \: ? \rangle ^{l_1}
    \langle ? \rightarrow \: ? \Leftarrow 
    \texttt{Nat} \rightarrow \texttt{Nat} \rangle ^{l_2}
    (\lambda x \! : \! \texttt{Nat} . \; x)) 
    \langle ? \Leftarrow \texttt{Bool} \rangle ^{l_3} \texttt{true} \\
    \longrightarrow \:
    & \langle \texttt{Nat} \Leftarrow \: ? \rangle ^{l_1} 
    (\langle ? \rightarrow \: ? \Leftarrow 
    \texttt{Nat} \rightarrow \texttt{Nat} \rangle ^{l_2}
    (\lambda x \! : \! \texttt{Nat} . \; x))
    \langle ? \Leftarrow \: ? \rangle ^{l_1} 
    \langle ? \Leftarrow \texttt{Bool} \rangle ^{l_3} \texttt{true} \\
    \longrightarrow \:
    & \langle \texttt{Nat} \Leftarrow \: ? \rangle ^{l_1} 
    (\langle ? \rightarrow \: ? \Leftarrow 
    \texttt{Nat} \rightarrow \texttt{Nat} \rangle ^{l_2}
    (\lambda x \! : \! \texttt{Nat} . \; x)) 
    \langle ? \Leftarrow \texttt{Bool} \rangle ^{l_3} \texttt{true} \\
    \longrightarrow \: 
    & \langle \texttt{Nat} \Leftarrow \: ? \rangle ^{l_1} 
    \langle ? \Leftarrow \texttt{Nat} \rangle ^{l_2} 
    ((\lambda x \! : \! \texttt{Nat} . \; x) 
    \highlight{\langle \texttt{Nat} \Leftarrow \: ? \rangle ^{l_2}}
    \langle ? \Leftarrow \texttt{Bool} \rangle ^{l_3} \texttt{true}) \\
    \longrightarrow \:
    & \textbf{blame} \: l_2
\end{align*}
The blame is shared between the downcast from ? to \texttt{Nat} in the argument type 
and the upcast from \texttt{Nat} to ? in the return type for $\lambda x \! : \! \texttt{Nat} . \; x$. 
This type of blame assignment is called UD. 
However, the cast $\langle ? \Leftarrow \texttt{Nat} \rightarrow \texttt{Nat} \rangle ^{l_2}$ 
is considered a safe cast based on the traditional subtype relation where 
? is the top element. The intermediate step of converting type ? to 
the dynamic function type $ ? \rightarrow \: ?$ changes $\langle ? \Leftarrow \texttt{Nat} 
\rightarrow \texttt{Nat} \rangle ^{l_2}$ to
$\langle ? \rightarrow \: ? \Leftarrow \texttt{Nat} \rightarrow \texttt{Nat} \rangle ^{l_2}$. While  
$(\texttt{Nat} \rightarrow \texttt{Nat}) <: \: ?$ is true,    
$(\texttt{Nat} \rightarrow \texttt{Nat}) <: \: (? \rightarrow \: ?)$ is not. Siek et al. [4] 
present a modified subtype relation (defined in Figure 3) to prove the soundness of UD 
blame assignment. In this way, ? is no longer the top element, and a function type $S \rightarrow T$ 
is only a subtype of ? if it is a subtype of $ ? \rightarrow \: ?$. 
\begin{figure}[ht]
    \caption{Subtype Relations}
    \hrule
    \[
        \begin{bprooftree}
            \AxiomC{}
            \UnaryInfC{$B <: B$}
        \end{bprooftree}
        \begin{bprooftree}
            \AxiomC{}
            \UnaryInfC{$? <: \: ?$}
        \end{bprooftree}
        \begin{bprooftree}
            \AxiomC{$S <: B$}
            \UnaryInfC{$S <: \: ?$}
        \end{bprooftree}
        \begin{bprooftree}
            \AxiomC{$S <: \: ? \rightarrow \: ?$}
            \UnaryInfC{$S <: \: ?$}
        \end{bprooftree}
        \begin{bprooftree}
            \AxiomC{$T_1 <: S_1$}
            \AxiomC{$S_2 <: T_2$}
            \BinaryInfC{$S_1 \rightarrow S_2 <: T_1 \rightarrow T_2$}
      \end{bprooftree}
    \] 
    \begin{align*}
        \text{where} \: B ::= \texttt{Bool} \: | \: \texttt{Nat}
    \end{align*}
    \hrule
\end{figure} 