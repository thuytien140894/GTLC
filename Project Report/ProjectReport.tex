\documentclass[11pt]{article}
\usepackage{afterpage}
\usepackage{amsmath}
\usepackage{amssymb}
\usepackage{bussproofs}
\usepackage{caption}
\usepackage[margin=4cm]{geometry}
\usepackage{listings}
\usepackage{mathtools}
\usepackage{setspace}
\usepackage{textcomp}
\usepackage[hyphens]{url}
\usepackage{xcolor}

\allowdisplaybreaks

\newcommand\blankpage{
    \null
    \thispagestyle{empty}
    \addtocounter{page}{-1}
    \newpage}
    
\makeatletter
\newenvironment{shiftedflalign}{
  \start@align\tw@\st@rredfalse\m@ne
  \hskip\parindent
}{
  \endalign
}
\newenvironment{shiftedflalign*}{
  \start@align\tw@\st@rredtrue\m@ne
  \hskip\parindent
}{
  \endalign
}
\makeatother

\lstset{basicstyle=\small\sffamily,
        breaklines=true,
        language=Haskell,
        columns=flexible,
        keepspaces=true,
        tabsize=4,
        commentstyle=\color{gray},
        morekeywords={throwError},
        literate= {<-}{$\leftarrow$}{1}{->}{$\rightarrow$}{1},
        deletekeywords={fst,snd}
      }

\newenvironment{bprooftree}
  {\leavevmode\hbox\bgroup}
  {\DisplayProof\egroup}

\newcommand{\highlight}[1]{
  \colorbox{gray!15}{$\displaystyle#1$}}

\captionsetup{labelfont=bf,font=bf,skip=0pt}

\title{\bf Interpretation of the Gradually-Typed Lambda Calculus}
\author{Tien Thuy Ho}

\begin{document}

% Title page
\pagenumbering{roman}
\thispagestyle{empty}

\pagebreak
\begin{center}
  UNIVERSITY OF CALIFORNIA \\
  SANTA CRUZ \\
  \bigskip
  \textbf{INTERPRETATION OF THE GRADUALLY-TYPED} \\
  \textbf{LAMBDA CALCULUS} \\ 
  \bigskip
  A project submitted in partial satisfaction \\ 
  of the requirements for the degree of \\ 
  \bigskip
  MASTER OF SCIENCE \\ 
  \bigskip
  in \\
  \bigskip
  COMPUTER SCIENCE \\
  \bigskip 
  by \\ 
  \bigskip
  \textbf{Tien Thuy Ho} \\
  \bigskip
  June 2018
  \bigskip
  \bigskip
\end{center}
\begin{flushleft}
  \hspace{7cm}The Project of Tien Thuy Ho \\
  \hspace{7cm}is approved: \\
  \bigskip
  \bigskip
  \hspace{7cm} \hrulefill \\
  \hspace{7cm} Professor Cormac Flanagan, Chair \\ 
  \bigskip
  \bigskip
  \hspace{7cm} \hrulefill \\
  \hspace{7cm} Professor Owen Arden \\
  \bigskip
  \bigskip 
  \underline{\hspace{6cm}} \\ 
  Tyrus Miller \\
  Vice Provost and Dean of Graduate Studies
\end{flushleft}
\pagebreak

% Copyright
\thispagestyle{empty}
\pagebreak
    \hspace{0pt}
    \vfill
      \begin{center}
        Copyright \textcopyright \hspace{0.4mm} by \\
        \bigskip 
        Tien Thuy Ho \\
        \bigskip 
        2018
      \end{center}
    \vfill
    \hspace{0pt}
\pagebreak

% Table of contents
\setcounter{page}{3}
\tableofcontents
\listoffigures

\maketitle
\pagenumbering{arabic}

\begin{abstract}
Gradual typing offers both type safety and programming flexibility by integrating static and dynamic type checking 
together within the same language. The runtime semantics of a gradually-typed system involves the translation of the 
source language to an intermediate one with explicit casts. Space comsumption of gradual typing can be improved by using 
the coercion calculus to collapse a sequence of adjacent casts into a single cast [4]. This paper details an efficient 
interpretation of the gradually-typed lambda calculus based on coercions with eager error detection, as described formally 
by Herman et al. [4]. We also augment the system with UD blame assignment [8] to pinpoint the particular expressions 
responsible for coercion failures.
\end{abstract}

\section{Introduction} The argument for static versus dynamic typing underlines the fundamentatal 
motivation behind programming language design and implementation. Static 
typing enforces explicit type annotations on variables and function 
declarations, thereby restricting their behaviors and interactions. 
Such strict constraint of types helps eliminate runtime bugs caused 
by type conflicts, thus improving code reliability and easing the 
burden of debugging. Statically-typed languages such as Java 
and C++ are known for their runtime performance since the overhead of 
typechecking is done during compilation instead. Additionally, a large 
and complex application benefits from early error detection and better 
documentation found in static typing. On the other hand, dynamic typing 
removes the necessity of type annotations in favor of programming 
flexibility and rapid development. JavaScript and Python are two notable 
examples of dynamically-typed languages suitable for quick prototyping 
and adjustment to changing specifications. However, the lack of 
annotations and static type checking could impair code quality and 
reliability. The respective strengths and 
weaknesses of both type systems motivate the endeavor to integrate them 
together into a hybrid system called \textit{\textbf{gradual typing}}. 

The term \textit{gradual typing} originally came up in Siek and 
Taha's work [4] to describe the combination of static and dynamic typing 
for functional languages. Gradual typing gives the programmer the ability 
to control the extent of static checking through optional type annotations. 
Function parameters with type annotations are typechecked during 
compilation as usually, whereas those with omitted types are deferred until 
runtime for typechecking. This approach benefits from the safety of static 
typing and flexibility of dynamic typing, making it suitable for developing 
software from quick prototyping to reliable code delivery. The formal 
definition of gradual typing was first introduced using the 
simply-typed lambda calculus with references, abbreviated as $\lambda ^? _{\rightarrow}$. 
The runtime semantics of $\lambda ^? _{\rightarrow}$ requires the 
translation of the source language to an intermediate one with explicit 
cast insertion. The interpretation of $\lambda ^? _{\rightarrow}$ 
is simply the evaluation of the resulting cast expressions to produce 
either a final unboxed value or a cast error.

There have been many studies on how to efficiently design the runtime 
semantics of $\lambda ^? _{\rightarrow}$. Herman et al. [4] observe 
the space inefficiency of gradual typing wherein a program repeatedly applies 
the same cast to every method invocation, resulting in an unbounded space 
consumption of duplicate wrappers. They present a space-efficient gradual 
type system based on Henglein's \textit{\textbf{coercion calculus}} [3], 
which allows the composition of adjacent casts and reduces space consumption 
as a result. The coercion calculus also makes it possible for early error detection 
since failure to combine inconsistent coercions gets propagated 
immediately. Siek et al. [8] call this the \textit{\textbf{eager coercion calculus}} 
to contrast with the traditional lazy cast checking, which only detects cast 
errors during function application. Consider the following higher-order cast 
expression: 
\begin{equation}
    \langle \texttt{Bool} \rightarrow \texttt{Nat} \Leftarrow  
    \: ? \rightarrow \: ? \rangle 
    \langle ? \rightarrow \: ? \Leftarrow 
    \texttt{Nat} \rightarrow \texttt{Nat} \rangle
    (\lambda x \! : \! \texttt{Nat} . \; x)
\end{equation}
The eager cast checking ensures that each cast is consistent with the prior 
casts. Hence, even though the type $\texttt{Bool} \rightarrow \texttt{Nat}$ 
is consistent with $ ? \rightarrow \: ? $, 
it conflicts with the underlying type $\texttt{Nat} \rightarrow \texttt{Nat}$. 
On the other hand, lazy type checking approves this cast expression until 
it is applied to a boolean argument.

Another design choice for gradual typing is the addition of \textit{\textbf{blame assignment}} 
to identify the culprit coercion. It is clear in the expression (1) that the cast 
$\langle \texttt{Bool} \rightarrow \texttt{Nat} \Leftarrow 
\: ? \rightarrow \: ? \rangle$ is to blame because any cast from a 
dynamic type to a specific type is always unsafe. In a more complicated program 
when there can be multiple downcasts from ?, it is helpful to inform 
the programmer of the expressions responsible for coercion failures. Siek et al. [8] 
discuss the two blame assignment strategies: (1) UD shares the blame between 
upcasts and downcasts, whereas (2) D places the blame on downcasts only. UD and D 
differ by how a function type is coerced to ?, which will be discussed 
further in later sections.

In this paper, we present an interpretation of $\lambda ^? _{\rightarrow}$ 
using the eager coercion calculus with UD blame assignment. We first review the syntax 
and type system of $\lambda ^? _{\rightarrow}$ in the next section. 
Section 3 discusses the UD blame tracking strategy, and Section 4 integrates this 
feature into the formal definition of the coercion calculus. We then describe the 
implementation of our interpreter for $\lambda ^? _{\rightarrow}$ using Haskell. 
The last two sections put our work into context.
\section{Gradually-Typed Lambda Calculus $\lambda ^? _{\rightarrow}$} The $\lambda ^? _{\rightarrow}$ syntax is essentially the same as 
the simply-typed $\lambda$-calculus with references, except for the 
addition of dynamic types ?.
\begin{alignat*}{3}
    &(\textit{expressions}) \qquad 
    & e ::= \: & x \; | \;  
          e \: e \; | \;
          \lambda x \! : \! T . \: e \; | \;
          k \\
    &&    & \texttt{succ} \: e \; | \;
          \texttt{pred} \: e \; | \;
          \texttt{iszero} \: e \; | \;
          \texttt{if} \: e \: \texttt{then} \: e \: \texttt{else} \: e \; | \; \\
    &&    & \texttt{ref} \: e \; | \;
          !e \; | \;
          e := e \\
    &(\textit{constants}) 
    & k ::= \: & n \; | \; \texttt{true} \; | \; \texttt{false} \\
    &(\textit{numbers}) 
    & n ::= \: & 0 \; | \; \texttt{succ} \: n \\
    &(\textit{types}) 
    & T ::= \: & \texttt{Bool} \; | \;
            \texttt{Nat} \; | \;
            T \rightarrow T \; | \;
            \texttt{Ref} \: T \; | \;
            ?     
\end{alignat*}

Expressions include variables, abstractions, and applications, as well as 
arithmethic, boolean, and conditional terms found in the untyped $\lambda$-calculus. 
Reference expressions consist of allocation, dereference, and assignment.
Types include the base types (\texttt{Bool} and \texttt{Nat}), function 
types, reference types, and finally dynamic types. 
Constants like \texttt{true} and \texttt{false} inherently  
have \texttt{Bool} type, whereas numerical values are of type \texttt{Nat}. 
Function parameters without type annotations automatically 
assume dynamic types. 

The type system of $\lambda ^? _{\rightarrow}$ is based on the 
consistency relation between known and unknown types, given in 
Figure 1. Note that the consistency relation is reflexive and 
symmetric, but not transitive. Given $S \sim \: ?$ and $? \: \sim T$, it 
is not always the case that $S \sim T$.  

\begin{figure}[h]
    \caption{Type Consistency}
    \hrule
    \vspace{4mm}
    \[
        \begin{bprooftree}
            \AxiomC{}
            \RightLabel{\scriptsize{\text{[C-REFL]}}}
            \UnaryInfC{$T \sim T$}
        \end{bprooftree}
        \begin{bprooftree}
            \AxiomC{}
            \RightLabel{\scriptsize{\text{[C-DYNR]}}}
            \UnaryInfC{$T \sim \: ?$}
        \end{bprooftree}
        \begin{bprooftree}
            \AxiomC{}
            \RightLabel{\scriptsize{\text{[C-DYNL]}}}
            \UnaryInfC{$? \: \sim T$}
        \end{bprooftree}
    \]
    \vspace{1mm}
    \[
        \begin{bprooftree}
            \AxiomC{$S_1 \sim T_1$}
            \AxiomC{$S_2 \sim T_2$}
            \RightLabel{\scriptsize{\text{[C-FUN]}}}
            \BinaryInfC{$(S_1 \rightarrow S_2) \sim (T_1 \rightarrow  T_2)$}
        \end{bprooftree}
        \begin{bprooftree}
            \AxiomC{$S \sim T$}
            \RightLabel{\scriptsize{\text{[C-REF]}}}
            \UnaryInfC{$\texttt{Ref} \: S \sim \texttt{Ref} \: T$}
        \end{bprooftree}
    \]
    \hrule
\end{figure} 

Figure 2 presents the type system of $\lambda ^? _{\rightarrow}$. Rules 
{\scriptsize{[T-VAR]}}, {\scriptsize{[T-CONST]}}, {\scriptsize{[T-FUN]}}, 
and {\scriptsize{[T-REF]}}  
are straightforward. The rules for function applications handle either 
known or unknown function types. Given a specific function type, an 
application only accepts an argument type that is consistent to the 
parameter type. Otherwise, the argument can have any type, and the return 
type of the application is unknown. Similarly, there are two assignment 
rules for whether the left-hand side is dynamically-typed. 
A dereference expression only has a specific type if its argument is a 
reference. \begin{figure}[ht]
    \caption{$\lambda ^? _{\rightarrow}$ Type System}
    \hrule
    \[
        \begin{bprooftree}
            \AxiomC{$(x\!:\!T) \in \Gamma$}
            \RightLabel{\scriptsize{\text{[T-VAR]}}}
            \UnaryInfC{$\Gamma \vdash x : T$}
        \end{bprooftree}
        \begin{bprooftree}
            \AxiomC{$\Delta k = T$}
            \RightLabel{\scriptsize{\text{[T-CONST]}}}
            \UnaryInfC{$\Gamma \vdash k : T$}
        \end{bprooftree}
        \begin{bprooftree}
            \AxiomC{$\Gamma,x\!:\!S \vdash e : T$}
            \RightLabel{\scriptsize{\text{[T-FUN]}}}
            \UnaryInfC{$\Gamma \vdash \lambda x\!:\!S. \: e : S \rightarrow T$}
        \end{bprooftree}
    \]
    \vspace{1mm}
    \[
        \begin{bprooftree}
                \AxiomC{$\Gamma \vdash e : S$}
                \AxiomC{$S \sim \texttt{Nat}$}
                \RightLabel{\scriptsize{\text{[T-SUCC]}}}
                \BinaryInfC{$\Gamma \vdash \texttt{succ} \: e : \texttt{Nat}$}
        \end{bprooftree}
        \begin{bprooftree}
            \AxiomC{$\Gamma \vdash e : S$}
            \AxiomC{$S \sim \texttt{Nat}$}
            \RightLabel{\scriptsize{\text{[T-PRED]}}}
            \BinaryInfC{$\Gamma \vdash \texttt{pred} \: e : \texttt{Nat}$}
        \end{bprooftree}
    \]
    \vspace{1mm}
    \[
        \begin{bprooftree}
            \AxiomC{$\Gamma \vdash e : S$}
            \AxiomC{$S \sim \texttt{Nat}$}
            \RightLabel{\scriptsize{\text{[T-ISZERO]}}}
            \BinaryInfC{$\Gamma \vdash \texttt{iszero} \: e : \texttt{Bool}$}
        \end{bprooftree}
    \]
    \vspace{1mm}
    \[
        \begin{bprooftree}
                \AxiomC{$\Gamma \vdash e_1 : B$}
                \AxiomC{$B \sim \texttt{Bool}$}
                \AxiomC{$\Gamma \vdash e_2 : S$}
                \AxiomC{$\Gamma \vdash e_3 : T$}
                \AxiomC{$S \sim T$}
                \RightLabel{\scriptsize{\text{[T-IF]}}}
                \QuinaryInfC{$\Gamma \vdash \texttt{if} \: e_1 \: \texttt{then} \: e_2 \: 
                \texttt{else} \: e_3 : \: S$}
        \end{bprooftree}
    \]
    \vspace{1mm}
    \[
        \begin{bprooftree}
            \AxiomC{$\Gamma \vdash e_1 : S \rightarrow T$}
            \AxiomC{$\Gamma \vdash e_2 : S'$}
            \AxiomC{$S' \sim S$}
            \RightLabel{\scriptsize{\text{[T-APP1]}}}
            \TrinaryInfC{$\Gamma \vdash e_1 \: e_2 : T$}
        \end{bprooftree}
    \]
    \vspace{1mm}
    \[
        \begin{bprooftree}
            \AxiomC{$\Gamma \vdash e_1 : \: ?$}
            \AxiomC{$\Gamma \vdash e_2 : S$}
            \RightLabel{\scriptsize{\text{[T-APP2]}}}
            \BinaryInfC{$\Gamma \vdash e_1 \: e_2 : \: ?$}
        \end{bprooftree}
        \begin{bprooftree}
            \AxiomC{$\Gamma \vdash e : T$}
            \RightLabel{\scriptsize{\text{[T-REF]}}}
            \UnaryInfC{$\Gamma \vdash \texttt{ref} \: e : \texttt{Ref} \: T$}
        \end{bprooftree}
    \]
    \vspace{1mm}
    \[
        \begin{bprooftree}
            \AxiomC{$\Gamma \vdash e : \texttt{Ref} \: T$}
            \RightLabel{\scriptsize{\text{[T-DEREF1]}}}
            \UnaryInfC{$\Gamma \vdash \: !e : T$}
        \end{bprooftree}
        \begin{bprooftree}
            \AxiomC{$\Gamma \vdash e : \: ?$}
            \RightLabel{\scriptsize{\text{[T-DEREF2]}}}
            \UnaryInfC{$\Gamma \vdash \: !e : \: ?$}
        \end{bprooftree}
    \]
    \vspace{1mm}
    \[
        \begin{bprooftree}
            \AxiomC{$\Gamma \vdash e_1 : \texttt{Ref} \: T$}
            \AxiomC{$\Gamma \vdash e_2 : S$}
            \AxiomC{$S \sim T$}
            \RightLabel{\scriptsize{\text{[T-ASSIGN1]}}}
            \TrinaryInfC{$\Gamma \vdash e_1 := e_2 : S$}
        \end{bprooftree}
    \]
    \vspace{1mm}
    \[
        \begin{bprooftree}
            \AxiomC{$\Gamma \vdash e_1 : \: ?$}
            \AxiomC{$\Gamma \vdash e_2 : T$}
            \RightLabel{\scriptsize{\text{[T-ASSIGN2]}}}
            \BinaryInfC{$\Gamma \vdash e_1 := e_2 : \: ?$}
        \end{bprooftree}
    \]
    \hrule
    \end{figure}   We treat arithmetic 
and boolean expressions ($\texttt{succ} \: e$, $\texttt{pred} \: e$, $\texttt{iszero} \: e$) 
as constant functions that accept any type consistent to \texttt{Nat} and 
return either a \texttt{Nat} or a \texttt{Bool}. Conditional expressions are more 
complicated because we have to typecheck both the condition and the two 
branches. The condition type must be consistent to 
\texttt{Bool}. The types of two branches need to be consistent. If so, the 
resulting type can derive from either branch type.

\section{The UD Blame Assignment} Casts can be illegal, safe or unsafe. Casts are immediately rejected 
when the source and target types are inconsistent. A safe cast 
$\langle T \Leftarrow S \rangle$ respects 
the subtype relation $S <: T$, and therefore should not be blamed. 
On the other hand, any downcast from a supertype is considered unsafe. 
Our gradual type system keeps track of unsafe casts and identifies the 
one that is responsible for a cast error. 

As briefly mentioned in the \textit{Introduction}, there are two blame 
tracking strategies depending on the way that a function type is coerced 
to ?. The coercion calculus presented in Herman et al. [4] 
incorporates the function dynamic type $ ? \rightarrow \: ? $ 
to describe more specifically a function type with dynamic argument and return 
types. Accordingly, any coercion from a precise function type $S \rightarrow T$ 
to a general dynamic type ? has to check first if type ? 
is a function type. Consider the following evaluation for an expression 
with labeled casts:
\begin{align*}
    & (\langle ? \rightarrow \texttt{Nat} \Leftarrow 
    \: ? \rangle ^{l_1}
    \langle ? \Leftarrow 
    \texttt{Nat} \rightarrow \texttt{Nat} \rangle ^{l_2}
    (\lambda x \! : \! \texttt{Nat} . \; x)) 
    \langle ? \Leftarrow \texttt{Bool} \rangle ^{l_3} \texttt{true} \\
    \longrightarrow \:
    & (\langle ? \rightarrow \texttt{Nat} \Leftarrow 
    \: ? \rightarrow \: ? \rangle ^{l_1}
    \langle ? \rightarrow \: ? \Leftarrow 
    \texttt{Nat} \rightarrow \texttt{Nat} \rangle ^{l_2}
    (\lambda x \! : \! \texttt{Nat} . \; x)) 
    \langle ? \Leftarrow \texttt{Bool} \rangle ^{l_3} \texttt{true} \\
    \longrightarrow \:
    & \langle \texttt{Nat} \Leftarrow \: ? \rangle ^{l_1} 
    (\langle ? \rightarrow \: ? \Leftarrow 
    \texttt{Nat} \rightarrow \texttt{Nat} \rangle ^{l_2}
    (\lambda x \! : \! \texttt{Nat} . \; x))
    \langle ? \Leftarrow \: ? \rangle ^{l_1} 
    \langle ? \Leftarrow \texttt{Bool} \rangle ^{l_3} \texttt{true} \\
    \longrightarrow \:
    & \langle \texttt{Nat} \Leftarrow \: ? \rangle ^{l_1} 
    (\langle ? \rightarrow \: ? \Leftarrow 
    \texttt{Nat} \rightarrow \texttt{Nat} \rangle ^{l_2}
    (\lambda x \! : \! \texttt{Nat} . \; x)) 
    \langle ? \Leftarrow \texttt{Bool} \rangle ^{l_3} \texttt{true} \\
    \longrightarrow \: 
    & \langle \texttt{Nat} \Leftarrow \: ? \rangle ^{l_1} 
    \langle ? \Leftarrow \texttt{Nat} \rangle ^{l_2} 
    ((\lambda x \! : \! \texttt{Nat} . \; x) 
    \highlight{\langle \texttt{Nat} \Leftarrow \: ? \rangle ^{l_2}}
    \langle ? \Leftarrow \texttt{Bool} \rangle ^{l_3} \texttt{true}) \\
    \longrightarrow \:
    & \textbf{blame} \: l_2
\end{align*}
The blame is shared between the downcast from ? to \texttt{Nat} in the argument type 
and the upcast from \texttt{Nat} to ? in the return type for $\lambda x \! : \! \texttt{Nat} . \; x$. 
This type of blame assignment is called UD. 
However, the cast $\langle ? \Leftarrow \texttt{Nat} \rightarrow \texttt{Nat} \rangle ^{l_2}$ 
is considered a safe cast based on the traditional subtype relation where 
? is the top element. The intermediate step of converting type ? to 
the dynamic function type $ ? \rightarrow \: ?$ changes $\langle ? \Leftarrow \texttt{Nat} 
\rightarrow \texttt{Nat} \rangle ^{l_2}$ to
$\langle ? \rightarrow \: ? \Leftarrow \texttt{Nat} \rightarrow \texttt{Nat} \rangle ^{l_2}$. 
$\texttt{Nat} \rightarrow \texttt{Nat} <: \: ?$ is true, but    
$\texttt{Nat} \rightarrow \texttt{Nat} <: \: ? \rightarrow \: ?$ is not. Siek et al. [4] 
present a modified subtype relation (defined in Figure 3) to prove the soundness of the UD 
blame assignment. In this way, ? is no longer the top element, and a function type $S \rightarrow T$ 
is only a subtype of ? if it is a subtype of $ ? \rightarrow \: ?$. 
\begin{figure}[ht]
    \caption{Subtype Relations}
    \hrule
    \[
        \begin{bprooftree}
            \AxiomC{}
            \UnaryInfC{$B <: B$}
        \end{bprooftree}
        \begin{bprooftree}
            \AxiomC{}
            \UnaryInfC{$? <: \: ?$}
        \end{bprooftree}
        \begin{bprooftree}
            \AxiomC{$S <: B$}
            \UnaryInfC{$S <: \: ?$}
        \end{bprooftree}
        \begin{bprooftree}
            \AxiomC{$S <: \: ? \rightarrow \: ?$}
            \UnaryInfC{$S <: \: ?$}
        \end{bprooftree}
        \begin{bprooftree}
            \AxiomC{$T_1 <: S_1$}
            \AxiomC{$S_2 <: T_2$}
            \BinaryInfC{$S_1 \rightarrow S_2 <: T_1 \rightarrow T_2$}
      \end{bprooftree}
    \] 
    \begin{align*}
        \text{where} \: B ::= \texttt{Bool} \: | \: \texttt{Nat}
    \end{align*}
    \hrule
\end{figure} 
\section{The Eager Coercion Calculus} \subsection{Coercions}
Herman et al. [4] represent casts as combinable coercions to reduce 
space comsumption of gradual typing while preserving its semantics. 
\subsection{Coercions}
Herman et al. [4] represent casts as combinable coercions to reduce 
space comsumption of gradual typing while preserving its semantics. 
\subsection{Coercions}
Herman et al. [4] represent casts as combinable coercions to reduce 
space comsumption of gradual typing while preserving its semantics. 
\input{figures/Coercion}
The syntax and type system for the coercion calculus are defined in 
Figure 4 [4]. Note that we add blame labels to certain coercions to 
enable blame tracking in our gradual type system. The judgment $\vdash c: S 
\rightsquigarrow T$ indicates that coercion $c$ casts a value of type 
$S$ to type $T$. The identity $I$ is a coercion of any type to itself. 
$D!$ injects a type into ? while $D?^l$ projects ? into a precise type. 
Since an injection is always safe, we do not need to blame-track it. 
A coercion of two inconsistent types results in $\texttt{Fail}^l$. 

As discussed in the previous section, UD blame assignment arises  
from the coercion calculus because a coercion 
between a function type and ? always goes through the 
dynamic function type $? \rightarrow \: ?$. Thus,
the coercion $? \rightsquigarrow (S \rightarrow T)$ is an 
abstraction of two coercions: $? \rightsquigarrow 
(? \rightarrow \: ?)$ first to make sure ? is a function type 
and then $(? \rightarrow \: ?) \rightsquigarrow 
(S \rightarrow T)$. Similarly, the coercion $(S \rightarrow T) \rightsquigarrow \: ?$ 
performs the coercion $(S \rightarrow T) \rightsquigarrow (? \rightarrow \: ?)$ 
followed by $(? \rightarrow \: ?) \rightsquigarrow \: ?$ to 
generalize a function dynamic type.  
The coercion calculus uses the indirection $\texttt{Fun}?^l$ and $\texttt{Fun}!$ 
to represent the injection $(? \rightarrow \: ?) \rightsquigarrow \: ?$ and 
projection $? \rightsquigarrow (? \rightarrow \: ?)$. The same concept is 
applied to reference coercions. The injection $\texttt{Ref}!$ and 
projection $\texttt{Ref}?^l$ are auxiliary to the coercions $\texttt{Ref} \: T 
\rightsquigarrow \: ?$ and $? \rightsquigarrow \texttt{Ref} \: T$.

Multiple coercions can behave as one. The function coercion $\texttt{Fun} \: c \: d$ 
applies coercion $c$ to the argument and $d$ to the result. Coercing 
two function types $(S_1 \rightarrow S_2) \rightsquigarrow (T_1 \rightarrow T_2)$ 
coerces $T_1$ to $S_1$ and $S_2$ to $T_2$. Likewise, the reference coercion 
$\texttt{Ref} \: S \rightsquigarrow \texttt{Ref} \: T$ results in 
$\texttt{Ref} \: c \: d$, where $c$ coerces all writes from 
$T$ to $S$ and $d$ coerces all reads from $S$ to $T$. Finally, the coercion 
composition $c;d$ applies each coercion from left to right.

\subsection{Reduction Rules}
The bounded space consumption of coercions derives from their reducibility. 
The core set of normalization rules is given in Figure 5. These rules model 
after those of Herman et al. [4] with some additions from Siek and Garcia [5]. 
The added rules are needed for blame assignment and its confluence. First, 
we omit the rule 
\begin{equation*}
    c;\texttt{Fail}^l = \texttt{Fail}^l
\end{equation*}
to ensure that blame should always come from the closest coercion to an 
expression, unless it is an injection. For instance, $\langle \texttt{Nat}?^{l_1};\texttt{Fail}^{l_2} 
\rangle \langle \texttt{Bool}! \rangle \texttt{true} \longrightarrow 
\langle \texttt{Bool}!;\texttt{Nat}?^{l_1};\texttt{Fail}^{l_2} \rangle \texttt{true}
\longrightarrow \langle \texttt{Fail}^{l_1};\texttt{Fail}^{l_2} \rangle \texttt{true}
\longrightarrow \texttt{Fail}^{l_1}$. Here the blame is assigned to the 
most inner coercion $l_1$, instead of $l_2$ 
if $\texttt{Fail}^{l_2}$ is allowed to propagate from the right. 

Coercions can be reassociated to enable further reduction. A coercion $c$ that 
cannot be further reduced is normalized, represented as $\tilde{c}$. Figure 
5 shows a list of normal coercions. Note that a composition of normalized 
coercion is only normal if neither is the identity $I$. Also, we omit 
the normalized coercion $\texttt{Fail}^l$, which is handled separately 
during evaluation. 

\input{figures/Reduction}

\subsection{Eager Error Detection}
Let us revisit the expression described in the \textit{Introduction}:
\begin{equation*}
    \langle \texttt{Bool} \rightarrow \texttt{Nat} \Leftarrow  
    \: ? \rightarrow \: ? \rangle ^l 
    \langle ? \rightarrow \: ? \Leftarrow 
    \texttt{Nat} \rightarrow \texttt{Nat} \rangle
    (\lambda x \! : \! \texttt{Nat} . \; x)
\end{equation*}
Observe that this cast would fail because type $\texttt{Bool} \rightarrow 
\texttt{Nat}$ is inconsistent with type $\texttt{Nat} \rightarrow 
\texttt{Nat}$. However, this error would not be detected in the traditional 
cast checking system until $(\lambda x \! : \! \texttt{Nat} . \; x)$ is 
applied to a boolean. On the other hand, our gradual type system would not 
perform any function application on this expression until its coercions are 
normalized. Using one of the core reduction rules, we obtain:
\begin{equation*}
    \langle \texttt{Fun} \: \texttt{Fail}^l \: I \rangle
    (\lambda x \! : \! \texttt{Nat} . \; x)
\end{equation*}
where $\texttt{Fail}^l$ results from the attempt to coerce $\texttt{Nat} \rightarrow 
\texttt{Nat}$ to $\texttt{Bool} \rightarrow \texttt{Nat}$. As a result, 
the evaluation process should effectively terminate to a cast error before 
performing any more unnecessary operations. In order to achieve this 
early error detection, we extend the coercion normalization rules with 
four eager patterns (see Figure 5) to propagate cast errors. These added rules are similar to 
those given by Herman et al. [4] but require the coercion in the argument 
position to be normalized when $\texttt{Fail}^l$ is in the result [8]. 
This restriction makes sure that a failed argument coercion takes the blame when 
the result coercion fails too. Failed reference coercions follow an 
analogous pattern: blame only goes to a failed ``read'' coercion if 
the ``write'' coercion is normal and not a failure.

We have covered the design specifications for gradual typing based on coercions 
with UD blame assignment. In the next section, we will apply this design choice 
to implement a $\lambda ^? _{\rightarrow}$ interpreter in Haskell. 
The syntax and type system for the coercion calculus are defined in 
Figure 4 [4]. Note that we add blame labels to certain coercions to 
enable blame tracking in our gradual type system. The judgment $\vdash c: S 
\rightsquigarrow T$ indicates that coercion $c$ casts a value of type 
$S$ to type $T$. The identity $I$ is a coercion of any type to itself. 
$D!$ injects a type into ? while $D?^l$ projects ? into a precise type. 
Since an injection is always safe, we do not need to blame-track it. 
A coercion of two inconsistent types results in $\texttt{Fail}^l$. 

As discussed in the previous section, UD blame assignment arises  
from the coercion calculus because a coercion 
between a function type and ? always goes through the 
dynamic function type $? \rightarrow \: ?$. Thus,
the coercion $? \rightsquigarrow (S \rightarrow T)$ is an 
abstraction of two coercions: $? \rightsquigarrow 
(? \rightarrow \: ?)$ first to make sure ? is a function type 
and then $(? \rightarrow \: ?) \rightsquigarrow 
(S \rightarrow T)$. Similarly, the coercion $(S \rightarrow T) \rightsquigarrow \: ?$ 
performs the coercion $(S \rightarrow T) \rightsquigarrow (? \rightarrow \: ?)$ 
followed by $(? \rightarrow \: ?) \rightsquigarrow \: ?$ to 
generalize a function dynamic type.  
The coercion calculus uses the indirection $\texttt{Fun}?^l$ and $\texttt{Fun}!$ 
to represent the injection $(? \rightarrow \: ?) \rightsquigarrow \: ?$ and 
projection $? \rightsquigarrow (? \rightarrow \: ?)$. The same concept is 
applied to reference coercions. The injection $\texttt{Ref}!$ and 
projection $\texttt{Ref}?^l$ are auxiliary to the coercions $\texttt{Ref} \: T 
\rightsquigarrow \: ?$ and $? \rightsquigarrow \texttt{Ref} \: T$.

Multiple coercions can behave as one. The function coercion $\texttt{Fun} \: c \: d$ 
applies coercion $c$ to the argument and $d$ to the result. Coercing 
two function types $(S_1 \rightarrow S_2) \rightsquigarrow (T_1 \rightarrow T_2)$ 
coerces $T_1$ to $S_1$ and $S_2$ to $T_2$. Likewise, the reference coercion 
$\texttt{Ref} \: S \rightsquigarrow \texttt{Ref} \: T$ results in 
$\texttt{Ref} \: c \: d$, where $c$ coerces all writes from 
$T$ to $S$ and $d$ coerces all reads from $S$ to $T$. Finally, the coercion 
composition $c;d$ applies each coercion from left to right.

\subsection{Reduction Rules}
The bounded space consumption of coercions derives from their reducibility. 
The core set of normalization rules is given in Figure 5. These rules model 
after those of Herman et al. [4] with some additions from Siek and Garcia [5]. 
The added rules are needed for blame assignment and its confluence. First, 
we omit the rule 
\begin{equation*}
    c;\texttt{Fail}^l = \texttt{Fail}^l
\end{equation*}
to ensure that blame should always come from the closest coercion to an 
expression, unless it is an injection. For instance, $\langle \texttt{Nat}?^{l_1};\texttt{Fail}^{l_2} 
\rangle \langle \texttt{Bool}! \rangle \texttt{true} \longrightarrow 
\langle \texttt{Bool}!;\texttt{Nat}?^{l_1};\texttt{Fail}^{l_2} \rangle \texttt{true}
\longrightarrow \langle \texttt{Fail}^{l_1};\texttt{Fail}^{l_2} \rangle \texttt{true}
\longrightarrow \texttt{Fail}^{l_1}$. Here the blame is assigned to the 
most inner coercion $l_1$, instead of $l_2$ 
if $\texttt{Fail}^{l_2}$ is allowed to propagate from the right. 

Coercions can be reassociated to enable further reduction. A coercion $c$ that 
cannot be further reduced is normalized, represented as $\tilde{c}$. Figure 
5 shows a list of normal coercions. Note that a composition of normalized 
coercion is only normal if neither is the identity $I$. Also, we omit 
the normalized coercion $\texttt{Fail}^l$, which is handled separately 
during evaluation. 

\begin{figure}[ht]
    \caption{Coercion Normalization}
    \hrule
    \vspace{4mm}
    \qquad \text{\underline{N}ormal Coercions:}
    \begin{alignat*}{3}
        &(\textit{normal coercions}) \qquad 
        & \tilde{c} ::= \: & I \; | \;  
              B?^l \; | \;
              B! \; | \; 
              \hat{c};B! \; | \; 
              B?^l;\hat{c} \; | \; 
              B?^l; \texttt{Fail}^l \; | \; \\
        &&    & \texttt{Fun} \: \hat{c} \: \hat{d} \; | \;
              \hat{c};\texttt{Fun}! \; | \; 
              \texttt{Fun}?^l;\hat{c} \; | \; 
              \texttt{Fun} \: \hat{c} \: \hat{d}; \texttt{Fail}^l \; | \; \\
        &&    & \texttt{Ref} \: \hat{c} \: \hat{d} \; | \; 
              \texttt{Ref} \: \hat{c} \: \hat{d}; \texttt{Fail}^l \; | \;
              \hat{c};\texttt{Ref}! \; | \;
              \texttt{Ref}?^l;\hat{c} \\
        &(\textit{regular coercions}) \qquad
        & \hat{c} ::= \: & \tilde{c} \: \: \text{where} \: \: \tilde{c} \neq I
    \end{alignat*}
    \qquad \text{\underline{C}ore Reduction:}
    \begin{alignat*}{3} 
        I;c & = c \hspace{1cm} & \texttt{Fun} \: I \: I & = I \\
        c;I & = c \hspace{1cm} & \texttt{Ref} \: I \: I & = I \\
        D!;D?^l & = I \hspace{1cm} & c_1;(\tilde{c_2;c_3}) & = (c_1;c_2);c_3 \\
        D!;\texttt{Fail}^l & = \texttt{Fail}^l \hspace{1cm} & (\tilde{c_1;c_2});c_3 & = c_1;(c_2;c_3) \\
        \texttt{Fail}^l;c & = \texttt{Fail}^l \hspace{1cm} & (\texttt{Fun} \: \tilde{c_1} \: \tilde{c_2});(\texttt{Fun} \: \tilde{d_1} \: \tilde{d_2}) & = 
        \texttt{Fun} \: (\tilde{d_1};\tilde{c_1}) \: (\tilde{c_2};\tilde{d_2}) \\
        D!;D'?^l & = \texttt{Fail}^l \hspace{1cm} & (\texttt{Ref} \: \tilde{c_1} \: \tilde{c_2});(\texttt{Ref} \: \tilde{d_1} \: \tilde{d_2}) & = 
        \texttt{Ref} \: (\tilde{d_1};\tilde{c_1}) \: (\tilde{c_2};\tilde{d_2})
    \end{alignat*}
    \qquad \text{\underline{E}ager Reduction:}
    \begin{alignat*}{3} 
        \texttt{Fun} \: \texttt{Fail}^l \: d & = \texttt{Fail}^l \hspace{1cm} & 
        \texttt{Ref} \: \texttt{Fail}^l \: d & = \texttt{Fail}^l \\
        \texttt{Fun} \: \tilde{c} \: \texttt{Fail}^l & = \texttt{Fail}^l \hspace{1cm} & 
        \texttt{Ref} \: \tilde{c} \: \texttt{Fail}^l & = \texttt{Fail}^l 
    \end{alignat*}
    \hrule
\end{figure} 

\subsection{Eager Error Detection}
Let us revisit the expression described in the \textit{Introduction}:
\begin{equation*}
    \langle \texttt{Bool} \rightarrow \texttt{Nat} \Leftarrow  
    \: ? \rightarrow \: ? \rangle ^l 
    \langle ? \rightarrow \: ? \Leftarrow 
    \texttt{Nat} \rightarrow \texttt{Nat} \rangle
    (\lambda x \! : \! \texttt{Nat} . \; x)
\end{equation*}
Observe that this cast would fail because type $\texttt{Bool} \rightarrow 
\texttt{Nat}$ is inconsistent with type $\texttt{Nat} \rightarrow 
\texttt{Nat}$. However, this error would not be detected in the traditional 
cast checking system until $(\lambda x \! : \! \texttt{Nat} . \; x)$ is 
applied to a boolean. On the other hand, our gradual type system would not 
perform any function application on this expression until its coercions are 
normalized. Using one of the core reduction rules, we obtain:
\begin{equation*}
    \langle \texttt{Fun} \: \texttt{Fail}^l \: I \rangle
    (\lambda x \! : \! \texttt{Nat} . \; x)
\end{equation*}
where $\texttt{Fail}^l$ results from the attempt to coerce $\texttt{Nat} \rightarrow 
\texttt{Nat}$ to $\texttt{Bool} \rightarrow \texttt{Nat}$. As a result, 
the evaluation process should effectively terminate to a cast error before 
performing any more unnecessary operations. In order to achieve this 
early error detection, we extend the coercion normalization rules with 
four eager patterns (see Figure 5) to propagate cast errors. These added rules are similar to 
those given by Herman et al. [4] but require the coercion in the argument 
position to be normalized when $\texttt{Fail}^l$ is in the result [8]. 
This restriction makes sure that a failed argument coercion takes the blame when 
the result coercion fails too. Failed reference coercions follow an 
analogous pattern: blame only goes to a failed ``read'' coercion if 
the ``write'' coercion is normal and not a failure.

We have covered the design specifications for gradual typing based on coercions 
with UD blame assignment. In the next section, we will apply this design choice 
to implement a $\lambda ^? _{\rightarrow}$ interpreter in Haskell. 
The syntax and type system for the coercion calculus are defined in 
Figure 4 [4]. Note that we add blame labels to certain coercions to 
enable blame tracking in our gradual type system. The judgment $\vdash c: S 
\rightsquigarrow T$ indicates that coercion $c$ casts a value of type 
$S$ to type $T$. The identity $I$ is a coercion of any type to itself. 
$D!$ injects a type into ? while $D?^l$ projects ? into a precise type. 
Since an injection is always safe, we do not need to blame-track it. 
A coercion of two inconsistent types results in $\texttt{Fail}^l$. 

As discussed in the previous section, UD blame assignment arises  
from the coercion calculus because a coercion 
between a function type and ? always goes through the 
dynamic function type $? \rightarrow \: ?$. Thus,
the coercion $? \rightsquigarrow (S \rightarrow T)$ is an 
abstraction of two coercions: $? \rightsquigarrow 
(? \rightarrow \: ?)$ first to make sure ? is a function type 
and then $(? \rightarrow \: ?) \rightsquigarrow 
(S \rightarrow T)$. Similarly, the coercion $(S \rightarrow T) \rightsquigarrow \: ?$ 
performs the coercion $(S \rightarrow T) \rightsquigarrow (? \rightarrow \: ?)$ 
followed by $(? \rightarrow \: ?) \rightsquigarrow \: ?$ to 
generalize a function dynamic type.  
The coercion calculus uses the indirection $\texttt{Fun}?^l$ and $\texttt{Fun}!$ 
to represent the injection $(? \rightarrow \: ?) \rightsquigarrow \: ?$ and 
projection $? \rightsquigarrow (? \rightarrow \: ?)$. The same concept is 
applied to reference coercions. The injection $\texttt{Ref}!$ and 
projection $\texttt{Ref}?^l$ are auxiliary to the coercions $\texttt{Ref} \: T 
\rightsquigarrow \: ?$ and $? \rightsquigarrow \texttt{Ref} \: T$.

Multiple coercions can behave as one. The function coercion $\texttt{Fun} \: c \: d$ 
applies coercion $c$ to the argument and $d$ to the result. Coercing 
two function types $(S_1 \rightarrow S_2) \rightsquigarrow (T_1 \rightarrow T_2)$ 
coerces $T_1$ to $S_1$ and $S_2$ to $T_2$. Likewise, the reference coercion 
$\texttt{Ref} \: S \rightsquigarrow \texttt{Ref} \: T$ results in 
$\texttt{Ref} \: c \: d$, where $c$ coerces all writes from 
$T$ to $S$ and $d$ coerces all reads from $S$ to $T$. Finally, the coercion 
composition $c;d$ applies each coercion from left to right.

\subsection{Reduction Rules}
The bounded space consumption of coercions derives from their reducibility. 
The core set of normalization rules is given in Figure 5. These rules model 
after those of Herman et al. [4] with some additions from Siek and Garcia [5]. 
The added rules are needed for blame assignment and its confluence. First, 
we omit the rule 
\begin{equation*}
    c;\texttt{Fail}^l = \texttt{Fail}^l
\end{equation*}
to ensure that blame should always come from the closest coercion to an 
expression, unless it is an injection. For instance, $\langle \texttt{Nat}?^{l_1};\texttt{Fail}^{l_2} 
\rangle \langle \texttt{Bool}! \rangle \texttt{true} \longrightarrow 
\langle \texttt{Bool}!;\texttt{Nat}?^{l_1};\texttt{Fail}^{l_2} \rangle \texttt{true}
\longrightarrow \langle \texttt{Fail}^{l_1};\texttt{Fail}^{l_2} \rangle \texttt{true}
\longrightarrow \texttt{Fail}^{l_1}$. Here the blame is assigned to the 
most inner coercion $l_1$, instead of $l_2$ 
if $\texttt{Fail}^{l_2}$ is allowed to propagate from the right. 

Coercions can be reassociated to enable further reduction. A coercion $c$ that 
cannot be further reduced is normalized, represented as $\tilde{c}$. Figure 
5 shows a list of normal coercions. Note that a composition of normalized 
coercion is only normal if neither is the identity $I$. Also, we omit 
the normalized coercion $\texttt{Fail}^l$, which is handled separately 
during evaluation. 

\begin{figure}[ht]
    \caption{Coercion Normalization}
    \hrule
    \vspace{4mm}
    \qquad \text{\underline{N}ormal Coercions:}
    \begin{alignat*}{3}
        &(\textit{normal coercions}) \qquad 
        & \tilde{c} ::= \: & I \; | \;  
              B?^l \; | \;
              B! \; | \; 
              \hat{c};B! \; | \; 
              B?^l;\hat{c} \; | \; 
              B?^l; \texttt{Fail}^l \; | \; \\
        &&    & \texttt{Fun} \: \hat{c} \: \hat{d} \; | \;
              \hat{c};\texttt{Fun}! \; | \; 
              \texttt{Fun}?^l;\hat{c} \; | \; 
              \texttt{Fun} \: \hat{c} \: \hat{d}; \texttt{Fail}^l \; | \; \\
        &&    & \texttt{Ref} \: \hat{c} \: \hat{d} \; | \; 
              \texttt{Ref} \: \hat{c} \: \hat{d}; \texttt{Fail}^l \; | \;
              \hat{c};\texttt{Ref}! \; | \;
              \texttt{Ref}?^l;\hat{c} \\
        &(\textit{regular coercions}) \qquad
        & \hat{c} ::= \: & \tilde{c} \: \: \text{where} \: \: \tilde{c} \neq I
    \end{alignat*}
    \qquad \text{\underline{C}ore Reduction:}
    \begin{alignat*}{3} 
        I;c & = c \hspace{1cm} & \texttt{Fun} \: I \: I & = I \\
        c;I & = c \hspace{1cm} & \texttt{Ref} \: I \: I & = I \\
        D!;D?^l & = I \hspace{1cm} & c_1;(\tilde{c_2;c_3}) & = (c_1;c_2);c_3 \\
        D!;\texttt{Fail}^l & = \texttt{Fail}^l \hspace{1cm} & (\tilde{c_1;c_2});c_3 & = c_1;(c_2;c_3) \\
        \texttt{Fail}^l;c & = \texttt{Fail}^l \hspace{1cm} & (\texttt{Fun} \: \tilde{c_1} \: \tilde{c_2});(\texttt{Fun} \: \tilde{d_1} \: \tilde{d_2}) & = 
        \texttt{Fun} \: (\tilde{d_1};\tilde{c_1}) \: (\tilde{c_2};\tilde{d_2}) \\
        D!;D'?^l & = \texttt{Fail}^l \hspace{1cm} & (\texttt{Ref} \: \tilde{c_1} \: \tilde{c_2});(\texttt{Ref} \: \tilde{d_1} \: \tilde{d_2}) & = 
        \texttt{Ref} \: (\tilde{d_1};\tilde{c_1}) \: (\tilde{c_2};\tilde{d_2})
    \end{alignat*}
    \qquad \text{\underline{E}ager Reduction:}
    \begin{alignat*}{3} 
        \texttt{Fun} \: \texttt{Fail}^l \: d & = \texttt{Fail}^l \hspace{1cm} & 
        \texttt{Ref} \: \texttt{Fail}^l \: d & = \texttt{Fail}^l \\
        \texttt{Fun} \: \tilde{c} \: \texttt{Fail}^l & = \texttt{Fail}^l \hspace{1cm} & 
        \texttt{Ref} \: \tilde{c} \: \texttt{Fail}^l & = \texttt{Fail}^l 
    \end{alignat*}
    \hrule
\end{figure} 

\subsection{Eager Error Detection}
Let us revisit the expression described in the \textit{Introduction}:
\begin{equation*}
    \langle \texttt{Bool} \rightarrow \texttt{Nat} \Leftarrow  
    \: ? \rightarrow \: ? \rangle ^l 
    \langle ? \rightarrow \: ? \Leftarrow 
    \texttt{Nat} \rightarrow \texttt{Nat} \rangle
    (\lambda x \! : \! \texttt{Nat} . \; x)
\end{equation*}
Observe that this cast would fail because type $\texttt{Bool} \rightarrow 
\texttt{Nat}$ is inconsistent with type $\texttt{Nat} \rightarrow 
\texttt{Nat}$. However, this error would not be detected in the traditional 
cast checking system until $(\lambda x \! : \! \texttt{Nat} . \; x)$ is 
applied to a boolean. On the other hand, our gradual type system would not 
perform any function application on this expression until its coercions are 
normalized. Using one of the core reduction rules, we obtain:
\begin{equation*}
    \langle \texttt{Fun} \: \texttt{Fail}^l \: I \rangle
    (\lambda x \! : \! \texttt{Nat} . \; x)
\end{equation*}
where $\texttt{Fail}^l$ results from the attempt to coerce $\texttt{Nat} \rightarrow 
\texttt{Nat}$ to $\texttt{Bool} \rightarrow \texttt{Nat}$. As a result, 
the evaluation process should effectively terminate to a cast error before 
performing any more unnecessary operations. In order to achieve this 
early error detection, we extend the coercion normalization rules with 
four eager patterns (see Figure 5) to propagate cast errors. These added rules are similar to 
those given by Herman et al. [4] but require the coercion in the argument 
position to be normalized when $\texttt{Fail}^l$ is in the result [8]. 
This restriction makes sure that a failed argument coercion takes the blame when 
the result coercion fails too. Failed reference coercions follow an 
analogous pattern: blame only goes to a failed ``read'' coercion if 
the ``write'' coercion is normal and not a failure.

We have covered the design specifications for gradual typing based on coercions 
with UD blame assignment. In the next section, we will apply this design choice 
to implement a $\lambda ^? _{\rightarrow}$ interpreter in Haskell. 
\section{Interpreter for $\lambda ^? _{\rightarrow}$} The interpretation of $\lambda ^? _{\rightarrow}$ involves the following 
six stages:
\begin{gather*}
    \texttt{standard I/O} \xrightarrow{\text{chars}}
    \texttt{lexing} \xrightarrow{\text{tokens}} \texttt{parsing} \\
    \lhook\joinrel\xrightarrow{\text{terms}} \texttt{typechecking} 
    \xrightarrow{\text{cast terms}} \texttt{evaluation} 
    \xrightarrow{\text{value}} \texttt{pretty-printing}
\end{gather*}
where a program as a sequence of characters is first read from the 
standard input, tokenized by a lexical analyzer, and parsed into 
an abstract syntax tree. The program is then typechecked and 
modified with casts before being evaluated. The final 
result is then printed in a readable format to the standard output.  We start by 
defining the abstract syntax for the language.

\begin{lstlisting}
    data Term = Zero                       
               | Tru                        
               | Fls                        
               | Var Int Type String        
               | If Term Term Term          
               | Succ Term                   
               | Pred Term                  
               | IsZero Term                
               | Lambda Type Term [String]  
               | App Term Term              
               | Ref Term                   
               | Deref Term                 
               | Loc Int                    
               | Assign Term Term          
               | Cast Coercion Term   
\end{lstlisting}

Variables (\lstinline{Var Int Type String}) are represented as nameless terms using de Bruijn indices 
in order to avert the problem of capturing free variables. Accordingly, each variable 
is assigned a number to indicate the position of its binder. For example, 
$\lambda x. \: \lambda y. \: x \: y$ is rewritten as $\lambda. \: \lambda. 
\: 1 \: 0$. That is, $y$ is bound by the first binder, and $x$ the second. 
Variables whose de Bruijn indices are greater than the total number of binders 
are considered free. Also, instead of maintaining a global type environment 
$\Gamma$, we have each variable remember its own bound type. We also 
include the original name of a variable for later printing.

Note that this syntax extends upon the one we have formalized in Section 2 
with cast terms (\lstinline{Cast Coercion Term}) and  
store locations (\lstinline{Loc Int}). These terms only arise as intermediate 
results from typechecking and evaluation, and hence are not made available 
to programmers. Likewise, the data types for types and coercions are transcribed 
directly from their formal definitions, with one addition of type \lstinline{TUnit} 
used for free variables.

\begin{lstlisting}
    data Type = TUnit         
               | Dyn            
               | Boolean          
               | Nat            
               | Arr Type Type 
               | TRef Type      
\end{lstlisting}

\begin{lstlisting}
    data Coercion = Iden Type              
                  | Project Type Label     
                  | Inject Type             
                  | CRef Coercion Coercion  
                  | Func Coercion Coercion  
                  | Seq Coercion Coercion   
                  | Fail Type Type Label
\end{lstlisting}  

Given the syntax definition, the rest of this section will mainly focus on 
implementing the typechecker and 
evaluator for $\lambda ^? _{\rightarrow}$. The details on the 
lexer, parser, and pretty printer can be consulted in 
Appendix A, B, and C.

\subsection{Intermediate Language and Cast Insertion}
The static analysis of $\lambda ^? _{\rightarrow}$ both 
typechecks annotated terms and inserts coercions for 
dynamically-typed terms to ensure type safety. The cast insertion 
rules, shown in Figure 6 [8], resemble the typing rules in 
Figure 2. 
The rules for variables, constants, and 
abstractions do not require any cast insertion. Rules {\scriptsize{[C-SUCC2]}}, 
{\scriptsize{[C-PRED2]}}, and {\scriptsize{[C-ISZERO2]}} handle the 
case when the argument type is dynamic by coercing it to \texttt{Nat}. 
The rules for \texttt{if} expressions make sure that the condition 
type is coerced to \texttt{Bool} if dynamic, and the two branch 
types are consistent by coercing one to another. Rule 
{\scriptsize{[C-APP1]}} coerces the argument type to the parameter 
type given that the operator is a function. Otherwise, we need to 
insert a function check $\langle \texttt{Fun}? \rangle$ to verify 
that the operator resolves to a function at runtime. The rules for 
assignment follow the same pattern, casting the right-hand side type 
to the left-hand side and inserting a reference check 
$\langle \texttt{Ref}? \rangle$ when the left-hand side type is 
unknown. A reference check is also important when dereferencing a 
dynamic term, as in rule {\scriptsize{[C-DEREF2]}}.
\begin{figure}[hp]
    \caption{Cast Insertion Rules}
    \hrule
    \[
        \begin{bprooftree}
            \AxiomC{$(x\!:\!T) \in \Gamma$}
            \RightLabel{\scriptsize{\text{[C-VAR]}}}
            \UnaryInfC{$\Gamma \vdash x \hookrightarrow x : T$}
        \end{bprooftree}
        \begin{bprooftree}
            \AxiomC{$\Delta k = T$}
            \RightLabel{\scriptsize{\text{[C-CONST]}}}
            \UnaryInfC{$\Gamma \vdash k \hookrightarrow k : T$}
        \end{bprooftree}
    \]
    \vspace{0.5mm}
    \[
        \begin{bprooftree}
            \AxiomC{$\Gamma,x\!:\!S \vdash e \hookrightarrow t : T$}
            \RightLabel{\scriptsize{\text{[C-FUN]}}}
            \UnaryInfC{$\Gamma \vdash (\lambda x\!:\!S. \: e) 
            \hookrightarrow (\lambda x\!:\!S. \: t) : (S \rightarrow T)$}
        \end{bprooftree}
    \]
    \vspace{0.5mm}
    \[
        \begin{bprooftree}
            \AxiomC{$\Gamma \vdash e \hookrightarrow t : \texttt{Nat}$}
            \RightLabel{\scriptsize{\text{[C-SUCC1]}}}
            \UnaryInfC{$\Gamma \vdash \texttt{succ} \: e \hookrightarrow \texttt{succ} \: t : \texttt{Nat}$}
        \end{bprooftree}
        \begin{bprooftree}
            \AxiomC{$\Gamma \vdash e \hookrightarrow t : \texttt{Nat}$}
            \RightLabel{\scriptsize{\text{[C-PRED1]}}}
            \UnaryInfC{$\Gamma \vdash \texttt{pred} \: e \hookrightarrow \texttt{pred} \: t : \texttt{Nat}$}
        \end{bprooftree}
    \]
    \vspace{0.5mm}
    \[
        \begin{bprooftree}
            \AxiomC{$\Gamma \vdash e \hookrightarrow t : \texttt{Nat}$}
            \RightLabel{\scriptsize{\text{[C-ISZERO1]}}}
            \UnaryInfC{$\Gamma \vdash \texttt{iszero} \: e \hookrightarrow \texttt{iszero} \: t : \texttt{Bool}$}
        \end{bprooftree}
    \]
    \vspace{0.5mm}
    \[
        \begin{bprooftree}
            \AxiomC{$\Gamma \vdash e \hookrightarrow t : \: ?$}
            \RightLabel{\scriptsize{\text{[C-SUCC2]}}}
            \UnaryInfC{$\Gamma \vdash \texttt{succ} \: e 
            \hookrightarrow \texttt{succ} \: \langle \texttt{Nat}?^l \rangle t: \texttt{Nat}$}
        \end{bprooftree}
    \]
    \vspace{0.5mm}
    \[
        \begin{bprooftree}
            \AxiomC{$\Gamma \vdash e \hookrightarrow t : \: ?$}
            \RightLabel{\scriptsize{\text{[C-PRED2]}}}
            \UnaryInfC{$\Gamma \vdash \texttt{succ} \: e 
            \hookrightarrow \texttt{pred} \: \langle \texttt{Nat}?^l \rangle t: \texttt{Nat}$}
        \end{bprooftree}
    \]
    \vspace{0.5mm}
    \[
        \begin{bprooftree}
            \AxiomC{$\Gamma \vdash e \hookrightarrow t : \: ?$}
            \RightLabel{\scriptsize{\text{[C-ISZERO2]}}}
            \UnaryInfC{$\Gamma \vdash \texttt{iszero} \: e 
            \hookrightarrow \texttt{iszero} \: \langle \texttt{Nat}?^l \rangle t: \texttt{Bool}$}
        \end{bprooftree}
    \]
    \vspace{0.5mm}
    \[
        \begin{bprooftree}
            \AxiomC{$\Gamma \vdash e_1 \hookrightarrow t_1 : \texttt{Bool}$}
            \AxiomC{$\Gamma \vdash e_2 \hookrightarrow t_2 : S$}
            \AxiomC{$\Gamma \vdash e_3 \hookrightarrow t_3 : T$}
            \AxiomC{$c = \langle T \rightsquigarrow S \rangle$}
            \RightLabel{\scriptsize{\text{[C-IF1]}}}
            \QuaternaryInfC{$\Gamma \vdash \texttt{if} \: e_1 \: \texttt{then} \: e_2 \: 
            \texttt{else} \: e_3  
            \hookrightarrow \texttt{if} \: t_1 \: \texttt{then} \: t_2 \: 
            \texttt{else} \: \langle c \rangle t_3: \texttt{S}$}
        \end{bprooftree}
    \]
    \vspace{0.5mm}
    \[
        \begin{bprooftree}
            \AxiomC{$\Gamma \vdash e_1 \hookrightarrow t_1 : \: ?$}
            \AxiomC{$\Gamma \vdash e_2 \hookrightarrow t_2 : S$}
            \AxiomC{$\Gamma \vdash e_3 \hookrightarrow t_3 : T$}
            \AxiomC{$c = \langle T \rightsquigarrow S \rangle$}
            \RightLabel{\scriptsize{\text{[C-IF2]}}}
            \QuaternaryInfC{$\Gamma \vdash \texttt{if} \: e_1 \: \texttt{then} \: e_2 \: 
            \texttt{else} \: e_3  
            \hookrightarrow \texttt{if} \: \langle \texttt{Bool}?^l \rangle t_1 \: 
            \texttt{then} \: t_2 \: 
            \texttt{else} \: \langle c \rangle t_3: \texttt{S}$}
        \end{bprooftree}
    \]
    \vspace{0.5mm}
    \[
        \begin{bprooftree}
            \AxiomC{$\Gamma \vdash e_1 \hookrightarrow t_1 : S \rightarrow T$}
            \AxiomC{$\Gamma \vdash e_2 \hookrightarrow t_2 : S'$}
            \AxiomC{$c = \langle S' \rightsquigarrow S \rangle$}
            \RightLabel{\scriptsize{\text{[C-APP1]}}}
            \TrinaryInfC{$\Gamma \vdash e_1 \: e_2  
            \hookrightarrow t_1 \: \langle c \rangle t_2 : T$}
        \end{bprooftree}
    \]
    \vspace{0.5mm}
    \[
        \begin{bprooftree}
            \AxiomC{$\Gamma \vdash e_1 \hookrightarrow t_1 : \: ?$}
            \AxiomC{$\Gamma \vdash e_2 \hookrightarrow t_2 : S$}
            \RightLabel{\scriptsize{\text{[C-APP2]}}}
            \BinaryInfC{$\Gamma \vdash e_1 \: e_2  
            \hookrightarrow \langle \texttt{Fun}?^l \rangle t_1 \: 
            \langle S! \rangle t_2 : ?$}
        \end{bprooftree}
    \]
    \vspace{0.5mm}
    \[
        \begin{bprooftree}
            \AxiomC{$\Gamma \vdash e \hookrightarrow t : T$}
            \RightLabel{\scriptsize{\text{[C-REF]}}}
            \UnaryInfC{$\Gamma \vdash \texttt{ref} \: e  
            \hookrightarrow \texttt{ref} \: t : \texttt{Ref} \: T$}
        \end{bprooftree}
        \begin{bprooftree}
            \AxiomC{$\Gamma \vdash e \hookrightarrow t : \texttt{Ref} \: T$}
            \RightLabel{\scriptsize{\text{[C-DEREF1]}}}
            \UnaryInfC{$\Gamma \vdash \: !e  
            \hookrightarrow \: !t : T$}
        \end{bprooftree}
    \]
    \vspace{0.5mm}
    \[
        \begin{bprooftree}
            \AxiomC{$\Gamma \vdash e \hookrightarrow t : \: ?$}
            \RightLabel{\scriptsize{\text{[C-DEREF2]}}}
            \UnaryInfC{$\Gamma \vdash \: !e  
            \hookrightarrow \: !(\langle \texttt{Ref}?^l \rangle t) : \: ?$}
        \end{bprooftree}
    \]
    \vspace{0.5mm}
    \[
        \begin{bprooftree}
            \AxiomC{$\Gamma \vdash e_1 \hookrightarrow t_1 : \texttt{Ref} \: S$}
            \AxiomC{$\Gamma \vdash e_2 \hookrightarrow t_2 : T$}
            \AxiomC{$c = \langle T \rightsquigarrow S \rangle$}
            \RightLabel{\scriptsize{\text{[C-ASSIGN1]}}}
            \TrinaryInfC{$\Gamma \vdash (e_1 := e_2)  
            \hookrightarrow (t_1 := \langle c \rangle t_2) : S$}
        \end{bprooftree}
    \]
    \vspace{0.5mm}
    \[
        \begin{bprooftree}
            \AxiomC{$\Gamma \vdash e_1 \hookrightarrow t_1 : \: ?$}
            \AxiomC{$\Gamma \vdash e_2 \hookrightarrow t_2 : T$}
            \RightLabel{\scriptsize{\text{[C-ASSIGN2]}}}
            \BinaryInfC{$\Gamma \vdash (e_1 := e_2)  
            \hookrightarrow (\langle \texttt{Ref}?^l \rangle t_1 := 
            \langle T! \rangle t_2) : \: ?$}
        \end{bprooftree}
    \]
    \hrule
\end{figure} 

The cast insertion process is implemented as part of the typechecker. 
We also need to incorporate blame tracking whenever an unsafe cast 
is introduced to the original program. To keep it simple, we 
represent a blame label as an integer. A new label is obtained 
by simply incrementing the value of the last assigned label by 1. 
Additionally, the typechecker should be able to report errors on 
ill-typed expressions. In order to achieve both blame tracking 
and exception handling, a new monad \lstinline{TCheckState} is defined 
as follows:
\begin{lstlisting} 
    type TCheckState a = ExceptT TypeError (State Label) a
\end{lstlisting}
By combining the \lstinline{State} monad and the \lstinline{Except} monad, \lstinline{TCheckState} 
helps maintain a global counter for labeling unsafe coercions 
as well as return errors during typechecking. The data type 
for type errors captures eight possible causes: (1) out-of-bound variables, 
(2) type difference between the two conditional branches, (3) non-boolean condition, 
(4) unexpected argument type for an arithmetic operator or (5) a function, 
(6) assigning to a non-reference term, 
(7) type mismatch between the two sides of an assignment, and (8) derefencing a 
non-reference term. 

\begin{lstlisting}
    data TypeError = NotBound Term                 
                   | Difference Type Type          
                   | FunMismatch Type Type Term    
                   | NotBool Type                  
                   | NotNat Type                   
                   | NotFunction Term               
                   | IllegalAssign Term            
                   | AssignMismatch Type Type Term  
                   | IllegalDeref Term   
\end{lstlisting}

We transliterate the cast insertion rules in Figure 6 
into the \lstinline{typeCheck'} function that takes in an expression as input 
and returns either a tuple of a modified expression and its type, 
or a type error.

\begin{lstlisting}
    typeCheck' :: Term -> TCheckState (Term, Type)
    -- | Constants
    typeCheck' e = case e of                              
        Tru  -> return (e, Boolean)                                    
        Fls  -> return (e, Boolean)                                   
        Zero -> return (e, Nat)     
    -- | Arithmetic                              
    typeCheck' (Succ e') -> 
        do (t', ty) <- typeCheck' e'
            case ty of  
                Dyn -> do c <- coerce ty Nat
                           return (Succ $ Cast c t', Nat)
                Nat -> return (Succ t', Nat)
                _   -> throwError $ NotNat ty
    typeCheck' (Pred e') -> 
        do (t', ty) <- typeCheck' e'
            case ty of
                Dyn -> do c <- coerce ty Nat
                           return (Pred $ Cast c t', Nat)
                Nat -> return (Pred t', Nat)
                _   -> throwError $ NotNat ty
    typeCheck' (IsZero e') -> 
        do (t', ty) <- typeCheck' e' 
            case ty of 
                Dyn -> do c <- coerce ty Nat
                           return (IsZero $ Cast c t', Boolean)
                Nat -> return (IsZero t', Boolean)
                _   -> throwError $ NotNat ty
    -- | Conditional
    typeCheck' (If e1 e2 e3) = 
        do (t1, cond) <- typeCheck' e1 
           (t2, fst)  <- typeCheck' e2 
           (t3, snd)  <- typeCheck' e3
           case cond of 
               Dyn 
                   | fst == snd -> do c1 <- coerce cond Boolean
                                       let t1' = Cast c1 t1
                                       return (If t1' t2 t3, fst)  
                   | fst `isConsistent` snd -> 
                       do c1 <- coerce cond Boolean
                          c2 <- coerce fst snd
                          c3 <- coerce snd fst 
                          let (t1', t2', t3') = (Cast c1 t1, Cast c2 t2, Cast c3 t3)
                          return (If t1' t2' t3', Dyn)   
                   | otherwise -> throwError $ Difference fst snd
               Boolean 
                   | fst == snd -> return (If t1 t2 t3, fst) 
                   | fst `isConsistent` snd -> 
                       do c2 <- coerce fst snd 
                          c3 <- coerce snd fst
                          let (t2', t3') = (Cast c2 t2, Cast c3 t3)
                          return (If t1 t2' t3', Dyn) 
                   | otherwise -> throwError $ Difference fst snd
               _ -> throwError $ NotBoolean cond      
    -- | Reference
    typeCheck' (Ref e') -> 
        do (t', ty) <- typeCheck' e'   
           return (Ref t', TRef ty)  
    -- | Dereference                           
    typeCheck' (Deref e') -> 
        do (t', ty) <- typeCheck' e' 
           case ty of 
               Dyn    -> do l <- GlobalS.newLabel
                            return (Deref $ Cast (RefProj l) t', Dyn)
               TRef s -> return (Deref t', s)
               _      -> throwError $ IllegalDeref e' 
    -- | Assignment
    typeCheck' (Assign e1 e2) -> 
        do (t1, s1)  <- typeCheck' e1 
           (t2, s2) <- typeCheck' e2 
           case s1 of 
               TRef s 
                   | s2 == s -> return (t1 `Assign` t2, s)
                   | s2 `isConsistent` s -> 
                       do c <- coerce s2 s
                          return (t1 `Assign` Cast c t2, s)
                   | otherwise -> 
                       throwError $ AssignMismatch s2 s (Assign e1 e2)
               Dyn -> 
                   do l  <- GlobalS.newLabel
                      c2 <- coerce s2 Dyn 
                      return (Cast (RefProj l) t1 `Assign` Cast c2 t2, Dyn)
               _ -> throwError $ IllegalAssign e1
    -- | Variable 
    typeCheck' (Var _ ty _) = case ty of                                            
        TUnit -> throwError $ NotBound e   
        _     -> return (e, ty) 
    -- | Abstraction
    typeCheck' (Lambda ty e' ctx) -> 
        do (t', retTy) <- typeCheck' e'            
            return (Lambda ty t' ctx, Arr ty retTy)  
    -- | Application
    typeCheck' (App e1 e2) -> 
        do (t1, funcTy) <- typeCheck' e1   
           (t2, argTy) <- typeCheck' e2
           case funcTy of 
               Dyn -> 
                   do l  <- GlobalS.newLabel 
                      c2 <- coerce argTy Dyn   
                      return (Cast (FuncProj l) t1 `App` Cast c2 t2, Dyn)
               Arr paramTy retTy 
                   | argTy == paramTy -> return (App t1 t2, retTy)
                   | argTy `isConsistent` paramTy -> 
                       do c <- coerce argTy paramTy  
                          return (App t1 $ Cast c t2, retTy)
                   | otherwise -> 
                       throwError $ FunMismatch argTy paramTy (App e1 e2)
               _ -> throwError $ NotFunction e1   
\end{lstlisting}

There are few slight changes to the formal type system. First is the 
omission of the global type environment for bound variables. Instead, each 
variable carries its own type, and those with \lstinline{TUnit} type are 
out of bound. Secondly, the function always checks for type 
consistency before performing coercions. \begin{figure}[ht!]
    \caption{Operational Semantics}
    \hrule
    \vspace{4mm}
    \qquad \text{\underline{S}yntax:}
    \begin{alignat*}{3}
        &(\textit{expressions}) \qquad 
        & e ::= \: & x \; | \;  
          e \: e \; | \;
          \lambda x \! : \! T . \: e \; | \;
          k \; | \; \langle c \rangle e\\
    &&    & \texttt{succ} \: e \; | \;
          \texttt{pred} \: e \; | \;
          \texttt{iszero} \: e \; | \; \\
    &&    & \texttt{if} \: e \: \texttt{then} \: e \: \texttt{else} \: e \; | \; \\
    &&    & \texttt{ref} \: e \; | \;
          !e \; | \;
          e := e \\
        &(\textit{uncoerced values}) \qquad
        & u ::= \: & k \; | \; a \; | \; \lambda x \! : \! T. \:e  \\
        &(\textit{values}) \qquad
        & v ::= \: & u \; | \; \langle \hat{c} \rangle u \\
        &(\textit{store locations}) \qquad
        & a \in \: & \{0 \texttt{x} m \: | \: m \in \mathbb{N}^+\}
    \end{alignat*}
    \begin{alignat*}{3}
        \qquad & \text{\underline{E}valuation Rules:} \\ \\
        & {\scriptsize{\text{[E-PREDZERO]}}} \qquad & \texttt{pred} \: 0 & \longrightarrow 0\\ 
        & {\scriptsize{\text{[E-PREDZERO]}}} \qquad & \texttt{pred} \: (\texttt{succ} \: n) & \longrightarrow n&\\
        & {\scriptsize{\text{[E-ISZEROZERO]}}} \qquad &\texttt{iszero} \: 0 & \longrightarrow \texttt{true} \\
        & {\scriptsize{\text{[E-ISZEROSUCC]}}} \qquad & \texttt{iszero} \: (\texttt{succ} \: n) & \longrightarrow \texttt{false} \\
        & {\scriptsize{\text{[E-IFTRUE]}}} \qquad & \texttt{if} \: \texttt{true} \: \texttt{then} \: t_2 \: \texttt{else} \: t_3 & \longrightarrow t_2 \\
        & {\scriptsize{\text{[E-IFFALSE]}}} \qquad & \texttt{if} \: \texttt{false} \: \texttt{then} \: t_2 \: \texttt{else} \: t_3 & \longrightarrow t_3 \\
        & {\scriptsize{\text{[E-REF]}}} \qquad & \texttt{ref} \: v & \longrightarrow a, \sigma[a := v] \hspace{1cm} & a \not \in dom(\sigma) \\
        & {\scriptsize{\text{[E-DEREF]}}} \qquad & !a & \longrightarrow \sigma(a) \\
        & {\scriptsize{\text{[E-ASSIGN]}}} \qquad & a := v & \longrightarrow v, \sigma[a := v] \\
        & {\scriptsize{\text{[E-APP]}}} \qquad & (\lambda x \! : \! T. \:e) \: v & \longrightarrow e[x \mapsto v] \\
        & {\scriptsize{\text{[E-ID]}}} \qquad & \langle I \rangle u & \longrightarrow u \\
        & {\scriptsize{\text{[E-COMP]}}} \qquad & \langle c \rangle \langle \hat{d} \rangle u & \longrightarrow \langle \hat{d};c \rangle u \\
        & {\scriptsize{\text{[E-FAIL]}}} \qquad & \langle \texttt{Fail}^l \rangle u & \longrightarrow \texttt{blame} \: l \\ 
        & {\scriptsize{\text{[E-GROUND]}}} \qquad & \langle \hat{c} \rangle u & \longrightarrow u \\
        & {\scriptsize{\text{[E-CSTEP]}}} \qquad & \langle c \rangle u & \longrightarrow \langle c' \rangle u \hspace{1cm} & c \longmapsto _X c' \\
        & {\scriptsize{\text{[E-CDEREF]}}} \qquad & !(\langle \texttt{Ref} \: c \: d \rangle a) & \longrightarrow \langle d \rangle !a \\
        & {\scriptsize{\text{[E-CASSIGN]}}} \qquad & \langle \texttt{Ref} \: c \: d \rangle a := v & \longrightarrow \langle d \rangle (a := \langle c \rangle v) \\
        & {\scriptsize{\text{[E-CAPP]}}} \qquad & (\langle \texttt{Fun} \: c \: d \rangle u) \: v & \longrightarrow \langle d \rangle (u \: \langle c \rangle v)
    \end{alignat*}
    \hrule
\end{figure} This sanity check allows the 
typechecker to handle some errors directly instead of delegating them 
to the evaluator through failure coercions. 
Additionally, we add another rule for application when the 
argument type is identical to the parameter type. As a result, no casts 
are needed. In this way, we eliminate unnecessary coercions that make 
no change to the semantics. The same pattern is applied for conditional 
expressions and assignments by including type equivalence check before 
type consistency. Finally, we call the subroutine \lstinline{GlobalS.newLabel} 
to generate a new blame label for every unsafe cast inserted. 

\subsection{Operational Semantics}
Figure 7 formalizes the operational semantics for a typechecked 
program with coercions. In addition to the standard rules for 
the $\lambda$-calculus with references, the semantics also 
defines rules for cast expressions. Rule  
{\scriptsize{[E-ID]}} removes the identity coercion and 
returns the inner value. 
Rule {\scriptsize{[E-COMP]}} composes two 
adjacent coercions given that the first coercion is normalized. 
This restriction makes sure that coercions do not get accumulated 
without being reduced. 
The {\scriptsize{[E-GROUND]}} rule unwraps a cast value 
when its coercion is normalized. Otherwise, we take a small 
step of evaluation inside the coercion using the reduction rules 
in Figure 5. The combination of {\scriptsize{[E-CSTEP]}} and 
{\scriptsize{[E-COMP]}} helps maintain a bounded size on coercions. 
Rule {\scriptsize{[E-CDEREF]}} casts the value read 
from a cell, whereas {\scriptsize{[E-CASSIGN]}} also casts the value 
written to it. Rule 
{\scriptsize{[E-CAPP]}} splits the function coercion 
into two separate coercions for the argument and the result. 
Finally, we terminate evaluation and throw a cast error 
immediately when the coercion $\texttt{Fail}^l$ is encountered, 
as described in the {\scriptsize{[E-FAIL]}} rule. The blame 
label consequently identifies the responsible coercion.

Similar to the \lstinline{typeCheck'} procedure, we define a 
monad \lstinline{SEvalState} to provide a global state for a set $\sigma$ of 
store locations and also to signal runtime errors.
\begin{lstlisting}
    type SEvalState a = ExceptT RuntimeError (State StoreEnv) a
\end{lstlisting}
The following implementation describes a small-step evaluator 
based on the established reduction rules in Figure 7.

\begin{lstlisting}
    evaluate' :: Term -> SEvalState Term
    evaluate' t = case t of
        -- | Arithmetic
        Pred Zero                       -> return Zero                         
        Pred (Succ nv) 
            | isNumeric nv              -> return nv                       
        Pred t'                         -> Pred <$> evaluate' t'
        IsZero Zero                     -> return Tru                        
        IsZero (Succ nv) 
            | isNumeric nv              -> return Fls                     
        IsZero t'                       -> IsZero <$> evaluate' t'
        Succ t'                         -> Succ <$> evaluate' t'

        -- | Conditional
        If Tru t2 t3                    -> return t2                        
        If Fls t2 t3                    -> return t3                         
        If t1 t2 t3                     -> (\t1' -> If t1' t2 t3) <$> evaluate' t1

        -- | Cast
        Cast c t' 
            | not (isVal t')            -> Cast c <$> evaluate' t'
        Cast c (Cast d u)               -> return $ Seq d c `Cast` u            
        Cast (Iden _) u                 -> return u                        
        Cast (Fail s1 s2 l) u           -> throwError $ Blame s1 s2 l                 
        Cast c u 
            | isNormalized c            -> unbox t                              
        Cast c u                        -> return $ reduceCoercion c `Cast` u
        Deref (Cast (CRef c d) (Loc l)) -> return $ Cast d $ Deref (Loc l)      
        Assign (Cast (CRef c d) (Loc l)) v                                                
            | isVal v                   -> return $ Cast d $ Loc l `Assign` Cast c v 
        App (Cast (Func c d) u) v 
            | isUncoercedVal u 
              && isVal v                -> return $ Cast d $ u `App` Cast c v                        

        -- | Reference
        Ref v     
            | isVal v                   -> GlobalS.allocate v                             
        Ref t'                          -> Ref <$> evaluate' t'
        Deref (Loc l)                   -> GlobalS.peek l

        -- | Dereference
        Deref t'                        -> Deref <$> evaluate' t'
                                            
        -- | Assignment
        Assign (Loc l) v
            | isVal v                   -> GlobalS.update l v                       
        Assign v1 t2 
            | isVal v1                  -> Assign v1 <$> evaluate' t2
        Assign t1 t2                    -> (`Assign` t2) <$> evaluate' t1

        -- | Application
        App (Lambda _ t1 _) v2 
            | isVal v2                  -> return $ subsFromTop v2 t1          
        App v1 t2 
            | isVal v1                  -> App v1 <$> evaluate' t2
        App t1 t2                       -> (`App` t2) <$> evaluate' t1                        
                        
        -- | No rules applied
        _                               -> throwError Stuck              
\end{lstlisting}

For the evaluation of a cast expression (\lstinline{Cast c t'}), 
we first check if the inner expression \lstinline{t'} is a value. If not, then 
we take a small-step of evaluation for \lstinline{t'}. Otherwise, 
we consider the next case where \lstinline{t'} is a cast value   
\lstinline{Cast d u}, where \lstinline{d} is normalized. In this 
case, we apply rule {\scriptsize{[E-COMP]}} to combine coercions \lstinline{c} 
and \lstinline{d}. The subsequent cases handle cast expressions where 
\lstinline{t'} is a simple value, and the coercion \lstinline{c} takes 
different patterns. The 
\lstinline{unbox} function unwraps a cast expression whose coercion is normalized 
provided the target type matches the runtime type.    
The subroutines \lstinline{GlobalS.allocate} and \lstinline{GlobalS.update} 
modify the global store $\sigma$ by allocating a new cell and 
updating an existing cell, respectively, while \lstinline{GlobalS.peek} 
reads from a cell. The subroutine \lstinline{subsFromTop} 
performs a substitution 
operation $e[x \mapsto v]$ on nameless terms by renumbering 
free variables. Appendix D discusses this process in more detail. 
Runtime errors include cast errors and stuckness. Cast expressions 
with a failure coercion gives a cast error. 
A stuck evaluation results when a term has no transition states 
because it is either a simple value or meaningless. Hence, a \lstinline{Stuck} 
exception is simply an indication that an evaluation has 
terminated. Otherwise, 
we keep applying \lstinline{evaluate'} on non-stuck terms 
to reach the final state. The following 
function performs the big-step semantics. 

\begin{lstlisting}
    evaluate :: Term -> Either RuntimeError Term
    evaluate t = 
        let (res, t') = GlobalS.runBEval $ evaluateToValue t StoreEnv.empty
        in case res of
               Right t'' 
                   | isUncoercedVal t'' -> Right t''
                   | otherwise          -> Left Stuck 
               Left (Blame s1 s2 l)     -> let cause = fromJust $ blame l t
                                               bres = BlameRes cause t'
                                           in Left $ CastError s1 s2 bres  
               Left err                 -> Left err 
\end{lstlisting}

The subroutine \lstinline{GlobalS.runBEval} initializes a global state 
for evaluation with an empty store. \lstinline{evaluateToValue} repeatedly 
applies the small-step evaluator until a stuck term or an error is encountered. 
A stuck term is only valid if it is an uncoerced value. Else, an actual error 
of a stuck evaluation is returned. In the case of a cast error, we call the 
\lstinline{blame} function with a label to search for the erroneous 
coercion in the input expression. The data type \lstinline{BlameRes} 
wraps this coercion along with the last evaluated term for the error message. 

\subsection{Overview}
Now that we have defined the typechecking and evaluation procedures, 
our interpreter for $\lambda ^? _{\rightarrow}$ works as follows:
\begin{lstlisting}
    interpret :: String -> IO ()
    interpret line = case parseExpr line of 
        Right validExpr -> case typeCheck validExpr of 
                               Right t  -> case evaluate t of 
                                               Right res -> printMsg res
                                               Left err  -> printMsg err
                               Left err -> printMsg err
        Left err        -> print err   
\end{lstlisting}
The interpreter first parses the input stream into an abstract syntax tree. 
If parsing succeeds, then the parsed expression is typechecked; 
otherwise, we output a parse error. A well-typed expression is subsequently 
evaluated to a value, or a runtime error is thrown. Funtion \lstinline{printMsg} 
prints the result in a readable format (see Appendix C).

\subsection{Examples}
Consider the following expression:
\begin{shiftedflalign*}
    & (\lambda x . \: \texttt{succ} \: x) \: \texttt{true} & 
\end{shiftedflalign*}
The type checking process results in the following modified expression 
with additional coercions.
\begin{shiftedflalign*}
    & (\lambda x \! : ?. \: \texttt{succ} \: (\langle \texttt{Nat}? ^{l_1} \rangle x)) \: 
    \langle \texttt{Bool}! \rangle \texttt{true} &
\end{shiftedflalign*}  
The first step of evaluation gives
\begin{shiftedflalign*}
    & \texttt{succ} \: (\langle \texttt{Nat}? ^{l_1} \rangle 
    \langle \texttt{Bool}! \rangle \texttt{true}) & 
\end{shiftedflalign*}
Rule {\scriptsize{[E-COMP]}} combines the adjacent coercions 
$\langle \texttt{Nat}? ^{l_1} \rangle$ and $\langle \texttt{Bool}! \rangle$,
which evaluates to $\texttt{Fail}^{l_1}$. The result is a cast error.
\begin{shiftedflalign*}
    & \texttt{blame} \: l_1 & 
\end{shiftedflalign*}
Next, we review another example that involves higher-order coercions. 
\begin{shiftedflalign*}
    & (\lambda m . \: (\lambda x \! : \! \texttt{Nat} \! \rightarrow \! \texttt{Nat} . \: 
    (x \: 0)) \: m) \: (\lambda y \! : \! \texttt{Nat} . \: \texttt{succ} \: y)& 
\end{shiftedflalign*}
The cast insertion judgment produces the following expression. To keep it short and 
intuitive, we abbreviate 
$\texttt{Fun} \: c \: d$ as $\langle c \! \rightarrow \! d \rangle$. 
\begin{shiftedflalign*}
    & (\lambda m\!:? . \: (\lambda x \! : \! \texttt{Nat} \! \rightarrow \! \texttt{Nat} . \: 
    (x \: 0)) \: (\langle \texttt{Nat}! \! \rightarrow \! \texttt{Nat}? ^{l_1} \rangle \langle \texttt{Fun}? ^{l_2} \rangle m)) & \\ 
    & \hspace{6.2cm} \: \langle \texttt{Fun}! \rangle \langle \texttt{Nat}? ^{l_3} \! \rightarrow \! \texttt{Nat}! \rangle 
    (\lambda y \! : \! \texttt{Nat} . \: \texttt{succ} \: y) & 
\end{shiftedflalign*}
$m$ is dynamically-typed and coerced to the argument type $\texttt{Nat} \! \rightarrow \! \texttt{Nat}$. 
The function check $\langle \texttt{Fun}? ^{l_2} \rangle$ is inserted to ensure that $m$ first evaluates 
to a function. The value $(\lambda y \! : \! \texttt{Nat} . \: \texttt{succ} \: y)$ is coerced to type ? to 
be consistent with $m$. The tag $\langle \texttt{Fun}! \rangle$ upcasts a function type to ?.
The following are the steps of evaluation.
\begin{align*}
    & (\lambda m \! :? . \: (\lambda x \! : \! \texttt{Nat} \! \rightarrow \! \texttt{Nat} . \: 
    (x \: 0)) \: (\langle \texttt{Nat}! \! \rightarrow \! \texttt{Nat}? ^{l_1} \rangle \langle \texttt{Fun}? ^{l_2} \rangle m)) & \\
    & \hspace{6.1cm} \: \langle \texttt{Fun}! \rangle \langle \texttt{Nat}? ^{l_3} \! \rightarrow \! \texttt{Nat}! \rangle 
    (\lambda y \! : \! \texttt{Nat} . \: \texttt{succ} \: y) \\
    \xrightarrow{\text{ 1 }} \:
    &  (\lambda x \! : \! \texttt{Nat} \! \rightarrow \! \texttt{Nat} . \: 
    (x \: 0)) & \\ 
    & \hspace{2.2cm} \: \langle \texttt{Nat}! \! \rightarrow \! \texttt{Nat}? ^{l_1} \rangle \langle \texttt{Fun}? ^{l_2} \rangle 
    \langle \texttt{Fun}! \rangle \langle \texttt{Nat}? ^{l_3} \! \rightarrow \! \texttt{Nat}! \rangle 
    (\lambda y \! : \! \texttt{Nat} . \: \texttt{succ} \: y) \\
    \xrightarrow{\text{ 2 }} \:    
    & (\lambda x \! : \! \texttt{Nat} \! \rightarrow \! \texttt{Nat} . \: 
    (x \: 0)) & \\ 
    & \hspace{1.4cm} \: \langle \texttt{Nat}! \! \rightarrow \! \texttt{Nat}? ^{l_1} \rangle (\highlight{(\langle \texttt{Fun}? ^{l_2} \rangle 
    \langle \texttt{Fun}! \rangle)} \langle \texttt{Nat}? ^{l_3} \! \rightarrow \! \texttt{Nat}! \rangle) 
    (\lambda y \! : \! \texttt{Nat} . \: \texttt{succ} \: y) \\
    \xrightarrow{\text{ 3 }} \:
    & (\lambda x \! : \! \texttt{Nat} \! \rightarrow \! \texttt{Nat} . \: 
    (x \: 0)) \: \langle \texttt{Nat}! \! \rightarrow \! \texttt{Nat}? ^{l_1} \rangle \highlight{(\langle I \rangle
     \langle \texttt{Nat}? ^{l_3} \! \rightarrow \! \texttt{Nat}! \rangle)} 
    (\lambda y \! : \! \texttt{Nat} . \: \texttt{succ} \: y) \\
    \xrightarrow{\text{ 4 }} \:
    & (\lambda x \! : \! \texttt{Nat} \! \rightarrow \! \texttt{Nat} . \: 
    (x \: 0)) \: \highlight{\langle \texttt{Nat}! \! \rightarrow \! \texttt{Nat}? ^{l_1} \rangle 
     \langle \texttt{Nat}? ^{l_3} \! \rightarrow \! \texttt{Nat}! \rangle}
    (\lambda y \! : \! \texttt{Nat} . \: \texttt{succ} \: y) \\
    \xrightarrow{\text{ 5 }} \:
    & (\lambda x \! : \! \texttt{Nat} \! \rightarrow \! \texttt{Nat} . \: 
    (x \: 0)) \: \langle \highlight{\langle \texttt{Nat}? ^{l_3} \rangle \langle \texttt{Nat}! \rangle} \! \rightarrow \! 
    \highlight{\langle \texttt{Nat}? ^{l_1} \rangle \langle \texttt{Nat}! \rangle} \rangle
    (\lambda y \! : \! \texttt{Nat} . \: \texttt{succ} \: y) \\
    \xrightarrow{\text{ 6 }} \:
    & (\lambda x \! : \! \texttt{Nat} \! \rightarrow \! \texttt{Nat} . \: 
    (x \: 0)) \: \highlight{\langle I \! \rightarrow \! I \rangle}
    (\lambda y \! : \! \texttt{Nat} . \: \texttt{succ} \: y) \\
    \xrightarrow{\text{ 7 }} \:
    & (\lambda x \! : \! \texttt{Nat} \! \rightarrow \! \texttt{Nat} . \: 
    (x \: 0)) \: \highlight{\langle I \rangle
    (\lambda y \! : \! \texttt{Nat} . \: \texttt{succ} \: y)} \\
    \xrightarrow{\text{ 8 }} \:
    & (\lambda x \! : \! \texttt{Nat} \! \rightarrow \! \texttt{Nat} . \: 
    (x \: 0)) \: 
    (\lambda y \! : \! \texttt{Nat} . \: \texttt{succ} \: y) \\
    \xrightarrow{\text{ 9 }} \:
    & (\lambda y \! : \! \texttt{Nat} . \: \texttt{succ} \: y) \: 0 \\
    \xrightarrow{\text{ 10}} \:
    & \texttt{succ} \: 0
\end{align*}
The first step applies the rule {\scriptsize{[E-APP]}} to substitute 
$\langle \texttt{Fun}! \rangle \langle \texttt{Nat}? ^{l_3} \! \rightarrow \! \texttt{Nat}! \rangle 
(\lambda y \! : \! \texttt{Nat} . \: \texttt{succ} \: y)$ for $m$. The result gives 
a sequence of adjacent coercions that needs to be reduced. First, we reassociate the coercions 
to find a pair that we can simplify, as in Step 2. The composition $\langle \texttt{Fun}? ^{l_2} \rangle 
\langle \texttt{Fun}! \rangle$ reduces to the identity $I$. This step satisfies the requirement that 
$m$ has to be a function. The next step is to check if $m$ has the expected function type of 
$\texttt{Nat} \! \rightarrow \! \texttt{Nat}$. The two function coercions $\langle \texttt{Nat}! \! 
\rightarrow \! \texttt{Nat}? ^{l_1} \rangle$ and $\langle \texttt{Nat}? ^{l_3} \! \rightarrow \! \texttt{Nat}! \rangle$ 
are reduced into one by composing the argument coercions and the output coercions respectively. The 
identity results from the compatibility of two coercions. We can then remove the cast from  
$(\lambda y \! : \! \texttt{Nat} . \: \texttt{succ} \: y)$ and perform another function application 
to get the final result. 
Our final example uses reference coercions. 
\begin{shiftedflalign*}
    & (\lambda y. \: (\lambda x . \: x := y)) \: 0 \: (\texttt{ref} \: \texttt{false})&
\end{shiftedflalign*}
which results in the following cast expression:
\begin{shiftedflalign*}
    & (\lambda y \! : ?. \: (\lambda x \! : ?. \: \langle \texttt{Ref}? ^{l_1} \rangle x := y)) 
    \: (\langle \texttt{Nat}! \rangle 0) \: (\langle \texttt{Ref}! \rangle
    \langle \texttt{Ref} \: \texttt{Bool}? ^{l_2} \: \texttt{Bool}! \rangle (\texttt{ref} \: \texttt{false}))&
\end{shiftedflalign*}
Since $x$ is on the left-hand side of the assignment, it is expected to be a reference through 
the reference check $\texttt{Ref}? ^{l_1}$. 0 is coerced to ? to match the dynamic type of $y$, 
and the tag $\texttt{Ref}!$ coerces $\texttt{ref} \: \texttt{false}$ 
to ? to be consistent with the dynamically-typed $x$. The steps of evaluation are as follows.
\begin{align*}
    & (\lambda y \! : ?. \: (\lambda x \! : ?. \: \langle \texttt{Ref}? ^{l_1} \rangle x := y)) 
    \: (\langle \texttt{Nat}! \rangle 0) \: (\langle \texttt{Ref}! \rangle
    \langle \texttt{Ref} \: \texttt{Bool}? ^{l_2} \: \texttt{Bool}! \rangle (\texttt{ref} \: \texttt{false})) \\
    \xrightarrow{\text{ 1 }} \:
    & (\lambda x \! : ?. \: \langle \texttt{Ref}? ^{l_1} \rangle x := \langle \texttt{Nat}! \rangle 0) 
    \: (\langle \texttt{Ref}! \rangle
    \langle \texttt{Ref} \: \texttt{Bool}? ^{l_2} \: \texttt{Bool}! \rangle (\texttt{ref} \: \texttt{false})) \\
    \xrightarrow{\text{ 2 }} \:
    & \highlight{(\langle \texttt{Ref}? ^{l_1} \rangle \langle \texttt{Ref}! \rangle)}
    \langle \texttt{Ref} \: \texttt{Bool}? ^{l_2} \: \texttt{Bool}! \rangle (\texttt{ref} \: \texttt{false})
    := \langle \texttt{Nat}! \rangle 0 \\
    \xrightarrow{\text{ 3 }} \:
    & \highlight{\langle I \rangle
    \langle \texttt{Ref} \: \texttt{Bool}? ^{l_2} \: \texttt{Bool}! \rangle} (\texttt{ref} \: \texttt{false})
    := \langle \texttt{Nat}! \rangle 0 \\
    \xrightarrow{\text{ 4 }} \:
    & \langle \texttt{Ref} \: \texttt{Bool}? ^{l_2} \: \texttt{Bool}! \rangle (\texttt{ref} \: \texttt{false})
    := \langle \texttt{Nat}! \rangle 0 \\
    \xrightarrow{\text{ 5 }} \:
    & \langle \texttt{Bool}! \rangle (\texttt{ref} \: \texttt{false})
    := \highlight{\langle \texttt{Bool}? ^{l_2} \rangle \langle \texttt{Nat}! \rangle} 0 \\
    \xrightarrow{\text{ 6 }} \:
    & \texttt{blame} \: l_2
\end{align*}
The first two steps substitute $\langle \texttt{Nat}! \rangle 0$ for $y$ and 
$\langle \texttt{Ref}! \rangle \langle \texttt{Ref} \hspace{0.2em} \texttt{Bool}? ^{l_2} \hspace{0.2em} \texttt{Bool}! \rangle 
(\texttt{ref} \hspace{0.2em} \texttt{false})$ 
for $x$. The reference checking coercions $\texttt{Ref}? ^{l_1}$ and $\texttt{Ref}!$ are eliminated 
since $x$ is a reference. We then apply rule {\scriptsize{[E-CASSIGN]}} 
to cast the \texttt{Nat} value written to $x$ to \texttt{Bool}, which results in a cast error.





\section{Related Work} We have mentioned two 
strategies of blame tracking, which are UD and D. We 
prefer the UD variant for our interpreter because it works 
naturally with the coercion calculus defined in Herman et al. [4]. 
Siek et al. [8] introduce a mechanism to perform the D blame 
tracking for the coercion calculus that respects the traditional 
subtype relation. In this case, ? 
is the top element, and any upcast to ? is considered safe. 
As a result, their gradual system allows direct 
coercion between function types and dynamic types ? without the 
intermediate function dynamic types $? \rightarrow \: ?$. This feature 
enriches the semantics with higher-order projections and 
injections.

Henglein's coercion calculus [3] is originially adopted for the 
interpretation of the dynamically-typed $\lambda$-calculus. 
Whereas our source language is essentially the 
simply-typed $\lambda$-calculus with optional type annotations, 
Henglein's system works with the untyped variant where 
all terms are dynamically-typed. Hence, coercions are used 
primarily for tagging inputs of primitive operators, whose 
types are fixed. Henglein also makes use of function tags 
$\texttt{Fun}!$ and $\texttt{Fun}?$ to differentiate between 
general dynamic types and function dynamic types.

Some languages that support gradual typing are TypeScript and 
C\#. TypeScript allows the programmer to opt out of typechecking 
during compilation by labeling variables that may have dynamic 
content as \texttt{any}. Variables with no type annotations 
default to the \texttt{any} type as well. Similarly, C\# provides
the \texttt{dynamic} annotation to mark variables as dynamically-typed. 



\section{Conclusion} Our interpretation of $\lambda ^? _{\rightarrow}$ closely 
mimics the formal definitions of gradual typing based on coercions introduced in 
Herman et al. [4], with some modifications to the syntax 
and coercion reduction rules found in Siek et al. [8] to 
support UD blame tracking. The use of coercions instead of 
traditional cast wrappers helps improve the space efficiency  
of casts by eagerly combining adjacent casts during 
evaluation. As a result, the coercion calculus enables 
early error detection whenever coercions fail to combine. 
We also add blame tracking to the gradual system  
to inform the programmer of the origin of a cast error. 
We focus the interpretation of $\lambda ^? _{\rightarrow}$ on its 
typechecking and evaluation procedures. The typechecker 
uses the consistency rules to insert necessary casts for 
dynamically-typed parts of a program. The evaluator is then 
responsible for reducing the modified program to a value 
or a cast error. 

Our current interpreter only accepts one statement. We would 
like to extend the abstract syntax tree to allow a program made of 
a sequence of statements. As a result, a cast error message 
should also include a line number to better identify the 
failed expression. Another possible improvement would be to 
explore how to incorporate subtyping 
into the gradual type system through the addition of records. 
The subtype relation in Figure 3 only 
serves to verify the soundness of the UD blame assignment. It 
it not actually part of the implementation. 


\appendix
\section{Lexer}
The library \texttt{Parsec} is used to implement the lexer and parser. 
We first create a language definition specifying how individual character are tokenized, as follows:
\begin{lstlisting}
    langDef :: Tok.LanguageDef ()
    langDef = Tok.LanguageDef
        { Tok.commentStart    = ""  
        , Tok.commentEnd      = ""
        , Tok.commentLine     = "//"
        , Tok.nestedComments  = False
        , Tok.identStart      = letter 
        , Tok.identLetter     = alphaNum 
        , Tok.opStart         = oneOf ":!#$%&*+./<=>?@\\^|-~"
        , Tok.opLetter        = oneOf ":!#$%&*+./<=>?@\\^|-~"
        , Tok.reservedNames   = [ "true"
                                , "false"
                                , "ref"
                                , "if"
                                , "then"
                                , "else"
                                , "succ"
                                , "pred"
                                , "iszero"
                                , "zero"
                                , "Bool"
                                , "Nat"
                                , "Dyn"
                                , "Top"
                                , "Ref" 
                                ]
        , Tok.reservedOpNames = [ "succ"
                                  , "pred"
                                  , "iszero" 
                                  ]
        , Tok.caseSensitive   = True
        }
\end{lstlisting}
and then initialize a lexer using this language definition.
\begin{lstlisting}
    lexer :: Tok.TokenParser ()
    lexer = Tok.makeTokenParser langDef
\end{lstlisting}

Our $\lambda$-calculus does not support block comments or nested comments, requires that an identifier starts with a 
letter and ends with an alphanumeric character. We also reserve some keywords that cannot be used for identifiers. 
The lexer then defines a number of lexical parsers for identifiers, parenthesized expressions, 
and reserved operations such as assignment ($:=$) and lambda ($\lambda$). 

\section{Parser}
Given the tokens provided by the lexer, the parser uses the syntax defined in Section 5 to construct a valid abstract 
syntax tree for the input program. We first define separate parsers for each expression type. Parsing constants, 
variables, and dereference is straightforward. Note that variables are initially parsed as free with type \lstinline{TUnit}.
\begin{lstlisting}
    -- | Parse a dereference.
    dereference :: Parser Term 
    dereference = do 
        reservedOp "!" >> whiteSpace 
        t <- expr
        return $ Deref t
    
    -- | Parse a variable.
    var :: Parser Term
    var = do
        id <- identifier
        return $ Var (-1) TUnit id 

    -- | Parse constants.
    true, false, zero :: Parser Term
    true  = reserved "true" >> return Tru
    false = reserved "false" >> return Fls
    zero  = reserved "0" >> return Zero
\end{lstlisting}
We parse arithmetic operators (\texttt{succ}, \texttt{pred}, \texttt{iszero}), and reference (\texttt{ref}) as unary 
prefix operators, and assignment ($:=$) as a binary infix operator.
\begin{lstlisting}
    operatorTable :: Ex.OperatorTable String () Identity Term
    operatorTable = 
        [ [ Ex.Prefix $ reserved "succ"   >> return Succ
          , Ex.Prefix $ reserved "pred"   >> return Pred
          , Ex.Prefix $ reserved "iszero" >> return IsZero
          , Ex.Prefix $ reserved "ref"    >> return Ref
          , Ex.Infix (reservedOp ":=" >> return Assign) Ex.AssocLeft
          ]
        ]
\end{lstlisting}
The most complicated expressions to parse are $\lambda$-abstractions. 
We first reserve \textbackslash \:  as a symbol for $\lambda$. We give the programmer the option to omit 
type annotations. The \lstinline{option} function checks if there is a type to parse; if not, 
the type is automatically set to ?. We then renumber free and bound variables, and update the type 
of the variable for the current binder. 
\begin{lstlisting}
    -- | Parse an abstraction.
    lambda :: Parser Term
    lambda = do
        reservedOp "\\" 
        arg <- identifier
        ty <- option Dyn $ try colon >> types  
        dot 
        body <- expr
        let t = fixBinding body arg 0
        let t' = updateVarType t arg ty
        let boundVars = arg : getBoundVar body
        let freeVars = getFreeVar body boundVars
        let t'' = fixFreeBinding t' freeVars boundVars
        return $ Lambda ty t'' boundVars
\end{lstlisting}
At the top level, the parser parses an expression as an application by first parsing one or 
more expressions separated by a space and then applying them from left to right.
\begin{lstlisting}
    app :: Parser Term
    app = do
        terms <- sepBy1 expr' whiteSpace 
        return $ applyFromLeft terms
\end{lstlisting}
The parsing for types is more simple. We also give the option to express a dynamically-typed 
expression by annotating it as \lstinline{Dyn}.
\begin{lstlisting}
    -- | Parse base types.
    boolean, nat, top, dynamic :: Parser Type
    boolean = reserved "Bool" >> return Bool
    nat     = reserved "Nat" >> return Nat
    dynamic = reserved "Dyn" >> return Dyn
    
    -- | Parse reference types. 
    ref :: Parser Type 
    ref = do 
        reserved "Ref"
        ty <- types 
        return $ TRef ty
\end{lstlisting}
Unlike applications, types associate to the right. Hence, we parse a function type by applying one 
or more nested types separated by $\rightarrow$ from right to left.
\begin{lstlisting}
    types :: Parser Type
    types = do
        list <- sepBy1 types' arrowSep
        return $ arrowFromRight list
\end{lstlisting}

\section{Pretty Printer}
The library \texttt{PrettyPrint} is used to implement a printer, which converts expressions from their AST representations to 
user-readable formats. We create separate printers for expressions, types, coercions, and errors, all derived from 
the typeclass \lstinline{Pretty}. 
\begin{lstlisting}
    class Pretty a where 
        output :: a -> Doc
        
        printMsg :: a -> IO ()
        printMsg = PP.putDoc . output
\end{lstlisting}
The \lstinline{output} function converts a type $a$ to a 
\texttt{Doc}, which is a set of layouts. This function is left as abstract because each \texttt{Pretty} instance  
requires different formatting styles depending on its type. \texttt{printMsg} has a default implementation of 
piping the result of \texttt{output} to \texttt{PP.putDoc} in order to print type $a$ to the standard output. 
This function is shared by all the \texttt{Pretty} instances. 

\section{Shifting and Substitution} 
Using de Bruijn indices to identify variables, the substitution operation requires a subroutine to renumber the indices of free variables 
in a term. For example, suppose we have a substitution $[x\rightarrow s](\lambda z. \: x)$, 
which is namelessly represented as $[0\rightarrow s](\lambda . \: 1)$, where all the x occurences in $\lambda z . \: x$ 
are replaced with $s$. For every binder that a variable is nested, its index is incremented by 1. Therefore, index 1 under the 
$\lambda$-abstraction and index 0 in the outer context refer to the same variable, that is $x$. When replacing $x$ with $s$, all the free variables 
in $s$ would have to be incremented by 1 as well because they are now under the $\lambda$-abstraction. However, this shifting 
operation is only applied to free variables. If $s$ contains a binder, as in $s = (\lambda y . \: y) \: h$, then 
the index for $h$ is shifted up by 1, but not the index for $y$ because it is bound. 

To limit the scope of shifting, we maintain a threshold parameter $c$ to distinguish free variables from bound variables. We initialize $c$ to 
0 and then increment it every time it passes a binder. When a term $t$ is to be shifted, $c$ indicates the number of binders within which $t$  
is nested. Therefore, any variable with an index $k < c$ is obviously bound to one of the binders; on the other hand, 
variables with indices $k \geq c$ are free and therefore should be shifted. The definition for shifting is defined as follows:
\begin{alignat*}{4}
&\uparrow _c^d(k) &= &  
    \begin{cases}   
        k & \text{if $k<c$} \\
        k + d & \text{if $k \geq c$} 
    \end{cases} \\
&\uparrow _c^d(\lambda . \: t_1) \: & = \: &\lambda . \: \uparrow _{c+1}^d (t_1) & \\
&\uparrow _c^d(t_1 \: t_2) \: & = \: & \uparrow _c^d(t_1) \: \uparrow _c^d(t_2)
\end{alignat*} 
where $c$ is the threshold number, $d$ is the shifting offset, and $k$ is the variable index. As a result, the substitution 
procedure $[j \rightarrow s]t$ is defined as:
\begin{alignat*}{3}
    &[j \rightarrow s]k &=& \:  
        \begin{cases}   
            s & \text{if $k=j$} \\
            k & \text{otherwise} 
        \end{cases} \\
    &[j \rightarrow s](\lambda . \: t_1) & \: \: = \: & \lambda . \: [j + 1 \rightarrow \uparrow^1 (s)]t_1 \\
    &[j \rightarrow s](t_1 \: t_2) & \: \: = \: & [j \rightarrow s]t_1 \: [j \rightarrow s]t_2
\end{alignat*} 
where $s$ is the term to substitute for the variable with index $j$ [6].

\begin{thebibliography}{9}
    \bibitem{diehl} 
    S. Diehl:  Parsing, \\
    \url{http://dev.stephendiehl.com/fun/002\char`_parsers.html}

    \bibitem{parsec-tutorial} 
    Haskell: Parsing a Simple Imperative Language, \\
    \url{https://wiki.haskell.org/Parsing\char`_a\char`_simple\char`_imperative\char`_language}

    \bibitem{henglein} 
    F. Henglein.
    Dynamic typing: Syntax and proof theory. 
    \textit{Sci. Comput. Program.},
    22(3):197-230, 1994.

    \bibitem{flanagan} 
    D. Herman, A. Tomb, and C. Flanagan. 
    Space-efficient gradual typing. 
    In \textit{Trends in Functional Prog.}, (TFP),
    pages XXVIII, April 2007.

    \bibitem{hutton} 
    Graham Hutton. 
    \textit{Programming in Haskell}. 
    University Press, Cambridge, 2007.

    \bibitem{pierce} 
    Benjamin C. Pierce. 
    \textit{Types and Programming Languages}. 
    The MIT Press, Cambridge, 2002.

    \bibitem{garcia} 
    J. G. Siek and R. Garcia. 
    Interpretations of the gradually-typed lambda calculus. 
    In \textit{Scheme and Functional Programming Workshop}, 
    pages 68-80, September 2012.

    \bibitem{taha} 
    J. G. Siek, R. Garcia, and W. Taha. 
    Exploring the design space of higher-order casts. 
    In \textit{European Symposium on Programming}, ESOP,
    pages 365-376, January 2010.

    \bibitem{siek} 
    J. G. Siek and W. Taha. 
    Gradual typing for functional languages. 
    In \textit{Scheme and Functional Programming Workshop}, pages 81-92, September 2006.

    \bibitem{source} 
    Source code, \\
    \url{https://github.com/thuytien140894/GTLC}
\end{thebibliography}
\end{document}