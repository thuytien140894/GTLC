Consider the following expression.
\begin{shiftedflalign*}
    & (\lambda x . \: \texttt{succ} \: x) \: \texttt{true} & 
\end{shiftedflalign*}
The typechecking process results in the following modified expression 
with additional coercions.
\begin{shiftedflalign*}
    & (\lambda x \! : ?. \: \texttt{succ} \: (\langle \texttt{Nat}? ^{l_1} \rangle x)) \: 
    \langle \texttt{Bool}! \rangle \texttt{true} &
\end{shiftedflalign*}  
The first step of evaluation gives
\begin{shiftedflalign*}
    & \texttt{succ} \: (\langle \texttt{Nat}? ^{l_1} \rangle 
    \langle \texttt{Bool}! \rangle \texttt{true}) & 
\end{shiftedflalign*}
Rule {\scriptsize{[E-CSTEP]}} combines the adjacent coercions 
$\langle \texttt{Nat}? ^{l_1} \rangle$ and $\langle \texttt{Bool}! \rangle$,
which evaluates to $\texttt{Fail}^{l_1}$. The result is a cast error.
\begin{shiftedflalign*}
    & \texttt{blame} \: l_1 & 
\end{shiftedflalign*}
Next, we review another example that involves higher-order coercions. 
\begin{shiftedflalign*}
    & (\lambda m . \: (\lambda x \! : \! \texttt{Nat} \! \rightarrow \! \texttt{Nat} . \: 
    (x \: 0)) \: m) \: (\lambda y \! : \! \texttt{Nat} . \: \texttt{succ} \: y)& 
\end{shiftedflalign*}
The cast insertion judgment produces the following expression. To keep it short and 
intuitive, we abbreviate 
$\texttt{Fun} \: c \: d$ as $\langle c \! \rightarrow \! d \rangle$. 
\begin{shiftedflalign*}
    & (\lambda m\!:? . \: (\lambda x \! : \! \texttt{Nat} \! \rightarrow \! \texttt{Nat} . \: 
    (x \: 0)) \: (\langle \texttt{Nat}! \! \rightarrow \! \texttt{Nat}? ^{l_1} \rangle \langle \texttt{Fun}? ^{l_2} \rangle m)) & \\ 
    & \hspace{6.2cm} \: \langle \texttt{Fun}! \rangle \langle \texttt{Nat}? ^{l_3} \! \rightarrow \! \texttt{Nat}! \rangle 
    (\lambda y \! : \! \texttt{Nat} . \: \texttt{succ} \: y) & 
\end{shiftedflalign*}
$m$ is dynamically-typed and coerced to the argument type $\texttt{Nat} \! \rightarrow \! \texttt{Nat}$. 
The function check $\langle \texttt{Fun}? ^{l_2} \rangle$ is inserted to ensure that $m$ first evaluates 
to a function. The value $\lambda y \! : \! \texttt{Nat} . \: \texttt{succ} \: y$ is coerced to type ? to 
be consistent with $m$. The tag $\langle \texttt{Fun}! \rangle$ upcasts a function type to ?.
The following are the steps of evaluation.
\begin{align*}
    & (\lambda m \! :? . \: (\lambda x \! : \! \texttt{Nat} \! \rightarrow \! \texttt{Nat} . \: 
    (x \: 0)) \: (\langle \texttt{Nat}! \! \rightarrow \! \texttt{Nat}? ^{l_1} \rangle \langle \texttt{Fun}? ^{l_2} \rangle m)) & \\
    & \hspace{6.1cm} \: \langle \texttt{Fun}! \rangle \langle \texttt{Nat}? ^{l_3} \! \rightarrow \! \texttt{Nat}! \rangle 
    (\lambda y \! : \! \texttt{Nat} . \: \texttt{succ} \: y) \\
    \xrightarrow{\text{ 1 }} \:
    &  (\lambda x \! : \! \texttt{Nat} \! \rightarrow \! \texttt{Nat} . \: 
    (x \: 0)) & \\ 
    & \hspace{2.2cm} \: \langle \texttt{Nat}! \! \rightarrow \! \texttt{Nat}? ^{l_1} \rangle \langle \texttt{Fun}? ^{l_2} \rangle 
    \langle \texttt{Fun}! \rangle \langle \texttt{Nat}? ^{l_3} \! \rightarrow \! \texttt{Nat}! \rangle 
    (\lambda y \! : \! \texttt{Nat} . \: \texttt{succ} \: y) \\
    \xrightarrow{\text{ 2 }} \:    
    & (\lambda x \! : \! \texttt{Nat} \! \rightarrow \! \texttt{Nat} . \: 
    (x \: 0)) & \\ 
    & \hspace{1.4cm} \: \langle \texttt{Nat}! \! \rightarrow \! \texttt{Nat}? ^{l_1} \rangle (\highlight{(\langle \texttt{Fun}? ^{l_2} \rangle 
    \langle \texttt{Fun}! \rangle)} \langle \texttt{Nat}? ^{l_3} \! \rightarrow \! \texttt{Nat}! \rangle) 
    (\lambda y \! : \! \texttt{Nat} . \: \texttt{succ} \: y) \\
    \xrightarrow{\text{ 3 }} \:
    & (\lambda x \! : \! \texttt{Nat} \! \rightarrow \! \texttt{Nat} . \: 
    (x \: 0)) \: \langle \texttt{Nat}! \! \rightarrow \! \texttt{Nat}? ^{l_1} \rangle \highlight{(\langle I \rangle
     \langle \texttt{Nat}? ^{l_3} \! \rightarrow \! \texttt{Nat}! \rangle)} 
    (\lambda y \! : \! \texttt{Nat} . \: \texttt{succ} \: y) \\
    \xrightarrow{\text{ 4 }} \:
    & (\lambda x \! : \! \texttt{Nat} \! \rightarrow \! \texttt{Nat} . \: 
    (x \: 0)) \: \highlight{\langle \texttt{Nat}! \! \rightarrow \! \texttt{Nat}? ^{l_1} \rangle 
     \langle \texttt{Nat}? ^{l_3} \! \rightarrow \! \texttt{Nat}! \rangle}
    (\lambda y \! : \! \texttt{Nat} . \: \texttt{succ} \: y) \\
    \xrightarrow{\text{ 5 }} \:
    & (\lambda x \! : \! \texttt{Nat} \! \rightarrow \! \texttt{Nat} . \: 
    (x \: 0)) \: \langle \highlight{\langle \texttt{Nat}? ^{l_3} \rangle \langle \texttt{Nat}! \rangle} \! \rightarrow \! 
    \highlight{\langle \texttt{Nat}? ^{l_1} \rangle \langle \texttt{Nat}! \rangle} \rangle
    (\lambda y \! : \! \texttt{Nat} . \: \texttt{succ} \: y) \\
    \xrightarrow{\text{ 6 }} \:
    & (\lambda x \! : \! \texttt{Nat} \! \rightarrow \! \texttt{Nat} . \: 
    (x \: 0)) \: \highlight{\langle I \! \rightarrow \! I \rangle}
    (\lambda y \! : \! \texttt{Nat} . \: \texttt{succ} \: y) \\
    \xrightarrow{\text{ 7 }} \:
    & (\lambda x \! : \! \texttt{Nat} \! \rightarrow \! \texttt{Nat} . \: 
    (x \: 0)) \: \highlight{\langle I \rangle
    (\lambda y \! : \! \texttt{Nat} . \: \texttt{succ} \: y)} \\
    \xrightarrow{\text{ 8 }} \:
    & (\lambda x \! : \! \texttt{Nat} \! \rightarrow \! \texttt{Nat} . \: 
    (x \: 0)) \: 
    (\lambda y \! : \! \texttt{Nat} . \: \texttt{succ} \: y) \\
    \xrightarrow{\text{ 9 }} \:
    & (\lambda y \! : \! \texttt{Nat} . \: \texttt{succ} \: y) \: 0 \\
    \xrightarrow{\text{ 10}} \:
    & \texttt{succ} \: 0
\end{align*}
The first step applies the rule {\scriptsize{[E-APP]}} to substitute 
$\langle \texttt{Fun}! \rangle \langle \texttt{Nat}? ^{l_3} \! \rightarrow \! \texttt{Nat}! \rangle 
(\lambda y \! : \! \texttt{Nat} . \: \texttt{succ} \: y)$ for $m$. The result gives 
a sequence of adjacent coercions that need to be reduced. First, we reassociate the coercions 
to find a pair that we can simplify, as in Step 2. The composition $\langle \texttt{Fun}? ^{l_2} \rangle 
\langle \texttt{Fun}! \rangle$ reduces to the identity $I$. This step satisfies the requirement that 
$m$ has to be a function. The next step is to check if $m$ has the expected function type of 
$\texttt{Nat} \! \rightarrow \! \texttt{Nat}$. The two function coercions $\langle \texttt{Nat}! \! 
\rightarrow \! \texttt{Nat}? ^{l_1} \rangle$ and $\langle \texttt{Nat}? ^{l_3} \! \rightarrow \! \texttt{Nat}! \rangle$ 
are reduced into one by composing the argument coercions and the output coercions respectively. The 
identity results from the compatibility of two coercions. We can then remove the cast from  
$\lambda y \! : \! \texttt{Nat} . \: \texttt{succ} \: y$ and perform another function application 
to get the final result. 
Our final example uses reference coercions. 
\begin{shiftedflalign*}
    & (\lambda y. \: (\lambda x . \: x := y)) \: 0 \: (\texttt{ref} \: \texttt{false})&
\end{shiftedflalign*}
which results in the following cast expression.
\begin{shiftedflalign*}
    & (\lambda y \! : ?. \: (\lambda x \! : ?. \: \langle \texttt{Ref}? ^{l_1} \rangle x := y)) 
    \: (\langle \texttt{Nat}! \rangle 0) \: (\langle \texttt{Ref}! \rangle
    \langle \texttt{Ref} \: \texttt{Bool}? ^{l_2} \: \texttt{Bool}! \rangle (\texttt{ref} \: \texttt{false}))&
\end{shiftedflalign*}
Since $x$ is on the left-hand side of the assignment, it is expected to be a reference through 
the reference check $\texttt{Ref}? ^{l_1}$. 0 is coerced to ? to match the dynamic type of $y$, 
and the tag $\texttt{Ref}!$ coerces $\texttt{ref} \: \texttt{false}$ 
to ? to be consistent with the dynamically-typed $x$. The steps of evaluation are as follows.
\begin{align*}
    & (\lambda y \! : ?. \: (\lambda x \! : ?. \: \langle \texttt{Ref}? ^{l_1} \rangle x := y)) 
    \: (\langle \texttt{Nat}! \rangle 0) \: (\langle \texttt{Ref}! \rangle
    \langle \texttt{Ref} \: \texttt{Bool}? ^{l_2} \: \texttt{Bool}! \rangle (\texttt{ref} \: \texttt{false})) \\
    \xrightarrow{\text{ 1 }} \:
    & (\lambda x \! : ?. \: \langle \texttt{Ref}? ^{l_1} \rangle x := \langle \texttt{Nat}! \rangle 0) 
    \: (\langle \texttt{Ref}! \rangle
    \langle \texttt{Ref} \: \texttt{Bool}? ^{l_2} \: \texttt{Bool}! \rangle (\texttt{ref} \: \texttt{false})) \\
    \xrightarrow{\text{ 2 }} \:
    & \highlight{(\langle \texttt{Ref}? ^{l_1} \rangle \langle \texttt{Ref}! \rangle)}
    \langle \texttt{Ref} \: \texttt{Bool}? ^{l_2} \: \texttt{Bool}! \rangle (\texttt{ref} \: \texttt{false})
    := \langle \texttt{Nat}! \rangle 0 \\
    \xrightarrow{\text{ 3 }} \:
    & \highlight{\langle I \rangle
    \langle \texttt{Ref} \: \texttt{Bool}? ^{l_2} \: \texttt{Bool}! \rangle} (\texttt{ref} \: \texttt{false})
    := \langle \texttt{Nat}! \rangle 0 \\
    \xrightarrow{\text{ 4 }} \:
    & \langle \texttt{Ref} \: \texttt{Bool}? ^{l_2} \: \texttt{Bool}! \rangle (\texttt{ref} \: \texttt{false})
    := \langle \texttt{Nat}! \rangle 0 \\
    \xrightarrow{\text{ 5 }} \:
    & \langle \texttt{Bool}! \rangle (\texttt{ref} \: \texttt{false})
    := \highlight{\langle \texttt{Bool}? ^{l_2} \rangle \langle \texttt{Nat}! \rangle} 0 \\
    \xrightarrow{\text{ 6 }} \:
    & \texttt{blame} \: l_2
\end{align*}
The first two steps substitute 0 for $y$ and 
$\langle \texttt{Ref} \hspace{0.2em} \texttt{Bool}? ^{l_2} \hspace{0.2em} \texttt{Bool}! \rangle 
(\texttt{ref} \hspace{0.2em} \texttt{false})$ 
for $x$. The reference checking coercions $\texttt{Ref}? ^{l_1}$ and $\texttt{Ref}!$ are eliminated 
since $x$ is a reference. We then apply rule {\scriptsize{[E-CASSIGN]}} 
to cast the \texttt{Nat} value written to $x$ to \texttt{Bool}, which results in a cast error.
