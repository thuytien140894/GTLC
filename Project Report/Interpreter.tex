The interpretation of $\lambda ^? _{\rightarrow}$ involves the following 
six stages:
\begin{gather*}
    \texttt{standard I/O} \xrightarrow{\text{chars}}
    \texttt{lexing} \xrightarrow{\text{tokens}} \texttt{parsing} \\
    \lhook\joinrel\xrightarrow{\text{terms}} \texttt{typechecking} 
    \xrightarrow{\text{cast terms}} \texttt{evaluation} 
    \xrightarrow{\text{value}/\text{error}} \texttt{pretty-printing}
\end{gather*}
where a program as a sequence of characters is first read from the 
standard input, tokenized by a lexical analyzer, and parsed into 
an abstract syntax tree. The program is then typechecked and 
modified with casts before being evaluated. The final 
result is then printed in a readable format to the standard output.  We start by 
defining the abstract syntax for the language.

\begin{lstlisting}
    data Term = Zero                       
               | Tru                        
               | Fls                        
               | Var Int Type String        
               | If Term Term Term          
               | Succ Term                   
               | Pred Term                  
               | IsZero Term                
               | Lambda Type Term [String]  
               | App Term Term              
               | Ref Term                   
               | Deref Term                 
               | Loc Int                    
               | Assign Term Term          
               | Cast Coercion Term   
\end{lstlisting}

Variables (\lstinline{Var Int Type String}) are represented as nameless terms using de Bruijn indices 
in order to avert the problem of capturing free variables. Accordingly, each variable 
is assigned a number to indicate the position of its binder. For example, 
$\lambda x. \: \lambda y. \: x \: y$ is rewritten as $\lambda. \: \lambda. 
\: 1 \: 0$. That is, $y$ is bound to the first binder, and $x$ to the second. 
Variables whose de Bruijn indices are greater than the total number of binders 
are considered free. Also, instead of maintaining a global type environment 
$\Gamma$, we have each variable remember its own bound type. We also 
include the original name of a variable for later printing.

Note that this syntax extends upon the one we have formalized in Section 2 
with cast terms (\lstinline{Cast Coercion Term}) and  
store locations (\lstinline{Loc Int}). These terms only arise as intermediate 
results from type checking and evaluation, and hence are not made available 
to programmers. Likewise, the data types for types and coercions are transcribed 
directly from their formal definitions, with one addition of type \lstinline{TUnit} 
used for free variables.

\begin{lstlisting}
    data Type = TUnit         
               | Dyn            
               | Boolean          
               | Nat            
               | Arr Type Type 
               | TRef Type      
\end{lstlisting}

\begin{lstlisting}
    data Coercion = Iden Type              
                  | Project Type Label     
                  | Inject Type             
                  | CRef Coercion Coercion  
                  | Func Coercion Coercion  
                  | Seq Coercion Coercion   
                  | Fail Type Type Label
\end{lstlisting}  

Given the syntax definition, the rest of this section will mainly focus on 
implementing the typechecker and 
evaluator for $\lambda ^? _{\rightarrow}$. The details on the 
lexer, parser, and pretty printer can be consulted in 
Appendix A, B, and C.

\subsection{Intermediate Language and Cast Insertion}
The static analysis of $\lambda ^? _{\rightarrow}$ both 
typechecks annotated terms and inserts coercions for 
dynamically-typed terms to ensure type soundness. The cast insertion 
rules, shown in Figure 6 [8], resemble the typing rules in 
Figure 2. 
The rules for variables, constants, and 
abstractions do not require any cast insertion. Rules {\scriptsize{[C-SUCC2]}}, 
{\scriptsize{[C-PRED2]}}, and {\scriptsize{[C-ISZERO2]}} handle the 
case when the argument type is dynamic by coercing it to \texttt{Nat}. 
The rules for \texttt{if} expressions make sure that the condition 
type is coerced to \texttt{Bool} if dynamic, and the two branch 
types are consistent by coercing one to another. Rule 
{\scriptsize{[C-APP1]}} coerces the argument type to the parameter 
type given that the operator is a function. Otherwise, we need to 
insert a function check $\langle \texttt{Fun}? ^l \rangle$ to verify 
that the operator resolves to a function at runtime. The rules for 
assignment follow the same pattern, casting the right-hand side type 
to the left-hand side type and inserting a reference check 
$\langle \texttt{Ref}? ^l \rangle$ when the left-hand side type is 
unknown. A reference check is also important when dereferencing a 
dynamic term, as in rule {\scriptsize{[C-DEREF2]}}.
\begin{figure}[hp]
    \caption{Cast Insertion Rules}
    \hrule
    \[
        \begin{bprooftree}
            \AxiomC{$(x\!:\!T) \in \Gamma$}
            \RightLabel{\scriptsize{\text{[C-VAR]}}}
            \UnaryInfC{$\Gamma \vdash x \hookrightarrow x : T$}
        \end{bprooftree}
        \begin{bprooftree}
            \AxiomC{$\Delta k = T$}
            \RightLabel{\scriptsize{\text{[C-CONST]}}}
            \UnaryInfC{$\Gamma \vdash k \hookrightarrow k : T$}
        \end{bprooftree}
    \]
    \vspace{0.5mm}
    \[
        \begin{bprooftree}
            \AxiomC{$\Gamma,x\!:\!S \vdash e \hookrightarrow t : T$}
            \RightLabel{\scriptsize{\text{[C-FUN]}}}
            \UnaryInfC{$\Gamma \vdash (\lambda x\!:\!S. \: e) 
            \hookrightarrow (\lambda x\!:\!S. \: t) : (S \rightarrow T)$}
        \end{bprooftree}
    \]
    \vspace{0.5mm}
    \[
        \begin{bprooftree}
            \AxiomC{$\Gamma \vdash e \hookrightarrow t : \texttt{Nat}$}
            \RightLabel{\scriptsize{\text{[C-SUCC1]}}}
            \UnaryInfC{$\Gamma \vdash \texttt{succ} \: e \hookrightarrow \texttt{succ} \: t : \texttt{Nat}$}
        \end{bprooftree}
        \begin{bprooftree}
            \AxiomC{$\Gamma \vdash e \hookrightarrow t : \texttt{Nat}$}
            \RightLabel{\scriptsize{\text{[C-PRED1]}}}
            \UnaryInfC{$\Gamma \vdash \texttt{pred} \: e \hookrightarrow \texttt{pred} \: t : \texttt{Nat}$}
        \end{bprooftree}
    \]
    \vspace{0.5mm}
    \[
        \begin{bprooftree}
            \AxiomC{$\Gamma \vdash e \hookrightarrow t : \texttt{Nat}$}
            \RightLabel{\scriptsize{\text{[C-ISZERO1]}}}
            \UnaryInfC{$\Gamma \vdash \texttt{iszero} \: e \hookrightarrow \texttt{iszero} \: t : \texttt{Bool}$}
        \end{bprooftree}
    \]
    \vspace{0.5mm}
    \[
        \begin{bprooftree}
            \AxiomC{$\Gamma \vdash e \hookrightarrow t : \: ?$}
            \RightLabel{\scriptsize{\text{[C-SUCC2]}}}
            \UnaryInfC{$\Gamma \vdash \texttt{succ} \: e 
            \hookrightarrow \texttt{succ} \: \langle \texttt{Nat}?^l \rangle t: \texttt{Nat}$}
        \end{bprooftree}
    \]
    \vspace{0.5mm}
    \[
        \begin{bprooftree}
            \AxiomC{$\Gamma \vdash e \hookrightarrow t : \: ?$}
            \RightLabel{\scriptsize{\text{[C-PRED2]}}}
            \UnaryInfC{$\Gamma \vdash \texttt{succ} \: e 
            \hookrightarrow \texttt{pred} \: \langle \texttt{Nat}?^l \rangle t: \texttt{Nat}$}
        \end{bprooftree}
    \]
    \vspace{0.5mm}
    \[
        \begin{bprooftree}
            \AxiomC{$\Gamma \vdash e \hookrightarrow t : \: ?$}
            \RightLabel{\scriptsize{\text{[C-ISZERO2]}}}
            \UnaryInfC{$\Gamma \vdash \texttt{iszero} \: e 
            \hookrightarrow \texttt{iszero} \: \langle \texttt{Nat}?^l \rangle t: \texttt{Bool}$}
        \end{bprooftree}
    \]
    \vspace{0.5mm}
    \[
        \begin{bprooftree}
            \AxiomC{$\Gamma \vdash e_1 \hookrightarrow t_1 : \texttt{Bool}$}
            \AxiomC{$\Gamma \vdash e_2 \hookrightarrow t_2 : S$}
            \AxiomC{$\Gamma \vdash e_3 \hookrightarrow t_3 : T$}
            \AxiomC{$c = \langle T \rightsquigarrow S \rangle$}
            \RightLabel{\scriptsize{\text{[C-IF1]}}}
            \QuaternaryInfC{$\Gamma \vdash \texttt{if} \: e_1 \: \texttt{then} \: e_2 \: 
            \texttt{else} \: e_3  
            \hookrightarrow \texttt{if} \: t_1 \: \texttt{then} \: t_2 \: 
            \texttt{else} \: \langle c \rangle t_3: \texttt{S}$}
        \end{bprooftree}
    \]
    \vspace{0.5mm}
    \[
        \begin{bprooftree}
            \AxiomC{$\Gamma \vdash e_1 \hookrightarrow t_1 : \: ?$}
            \AxiomC{$\Gamma \vdash e_2 \hookrightarrow t_2 : S$}
            \AxiomC{$\Gamma \vdash e_3 \hookrightarrow t_3 : T$}
            \AxiomC{$c = \langle T \rightsquigarrow S \rangle$}
            \RightLabel{\scriptsize{\text{[C-IF2]}}}
            \QuaternaryInfC{$\Gamma \vdash \texttt{if} \: e_1 \: \texttt{then} \: e_2 \: 
            \texttt{else} \: e_3  
            \hookrightarrow \texttt{if} \: \langle \texttt{Bool}?^l \rangle t_1 \: 
            \texttt{then} \: t_2 \: 
            \texttt{else} \: \langle c \rangle t_3: \texttt{S}$}
        \end{bprooftree}
    \]
    \vspace{0.5mm}
    \[
        \begin{bprooftree}
            \AxiomC{$\Gamma \vdash e_1 \hookrightarrow t_1 : S \rightarrow T$}
            \AxiomC{$\Gamma \vdash e_2 \hookrightarrow t_2 : S'$}
            \AxiomC{$c = \langle S' \rightsquigarrow S \rangle$}
            \RightLabel{\scriptsize{\text{[C-APP1]}}}
            \TrinaryInfC{$\Gamma \vdash e_1 \: e_2  
            \hookrightarrow t_1 \: \langle c \rangle t_2 : T$}
        \end{bprooftree}
    \]
    \vspace{0.5mm}
    \[
        \begin{bprooftree}
            \AxiomC{$\Gamma \vdash e_1 \hookrightarrow t_1 : \: ?$}
            \AxiomC{$\Gamma \vdash e_2 \hookrightarrow t_2 : S$}
            \RightLabel{\scriptsize{\text{[C-APP2]}}}
            \BinaryInfC{$\Gamma \vdash e_1 \: e_2  
            \hookrightarrow \langle \texttt{Fun}?^l \rangle t_1 \: 
            \langle S! \rangle t_2 : ?$}
        \end{bprooftree}
    \]
    \vspace{0.5mm}
    \[
        \begin{bprooftree}
            \AxiomC{$\Gamma \vdash e \hookrightarrow t : T$}
            \RightLabel{\scriptsize{\text{[C-REF]}}}
            \UnaryInfC{$\Gamma \vdash \texttt{ref} \: e  
            \hookrightarrow \texttt{ref} \: t : \texttt{Ref} \: T$}
        \end{bprooftree}
        \begin{bprooftree}
            \AxiomC{$\Gamma \vdash e \hookrightarrow t : \texttt{Ref} \: T$}
            \RightLabel{\scriptsize{\text{[C-DEREF1]}}}
            \UnaryInfC{$\Gamma \vdash \: !e  
            \hookrightarrow \: !t : T$}
        \end{bprooftree}
    \]
    \vspace{0.5mm}
    \[
        \begin{bprooftree}
            \AxiomC{$\Gamma \vdash e \hookrightarrow t : \: ?$}
            \RightLabel{\scriptsize{\text{[C-DEREF2]}}}
            \UnaryInfC{$\Gamma \vdash \: !e  
            \hookrightarrow \: !(\langle \texttt{Ref}?^l \rangle t) : \: ?$}
        \end{bprooftree}
    \]
    \vspace{0.5mm}
    \[
        \begin{bprooftree}
            \AxiomC{$\Gamma \vdash e_1 \hookrightarrow t_1 : \texttt{Ref} \: S$}
            \AxiomC{$\Gamma \vdash e_2 \hookrightarrow t_2 : T$}
            \AxiomC{$c = \langle T \rightsquigarrow S \rangle$}
            \RightLabel{\scriptsize{\text{[C-ASSIGN1]}}}
            \TrinaryInfC{$\Gamma \vdash (e_1 := e_2)  
            \hookrightarrow (t_1 := \langle c \rangle t_2) : S$}
        \end{bprooftree}
    \]
    \vspace{0.5mm}
    \[
        \begin{bprooftree}
            \AxiomC{$\Gamma \vdash e_1 \hookrightarrow t_1 : \: ?$}
            \AxiomC{$\Gamma \vdash e_2 \hookrightarrow t_2 : T$}
            \RightLabel{\scriptsize{\text{[C-ASSIGN2]}}}
            \BinaryInfC{$\Gamma \vdash (e_1 := e_2)  
            \hookrightarrow (\langle \texttt{Ref}?^l \rangle t_1 := 
            \langle T! \rangle t_2) : \: ?$}
        \end{bprooftree}
    \]
    \hrule
\end{figure} 

The cast insertion process is implemented as part of the typechecker. 
We also need to incorporate blame tracking whenever an unsafe cast 
is introduced to the original program. To keep it simple, we 
represent a blame label as an integer. A new label is obtained 
by simply incrementing the value of the last assigned label by 1. 
Additionally, the typechecker should be able to report errors on 
ill-typed expressions. In order to achieve both blame tracking 
and exception handling, a new monad \lstinline{TCheckState} is defined 
as follows:
\begin{lstlisting} 
    type TCheckState a = ExceptT TypeError (State Label) a
\end{lstlisting}
By combining the \lstinline{State} monad and the \lstinline{Except} monad, \lstinline{TCheckState} 
helps maintain a global counter for labeling unsafe coercions 
as well as return errors during type checking. The data type 
for type errors captures eight possible causes: (1) out-of-bound variables, 
(2) type difference between the two conditional branches, (3) non-boolean condition, 
(4) unexpected argument type for an arithmetic operator or (5) a function, 
(6) assigning to a non-reference term, 
(7) type mismatch between the two sides of an assignment, and (8) derefencing a 
non-reference term. 

\begin{lstlisting}
    data TypeError = NotBound Term                 
                   | Difference Type Type           
                   | NotBool Type                  
                   | NotNat Type    
                   | FunMismatch Type Type Term                  
                   | NotFunction Term               
                   | IllegalAssign Term            
                   | AssignMismatch Type Type Term  
                   | IllegalDeref Term   
\end{lstlisting}

We transliterate the cast insertion rules in Figure 6 
into the \lstinline{typeCheck'} function that takes in an expression as input 
and returns either a tuple of a modified expression and its type, 
or a type error.

\begin{lstlisting}
    typeCheck' :: Term -> TCheckState (Term, Type)
    -- | Constants
    typeCheck' e = case e of                              
        Tru  -> return (e, Boolean)                                    
        Fls  -> return (e, Boolean)                                   
        Zero -> return (e, Nat)     
    -- | Arithmetic                              
    typeCheck' (Succ e') -> 
        do (t', ty) <- typeCheck' e'
            case ty of  
                Dyn -> do c <- coerce ty Nat
                           return (Succ $ Cast c t', Nat)
                Nat -> return (Succ t', Nat)
                _   -> throwError $ NotNat ty
    typeCheck' (Pred e') -> 
        do (t', ty) <- typeCheck' e'
            case ty of
                Dyn -> do c <- coerce ty Nat
                           return (Pred $ Cast c t', Nat)
                Nat -> return (Pred t', Nat)
                _   -> throwError $ NotNat ty
    typeCheck' (IsZero e') -> 
        do (t', ty) <- typeCheck' e' 
            case ty of 
                Dyn -> do c <- coerce ty Nat
                           return (IsZero $ Cast c t', Boolean)
                Nat -> return (IsZero t', Boolean)
                _   -> throwError $ NotNat ty
    -- | Conditional
    typeCheck' (If e1 e2 e3) = 
        do (t1, cond) <- typeCheck' e1 
           (t2, fst)  <- typeCheck' e2 
           (t3, snd)  <- typeCheck' e3
           case cond of 
               Dyn 
                   | fst == snd -> do c1 <- coerce cond Boolean
                                       let t1' = Cast c1 t1
                                       return (If t1' t2 t3, fst)  
                   | fst `isConsistent` snd -> 
                       do c1 <- coerce cond Boolean
                          c2 <- coerce fst snd
                          c3 <- coerce snd fst 
                          let (t1', t2', t3') = (Cast c1 t1, Cast c2 t2, Cast c3 t3)
                          return (If t1' t2' t3', Dyn)   
                   | otherwise -> throwError $ Difference fst snd
               Boolean 
                   | fst == snd -> return (If t1 t2 t3, fst) 
                   | fst `isConsistent` snd -> 
                       do c2 <- coerce fst snd 
                          c3 <- coerce snd fst
                          let (t2', t3') = (Cast c2 t2, Cast c3 t3)
                          return (If t1 t2' t3', Dyn) 
                   | otherwise -> throwError $ Difference fst snd
               _ -> throwError $ NotBoolean cond      
    -- | Reference
    typeCheck' (Ref e') -> 
        do (t', ty) <- typeCheck' e'   
           return (Ref t', TRef ty)  
    -- | Dereference                           
    typeCheck' (Deref e') -> 
        do (t', ty) <- typeCheck' e' 
           case ty of 
               Dyn    -> do l <- GlobalS.newLabel
                            return (Deref $ Cast (RefProj l) t', Dyn)
               TRef s -> return (Deref t', s)
               _      -> throwError $ IllegalDeref e' 
    -- | Assignment
    typeCheck' (Assign e1 e2) -> 
        do (t1, s1)  <- typeCheck' e1 
           (t2, s2) <- typeCheck' e2 
           case s1 of 
               TRef s 
                   | s2 == s -> return (t1 `Assign` t2, s)
                   | s2 `isConsistent` s -> 
                       do c <- coerce s2 s
                          return (t1 `Assign` Cast c t2, s)
                   | otherwise -> 
                       throwError $ AssignMismatch s2 s (Assign e1 e2)
               Dyn -> 
                   do l  <- GlobalS.newLabel
                      c2 <- coerce s2 Dyn 
                      return (Cast (RefProj l) t1 `Assign` Cast c2 t2, Dyn)
               _ -> throwError $ IllegalAssign e1
    -- | Variable 
    typeCheck' (Var _ ty _) = case ty of                                            
        TUnit -> throwError $ NotBound e   
        _     -> return (e, ty) 
    -- | Abstraction
    typeCheck' (Lambda ty e' ctx) -> 
        do (t', retTy) <- typeCheck' e'            
            return (Lambda ty t' ctx, Arr ty retTy)  
    -- | Application
    typeCheck' (App e1 e2) -> 
        do (t1, funcTy) <- typeCheck' e1   
           (t2, argTy) <- typeCheck' e2
           case funcTy of 
               Dyn -> 
                   do l  <- GlobalS.newLabel 
                      c2 <- coerce argTy Dyn   
                      return (Cast (FuncProj l) t1 `App` Cast c2 t2, Dyn)
               Arr paramTy retTy 
                   | argTy == paramTy -> return (App t1 t2, retTy)
                   | argTy `isConsistent` paramTy -> 
                       do c <- coerce argTy paramTy  
                          return (App t1 $ Cast c t2, retTy)
                   | otherwise -> 
                       throwError $ FunMismatch argTy paramTy (App e1 e2)
               _ -> throwError $ NotFunction e1   
\end{lstlisting}

There are few slight changes to the formal type system. First is the 
omission of the global type environment for bound variables. Instead, each 
variable carries its own type, and those with \lstinline{TUnit} type are 
out of bound. Secondly, the function always checks for type 
consistency before performing coercions. \begin{figure}[ht!]
    \caption{Operational Semantics}
    \hrule
    \vspace{4mm}
    \qquad \text{\underline{S}yntax:}
    \begin{alignat*}{3}
        &(\textit{expressions}) \qquad 
        & e ::= \: & x \; | \;  
          e \: e \; | \;
          \lambda x \! : \! T . \: e \; | \;
          k \; | \; \langle c \rangle e\\
    &&    & \texttt{succ} \: e \; | \;
          \texttt{pred} \: e \; | \;
          \texttt{iszero} \: e \; | \; \\
    &&    & \texttt{if} \: e \: \texttt{then} \: e \: \texttt{else} \: e \; | \; \\
    &&    & \texttt{ref} \: e \; | \;
          !e \; | \;
          e := e \\
        &(\textit{uncoerced values}) \qquad
        & u ::= \: & k \; | \; a \; | \; \lambda x \! : \! T. \:e  \\
        &(\textit{values}) \qquad
        & v ::= \: & u \; | \; \langle \hat{c} \rangle u \\
        &(\textit{store locations}) \qquad
        & a \in \: & \{0 \texttt{x} m \: | \: m \in \mathbb{N}^+\}
    \end{alignat*}
    \begin{alignat*}{3}
        \qquad & \text{\underline{E}valuation Rules:} \\ \\
        & {\scriptsize{\text{[E-PREDZERO]}}} \qquad & \texttt{pred} \: 0 & \longrightarrow 0\\ 
        & {\scriptsize{\text{[E-PREDZERO]}}} \qquad & \texttt{pred} \: (\texttt{succ} \: n) & \longrightarrow n&\\
        & {\scriptsize{\text{[E-ISZEROZERO]}}} \qquad &\texttt{iszero} \: 0 & \longrightarrow \texttt{true} \\
        & {\scriptsize{\text{[E-ISZEROSUCC]}}} \qquad & \texttt{iszero} \: (\texttt{succ} \: n) & \longrightarrow \texttt{false} \\
        & {\scriptsize{\text{[E-IFTRUE]}}} \qquad & \texttt{if} \: \texttt{true} \: \texttt{then} \: t_2 \: \texttt{else} \: t_3 & \longrightarrow t_2 \\
        & {\scriptsize{\text{[E-IFFALSE]}}} \qquad & \texttt{if} \: \texttt{false} \: \texttt{then} \: t_2 \: \texttt{else} \: t_3 & \longrightarrow t_3 \\
        & {\scriptsize{\text{[E-REF]}}} \qquad & \texttt{ref} \: v & \longrightarrow a, \sigma[a := v] \hspace{1cm} & a \not \in dom(\sigma) \\
        & {\scriptsize{\text{[E-DEREF]}}} \qquad & !a & \longrightarrow \sigma(a) \\
        & {\scriptsize{\text{[E-ASSIGN]}}} \qquad & a := v & \longrightarrow v, \sigma[a := v] \\
        & {\scriptsize{\text{[E-APP]}}} \qquad & (\lambda x \! : \! T. \:e) \: v & \longrightarrow e[x \mapsto v] \\
        & {\scriptsize{\text{[E-ID]}}} \qquad & \langle I \rangle u & \longrightarrow u \\
        & {\scriptsize{\text{[E-COMP]}}} \qquad & \langle c \rangle \langle \hat{d} \rangle u & \longrightarrow \langle \hat{d};c \rangle u \\
        & {\scriptsize{\text{[E-FAIL]}}} \qquad & \langle \texttt{Fail}^l \rangle u & \longrightarrow \texttt{blame} \: l \\ 
        & {\scriptsize{\text{[E-GROUND]}}} \qquad & \langle \hat{c} \rangle u & \longrightarrow u \\
        & {\scriptsize{\text{[E-CSTEP]}}} \qquad & \langle c \rangle u & \longrightarrow \langle c' \rangle u \hspace{1cm} & c \longmapsto _X c' \\
        & {\scriptsize{\text{[E-CDEREF]}}} \qquad & !(\langle \texttt{Ref} \: c \: d \rangle a) & \longrightarrow \langle d \rangle !a \\
        & {\scriptsize{\text{[E-CASSIGN]}}} \qquad & \langle \texttt{Ref} \: c \: d \rangle a := v & \longrightarrow \langle d \rangle (a := \langle c \rangle v) \\
        & {\scriptsize{\text{[E-CAPP]}}} \qquad & (\langle \texttt{Fun} \: c \: d \rangle u) \: v & \longrightarrow \langle d \rangle (u \: \langle c \rangle v)
    \end{alignat*}
    \hrule
\end{figure} This sanity check allows the 
typechecker to handle some errors directly instead of delegating them 
to the evaluator through failure coercions. 
Additionally, we add another rule for application when the 
argument type is identical to the parameter type. As a result, no casts 
are needed. In this way, we eliminate unnecessary coercions that make 
no changes to the semantics. The same pattern is applied for conditional 
expressions and assignments by including a check for type equivalence before 
type consistency. Finally, we call the subroutine \lstinline{GlobalS.newLabel} 
to generate a new blame label for every unsafe cast inserted. 

\subsection{Operational Semantics}
Figure 7 formalizes the operational semantics for a typechecked 
program with coercions. In addition to the standard rules for 
the $\lambda$-calculus with references, the semantics also 
defines rules for cast expressions. Rule  
{\scriptsize{[E-ID]}} removes the identity coercion and 
returns the inner value. 
Rule {\scriptsize{[E-COMP]}} composes two 
adjacent coercions given that the first coercion is normalized. 
This restriction makes sure that coercions do not get accumulated 
without being reduced. 
The {\scriptsize{[E-GROUND]}} rule unwraps a cast value 
when its coercion is normalized. Otherwise, we take a small 
step of evaluation inside the coercion using the reduction rules 
in Figure 5. The combination of {\scriptsize{[E-CSTEP]}} and 
{\scriptsize{[E-COMP]}} helps maintain a bounded size on coercions. 
Rule {\scriptsize{[E-CDEREF]}} casts the value read 
from a cell, whereas {\scriptsize{[E-CASSIGN]}} also casts the value 
written to it. Rule 
{\scriptsize{[E-CAPP]}} splits the function coercion 
into two separate coercions for the argument and the result. 
Finally, we terminate the evaluation and throw a cast error 
immediately when the coercion $\texttt{Fail}^l$ is encountered, 
as described in the {\scriptsize{[E-FAIL]}} rule. The blame 
label consequently identifies the responsible coercion.

Similar to the \lstinline{typeCheck'} procedure, we define a 
monad \lstinline{SEvalState} to provide a global state for a set $\sigma$ of 
store locations and also to signal runtime errors.
\begin{lstlisting}
    type SEvalState a = ExceptT RuntimeError (State StoreEnv) a
\end{lstlisting}
The following implementation describes a small-step evaluator 
based on the established reduction rules in Figure 7.

\begin{lstlisting}
    evaluate' :: Term -> SEvalState Term
    evaluate' t = case t of
        -- | Arithmetic
        Pred Zero                       -> return Zero                         
        Pred (Succ nv) 
            | isNumeric nv              -> return nv                       
        Pred t'                         -> Pred <$> evaluate' t'
        IsZero Zero                     -> return Tru                        
        IsZero (Succ nv) 
            | isNumeric nv              -> return Fls                     
        IsZero t'                       -> IsZero <$> evaluate' t'
        Succ t'                         -> Succ <$> evaluate' t'

        -- | Conditional
        If Tru t2 t3                    -> return t2                        
        If Fls t2 t3                    -> return t3                         
        If t1 t2 t3                     -> (\t1' -> If t1' t2 t3) <$> evaluate' t1

        -- | Cast
        Cast c t' 
            | not (isVal t')            -> Cast c <$> evaluate' t'
        Cast c (Cast d u)               -> return $ Seq d c `Cast` u            
        Cast (Iden _) u                 -> return u                        
        Cast (Fail s1 s2 l) u           -> throwError $ Blame s1 s2 l                 
        Cast c u 
            | isNormalized c            -> unbox t                              
        Cast c u                        -> return $ reduceCoercion c `Cast` u
        Deref (Cast (CRef c d) (Loc l)) -> return $ Cast d $ Deref (Loc l)      
        Assign (Cast (CRef c d) (Loc l)) v                                                
            | isVal v                   -> return $ Cast d $ Loc l `Assign` Cast c v 
        App (Cast (Func c d) u) v 
            | isUncoercedVal u 
              && isVal v                -> return $ Cast d $ u `App` Cast c v                        

        -- | Reference
        Ref v     
            | isVal v                   -> GlobalS.allocate v                             
        Ref t'                          -> Ref <$> evaluate' t'
        Deref (Loc l)                   -> GlobalS.peek l

        -- | Dereference
        Deref t'                        -> Deref <$> evaluate' t'
                                            
        -- | Assignment
        Assign (Loc l) v
            | isVal v                   -> GlobalS.update l v                       
        Assign v1 t2 
            | isVal v1                  -> Assign v1 <$> evaluate' t2
        Assign t1 t2                    -> (`Assign` t2) <$> evaluate' t1

        -- | Application
        App (Lambda _ t1 _) v2 
            | isVal v2                  -> return $ subsFromTop v2 t1          
        App v1 t2 
            | isVal v1                  -> App v1 <$> evaluate' t2
        App t1 t2                       -> (`App` t2) <$> evaluate' t1                        
                        
        -- | No rules applied
        _                               -> throwError Stuck              
\end{lstlisting}

For the evaluation of a cast expression (\lstinline{Cast c t'}), 
we first check if the inner expression \lstinline{t'} is a value. If not, then 
we take a small-step of evaluation for \lstinline{t'}. Otherwise, 
we consider the next case where \lstinline{t'} is a cast value   
\lstinline{Cast d u}, where \lstinline{d} is normalized. In this 
case, we apply rule {\scriptsize{[E-COMP]}} to combine coercions \lstinline{c} 
and \lstinline{d}. The subsequent cases handle cast expressions where 
\lstinline{t'} is a simple value, and the coercion \lstinline{c} takes 
on different patterns. The 
\lstinline{unbox} function unwraps a cast expression whose coercion is normalized 
provided that the target type matches the runtime type.    
The subroutines \lstinline{GlobalS.allocate} and \lstinline{GlobalS.update} 
modify the global store $\sigma$ by allocating a new cell and 
updating an existing cell, respectively, while \lstinline{GlobalS.peek} 
reads from a cell. Function \lstinline{subsFromTop} 
performs a substitution 
operation $e[x \mapsto v]$ on nameless terms by renumbering 
free variables. Appendix D discusses this process in more detail. 
Runtime errors include cast errors and stuckness. Cast expressions 
with a failure coercion give a cast error. 
A stuck evaluation results when a term has no transition states 
because it is either a simple value or meaningless. Hence, a \lstinline{Stuck} 
exception is simply an indication that an evaluation has 
terminated. Otherwise, 
we keep applying the small-step \lstinline{evaluate'} on non-stuck terms 
to reach the final state. The following 
function performs the big-step semantics. 

\begin{lstlisting}
    evaluate :: Term -> Either RuntimeError Term
    evaluate t = 
        let (res, t') = GlobalS.runBEval $ evaluateToValue t StoreEnv.empty
        in case res of
               Right t'' 
                   | isUncoercedVal t'' -> Right t''
                   | otherwise          -> Left Stuck 
               Left (Blame s1 s2 l)     -> let cause = fromJust $ blame l t
                                               bres = BlameRes cause t'
                                           in Left $ CastError s1 s2 bres  
               Left err                 -> Left err 
\end{lstlisting}

The subroutine \lstinline{GlobalS.runBEval} initializes a global state 
for evaluation with an empty store. \lstinline{evaluateToValue} repeatedly 
applies the small-step evaluator until a stuck term or an error is encountered. 
A stuck term is only valid if it is an uncoerced value. Else, an actual error 
of a stuck evaluation is returned. In the case of a cast error, we call the 
\lstinline{blame} function with a label to search for the erroneous 
coercion in the input expression. The data type \lstinline{BlameRes} 
wraps this coercion along with the last evaluated term for the error message. 

\subsection{Overview}
Now that we have defined the type checking and evaluation procedures, 
our interpreter for $\lambda ^? _{\rightarrow}$ works as follows:
\begin{lstlisting}
    interpret :: String -> IO ()
    interpret line = case parseExpr line of 
        Right validExpr -> case typeCheck validExpr of 
                               Right t  -> case evaluate t of 
                                               Right res -> printMsg res
                                               Left err  -> printMsg err
                               Left err -> printMsg err
        Left err        -> print err   
\end{lstlisting}
The interpreter first parses the input stream into an abstract syntax tree. 
If parsing succeeds, then the parsed expression is typechecked; 
otherwise, we output a parse error. A well-typed expression is subsequently 
evaluated to a value, or a runtime error is thrown. Function \lstinline{printMsg} 
prints the result in a readable format (see Appendix C).

\subsection{Examples}
Consider the following expression:
\begin{shiftedflalign*}
    & (\lambda x . \: \texttt{succ} \: x) \: \texttt{true} & 
\end{shiftedflalign*}
The type checking process results in the following modified expression 
with additional coercions.
\begin{shiftedflalign*}
    & (\lambda x \! : ?. \: \texttt{succ} \: (\langle \texttt{Nat}? ^{l_1} \rangle x)) \: 
    \langle \texttt{Bool}! \rangle \texttt{true} &
\end{shiftedflalign*}  
The first step of evaluation gives
\begin{shiftedflalign*}
    & \texttt{succ} \: (\langle \texttt{Nat}? ^{l_1} \rangle 
    \langle \texttt{Bool}! \rangle \texttt{true}) & 
\end{shiftedflalign*}
Rule {\scriptsize{[E-COMP]}} combines the adjacent coercions 
$\langle \texttt{Nat}? ^{l_1} \rangle$ and $\langle \texttt{Bool}! \rangle$,
which evaluates to $\texttt{Fail}^{l_1}$. The result is a cast error.
\begin{shiftedflalign*}
    & \texttt{blame} \: l_1 & 
\end{shiftedflalign*}
Next, we review another example that involves higher-order coercions. 
\begin{shiftedflalign*}
    & (\lambda m . \: (\lambda x \! : \! \texttt{Nat} \! \rightarrow \! \texttt{Nat} . \: 
    (x \: 0)) \: m) \: (\lambda y \! : \! \texttt{Nat} . \: \texttt{succ} \: y)& 
\end{shiftedflalign*}
The cast insertion judgment produces the following expression. To keep it short and 
intuitive, we abbreviate 
$\texttt{Fun} \: c \: d$ as $\langle c \! \rightarrow \! d \rangle$. 
\begin{shiftedflalign*}
    & (\lambda m\!:? . \: (\lambda x \! : \! \texttt{Nat} \! \rightarrow \! \texttt{Nat} . \: 
    (x \: 0)) \: (\langle \texttt{Nat}! \! \rightarrow \! \texttt{Nat}? ^{l_1} \rangle \langle \texttt{Fun}? ^{l_2} \rangle m)) & \\ 
    & \hspace{6.2cm} \: \langle \texttt{Fun}! \rangle \langle \texttt{Nat}? ^{l_3} \! \rightarrow \! \texttt{Nat}! \rangle 
    (\lambda y \! : \! \texttt{Nat} . \: \texttt{succ} \: y) & 
\end{shiftedflalign*}
$m$ is dynamically-typed and coerced to the argument type $\texttt{Nat} \! \rightarrow \! \texttt{Nat}$. 
The function check $\langle \texttt{Fun}? ^{l_2} \rangle$ is inserted to ensure that $m$ first evaluates 
to a function. The value $(\lambda y \! : \! \texttt{Nat} . \: \texttt{succ} \: y)$ is coerced to type ? to 
be consistent with $m$. The tag $\langle \texttt{Fun}! \rangle$ upcasts a function type to ?.
The following are the steps of evaluation.
\begin{align*}
    & (\lambda m \! :? . \: (\lambda x \! : \! \texttt{Nat} \! \rightarrow \! \texttt{Nat} . \: 
    (x \: 0)) \: (\langle \texttt{Nat}! \! \rightarrow \! \texttt{Nat}? ^{l_1} \rangle \langle \texttt{Fun}? ^{l_2} \rangle m)) & \\
    & \hspace{6.1cm} \: \langle \texttt{Fun}! \rangle \langle \texttt{Nat}? ^{l_3} \! \rightarrow \! \texttt{Nat}! \rangle 
    (\lambda y \! : \! \texttt{Nat} . \: \texttt{succ} \: y) \\
    \xrightarrow{\text{ 1 }} \:
    &  (\lambda x \! : \! \texttt{Nat} \! \rightarrow \! \texttt{Nat} . \: 
    (x \: 0)) & \\ 
    & \hspace{2.2cm} \: \langle \texttt{Nat}! \! \rightarrow \! \texttt{Nat}? ^{l_1} \rangle \langle \texttt{Fun}? ^{l_2} \rangle 
    \langle \texttt{Fun}! \rangle \langle \texttt{Nat}? ^{l_3} \! \rightarrow \! \texttt{Nat}! \rangle 
    (\lambda y \! : \! \texttt{Nat} . \: \texttt{succ} \: y) \\
    \xrightarrow{\text{ 2 }} \:    
    & (\lambda x \! : \! \texttt{Nat} \! \rightarrow \! \texttt{Nat} . \: 
    (x \: 0)) & \\ 
    & \hspace{1.4cm} \: \langle \texttt{Nat}! \! \rightarrow \! \texttt{Nat}? ^{l_1} \rangle (\highlight{(\langle \texttt{Fun}? ^{l_2} \rangle 
    \langle \texttt{Fun}! \rangle)} \langle \texttt{Nat}? ^{l_3} \! \rightarrow \! \texttt{Nat}! \rangle) 
    (\lambda y \! : \! \texttt{Nat} . \: \texttt{succ} \: y) \\
    \xrightarrow{\text{ 3 }} \:
    & (\lambda x \! : \! \texttt{Nat} \! \rightarrow \! \texttt{Nat} . \: 
    (x \: 0)) \: \langle \texttt{Nat}! \! \rightarrow \! \texttt{Nat}? ^{l_1} \rangle \highlight{(\langle I \rangle
     \langle \texttt{Nat}? ^{l_3} \! \rightarrow \! \texttt{Nat}! \rangle)} 
    (\lambda y \! : \! \texttt{Nat} . \: \texttt{succ} \: y) \\
    \xrightarrow{\text{ 4 }} \:
    & (\lambda x \! : \! \texttt{Nat} \! \rightarrow \! \texttt{Nat} . \: 
    (x \: 0)) \: \highlight{\langle \texttt{Nat}! \! \rightarrow \! \texttt{Nat}? ^{l_1} \rangle 
     \langle \texttt{Nat}? ^{l_3} \! \rightarrow \! \texttt{Nat}! \rangle}
    (\lambda y \! : \! \texttt{Nat} . \: \texttt{succ} \: y) \\
    \xrightarrow{\text{ 5 }} \:
    & (\lambda x \! : \! \texttt{Nat} \! \rightarrow \! \texttt{Nat} . \: 
    (x \: 0)) \: \langle \highlight{\langle \texttt{Nat}? ^{l_3} \rangle \langle \texttt{Nat}! \rangle} \! \rightarrow \! 
    \highlight{\langle \texttt{Nat}? ^{l_1} \rangle \langle \texttt{Nat}! \rangle} \rangle
    (\lambda y \! : \! \texttt{Nat} . \: \texttt{succ} \: y) \\
    \xrightarrow{\text{ 6 }} \:
    & (\lambda x \! : \! \texttt{Nat} \! \rightarrow \! \texttt{Nat} . \: 
    (x \: 0)) \: \highlight{\langle I \! \rightarrow \! I \rangle}
    (\lambda y \! : \! \texttt{Nat} . \: \texttt{succ} \: y) \\
    \xrightarrow{\text{ 7 }} \:
    & (\lambda x \! : \! \texttt{Nat} \! \rightarrow \! \texttt{Nat} . \: 
    (x \: 0)) \: \highlight{\langle I \rangle
    (\lambda y \! : \! \texttt{Nat} . \: \texttt{succ} \: y)} \\
    \xrightarrow{\text{ 8 }} \:
    & (\lambda x \! : \! \texttt{Nat} \! \rightarrow \! \texttt{Nat} . \: 
    (x \: 0)) \: 
    (\lambda y \! : \! \texttt{Nat} . \: \texttt{succ} \: y) \\
    \xrightarrow{\text{ 9 }} \:
    & (\lambda y \! : \! \texttt{Nat} . \: \texttt{succ} \: y) \: 0 \\
    \xrightarrow{\text{ 10}} \:
    & \texttt{succ} \: 0
\end{align*}
The first step applies the rule {\scriptsize{[E-APP]}} to substitute 
$\langle \texttt{Fun}! \rangle \langle \texttt{Nat}? ^{l_3} \! \rightarrow \! \texttt{Nat}! \rangle 
(\lambda y \! : \! \texttt{Nat} . \: \texttt{succ} \: y)$ for $m$. The result gives 
a sequence of adjacent coercions that needs to be reduced. First, we reassociate the coercions 
to find a pair that we can simplify, as in Step 2. The composition $\langle \texttt{Fun}? ^{l_2} \rangle 
\langle \texttt{Fun}! \rangle$ reduces to the identity $I$. This step satisfies the requirement that 
$m$ has to be a function. The next step is to check if $m$ has the expected function type of 
$\texttt{Nat} \! \rightarrow \! \texttt{Nat}$. The two function coercions $\langle \texttt{Nat}! \! 
\rightarrow \! \texttt{Nat}? ^{l_1} \rangle$ and $\langle \texttt{Nat}? ^{l_3} \! \rightarrow \! \texttt{Nat}! \rangle$ 
are reduced into one by composing the argument coercions and the output coercions respectively. The 
identity results from the compatibility of two coercions. We can then remove the cast from  
$(\lambda y \! : \! \texttt{Nat} . \: \texttt{succ} \: y)$ and perform another function application 
to get the final result. 
Our final example uses reference coercions. 
\begin{shiftedflalign*}
    & (\lambda y. \: (\lambda x . \: x := y)) \: 0 \: (\texttt{ref} \: \texttt{false})&
\end{shiftedflalign*}
which results in the following cast expression:
\begin{shiftedflalign*}
    & (\lambda y \! : ?. \: (\lambda x \! : ?. \: \langle \texttt{Ref}? ^{l_1} \rangle x := y)) 
    \: (\langle \texttt{Nat}! \rangle 0) \: (\langle \texttt{Ref}! \rangle
    \langle \texttt{Ref} \: \texttt{Bool}? ^{l_2} \: \texttt{Bool}! \rangle (\texttt{ref} \: \texttt{false}))&
\end{shiftedflalign*}
Since $x$ is on the left-hand side of the assignment, it is expected to be a reference through 
the reference check $\texttt{Ref}? ^{l_1}$. 0 is coerced to ? to match the dynamic type of $y$, 
and the tag $\texttt{Ref}!$ coerces $\texttt{ref} \: \texttt{false}$ 
to ? to be consistent with the dynamically-typed $x$. The steps of evaluation are as follows.
\begin{align*}
    & (\lambda y \! : ?. \: (\lambda x \! : ?. \: \langle \texttt{Ref}? ^{l_1} \rangle x := y)) 
    \: (\langle \texttt{Nat}! \rangle 0) \: (\langle \texttt{Ref}! \rangle
    \langle \texttt{Ref} \: \texttt{Bool}? ^{l_2} \: \texttt{Bool}! \rangle (\texttt{ref} \: \texttt{false})) \\
    \xrightarrow{\text{ 1 }} \:
    & (\lambda x \! : ?. \: \langle \texttt{Ref}? ^{l_1} \rangle x := \langle \texttt{Nat}! \rangle 0) 
    \: (\langle \texttt{Ref}! \rangle
    \langle \texttt{Ref} \: \texttt{Bool}? ^{l_2} \: \texttt{Bool}! \rangle (\texttt{ref} \: \texttt{false})) \\
    \xrightarrow{\text{ 2 }} \:
    & \highlight{(\langle \texttt{Ref}? ^{l_1} \rangle \langle \texttt{Ref}! \rangle)}
    \langle \texttt{Ref} \: \texttt{Bool}? ^{l_2} \: \texttt{Bool}! \rangle (\texttt{ref} \: \texttt{false})
    := \langle \texttt{Nat}! \rangle 0 \\
    \xrightarrow{\text{ 3 }} \:
    & \highlight{\langle I \rangle
    \langle \texttt{Ref} \: \texttt{Bool}? ^{l_2} \: \texttt{Bool}! \rangle} (\texttt{ref} \: \texttt{false})
    := \langle \texttt{Nat}! \rangle 0 \\
    \xrightarrow{\text{ 4 }} \:
    & \langle \texttt{Ref} \: \texttt{Bool}? ^{l_2} \: \texttt{Bool}! \rangle (\texttt{ref} \: \texttt{false})
    := \langle \texttt{Nat}! \rangle 0 \\
    \xrightarrow{\text{ 5 }} \:
    & \langle \texttt{Bool}! \rangle (\texttt{ref} \: \texttt{false})
    := \highlight{\langle \texttt{Bool}? ^{l_2} \rangle \langle \texttt{Nat}! \rangle} 0 \\
    \xrightarrow{\text{ 6 }} \:
    & \texttt{blame} \: l_2
\end{align*}
The first two steps substitute $\langle \texttt{Nat}! \rangle 0$ for $y$ and 
$\langle \texttt{Ref}! \rangle \langle \texttt{Ref} \hspace{0.2em} \texttt{Bool}? ^{l_2} \hspace{0.2em} \texttt{Bool}! \rangle 
(\texttt{ref} \hspace{0.2em} \texttt{false})$ 
for $x$. The reference checking coercions $\texttt{Ref}? ^{l_1}$ and $\texttt{Ref}!$ are eliminated 
since $x$ is a reference. We then apply rule {\scriptsize{[E-CASSIGN]}} 
to cast the \texttt{Nat} value written to $x$ to \texttt{Bool}, which results in a cast error.




