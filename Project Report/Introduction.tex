The argument for static versus dynamic typing underlines the fundamentatal 
motivation behind programming language design and implementation. Static 
typing enforces explicit type annotations on variables and function 
declarations, thereby restricting their behaviors and interactions. 
Such strict constraint of types helps eliminate runtime bugs caused 
by type conflicts, thus improving code reliability and easing the 
burden of debugging. Statically-typed languages such as Java 
and C++ are known for their runtime performance since the overhead of 
type checking is done during compilation instead. Additionally, a large 
and complex application benefits from early error detection and better 
documentation found in a static type system. On the other hand, dynamic typing 
removes the necessity of type annotations in favor of programming 
flexibility and rapid development. JavaScript and Python are two notable 
examples of dynamically-typed languages suitable for quick prototyping 
and adjustment to changing specifications. However, the lack of 
annotations and static type checking could impair code quality and 
reliability. The respective strengths and 
weaknesses of both type systems motivate the endeavor to integrate them 
together into a hybrid system called \textit{\textbf{gradual typing}}. 

The term \textit{gradual typing} originally came up in Siek and 
Taha's work [9] to describe the combination of static and dynamic typing 
for functional languages. Gradual typing gives the programmer the ability 
to control the extent of static checking through optional type annotations. 
Function parameters with type annotations are typechecked during 
compilation as usually, whereas those without types are deferred until 
runtime for type checking. This approach benefits from the safety of static 
typing and flexibility of dynamic typing, making it suitable for developing 
software from quick prototyping to reliable code delivery. The formal 
definition of gradual typing was first introduced using the 
simply-typed lambda calculus with references, abbreviated as $\lambda ^? _{\rightarrow}$. 
The runtime semantics of $\lambda ^? _{\rightarrow}$ requires the 
translation of the source language to an intermediate one with explicit 
casts. The interpretation of $\lambda ^? _{\rightarrow}$ 
is simply the evaluation of the resulting cast expressions to produce 
either a final unboxed value or a cast error.

There have been many studies on how to efficiently design the runtime 
semantics of $\lambda ^? _{\rightarrow}$. Herman et al. [4] observe 
the space inefficiency of gradual typing wherein a program repeatedly applies 
the same cast to every method invocation, resulting in an unbounded space 
consumption of duplicate wrappers. They present a space-efficient gradual 
type system based on Henglein's \textit{\textbf{coercion calculus}} [3], 
which allows the composition of adjacent casts and reduces space consumption 
as a result. The coercion calculus also makes it possible for early error detection 
since failure to combine inconsistent coercions gets propagated 
immediately. Siek et al. [8] call this the \textit{\textbf{eager coercion calculus}} 
to contrast with the traditional lazy cast checking, which only detects cast 
errors during function application. Consider the following higher-order cast 
expression: 
\begin{equation}
    \langle \texttt{Bool} \rightarrow \texttt{Nat} \Leftarrow  
    \: ? \rightarrow \: ? \rangle 
    \langle ? \rightarrow \: ? \Leftarrow 
    \texttt{Nat} \rightarrow \texttt{Nat} \rangle
    (\lambda x \! : \! \texttt{Nat} . \; x)
\end{equation}
The eager cast checking ensures that each cast is consistent with the prior 
casts. Hence, even though the type $\texttt{Bool} \rightarrow \texttt{Nat}$ 
is consistent with $ ? \rightarrow \: ? $, 
it conflicts with the underlying type $\texttt{Nat} \rightarrow \texttt{Nat}$. 
On the other hand, lazy type checking approves this cast expression until 
it is applied to a boolean argument.

Another design choice for gradual typing is the addition of \textit{\textbf{blame assignment}} 
to identify the culprit coercion. It is clear in the expression (1) that the cast 
$\langle \texttt{Bool} \rightarrow \texttt{Nat} \Leftarrow 
\: ? \rightarrow \: ? \rangle$ is to blame because any cast from a 
dynamic type to a specific type is always unsafe. In a more complicated program 
when there can be multiple downcasts from ?, it is helpful to inform 
the programmer of the expressions responsible for coercion failures. Siek et al. [8] 
discuss the two blame assignment strategies: (1) UD shares the blame between 
upcasts and downcasts, whereas (2) D places the blame on downcasts only. UD and D 
differ by how a function type is coerced to ?, which will be discussed 
further in later sections.

In this paper, we present an interpretation of $\lambda ^? _{\rightarrow}$ 
using the eager coercion calculus with UD blame assignment. We first review the syntax 
and type system of $\lambda ^? _{\rightarrow}$ in the next section. 
Section 3 discusses the UD blame tracking strategy, and Section 4 integrates this 
feature into the formal definition of the coercion calculus. We then describe the 
implementation of our interpreter for $\lambda ^? _{\rightarrow}$ using Haskell. 
The last two sections put our work into context.