The $\lambda ^? _{\rightarrow}$ syntax is essentially the same as 
the simply-typed $\lambda$-calculus with references, except for the 
addition of dynamic types ?.
\begin{alignat*}{3}
    &(\textit{expressions}) \qquad 
    & e ::= \: & x \; | \;  
          e \: e \; | \;
          \lambda x \! : \! T . \: e \; | \;
          k \\
    &&    & \texttt{succ} \: e \; | \;
          \texttt{pred} \: e \; | \;
          \texttt{iszero} \: e \; | \;
          \texttt{if} \: e \: \texttt{then} \: e \: \texttt{else} \: e \; | \; \\
    &&    & \texttt{ref} \: e \; | \;
          !e \; | \;
          e := e \\
    &(\textit{constants}) 
    & k ::= \: & n \; | \; \texttt{true} \; | \; \texttt{false} \\
    &(\textit{numbers}) 
    & n ::= \: & 0 \; | \; \texttt{succ} \: n \\
    &(\textit{types}) 
    & T ::= \: & \texttt{Bool} \; | \;
            \texttt{Nat} \; | \;
            T \rightarrow T \; | \;
            \texttt{Ref} \: T \; | \;
            ?     
\end{alignat*}

Expressions include variables, abstractions, and applications, as well as 
arithmethic, boolean, and conditional terms found in the untyped $\lambda$-calculus. 
Reference expressions consist of allocation, dereference, and assignment.
Types include the base types (\texttt{Bool} and \texttt{Nat}), function 
types, reference types, and finally dynamic types. 
Constants like \texttt{true} and \texttt{false} inherently  
have \texttt{Bool} type, whereas numerical values are of type \texttt{Nat}. 
Function parameters without type annotations automatically 
assume dynamic types. 

The type system of $\lambda ^? _{\rightarrow}$ is based on the 
consistency relation between known and unknown types, given in 
Figure 1 [8]. Note that the consistency relation is reflexive and 
symmetric, but not transitive. Given $S \sim \: ?$ and $? \: \sim T$, it 
is not always the case that $S \sim T$.  

\begin{figure}[h]
    \caption{Type Consistency}
    \hrule
    \vspace{4mm}
    \[
        \begin{bprooftree}
            \AxiomC{}
            \RightLabel{\scriptsize{\text{[C-REFL]}}}
            \UnaryInfC{$T \sim T$}
        \end{bprooftree}
        \begin{bprooftree}
            \AxiomC{}
            \RightLabel{\scriptsize{\text{[C-DYNR]}}}
            \UnaryInfC{$T \sim \: ?$}
        \end{bprooftree}
        \begin{bprooftree}
            \AxiomC{}
            \RightLabel{\scriptsize{\text{[C-DYNL]}}}
            \UnaryInfC{$? \: \sim T$}
        \end{bprooftree}
    \]
    \vspace{1mm}
    \[
        \begin{bprooftree}
            \AxiomC{$S_1 \sim T_1$}
            \AxiomC{$S_2 \sim T_2$}
            \RightLabel{\scriptsize{\text{[C-FUN]}}}
            \BinaryInfC{$(S_1 \rightarrow S_2) \sim (T_1 \rightarrow  T_2)$}
        \end{bprooftree}
        \begin{bprooftree}
            \AxiomC{$S \sim T$}
            \RightLabel{\scriptsize{\text{[C-REF]}}}
            \UnaryInfC{$\texttt{Ref} \: S \sim \texttt{Ref} \: T$}
        \end{bprooftree}
    \]
    \hrule
\end{figure} 

Figure 2 presents the type system of $\lambda ^? _{\rightarrow}$. Rules 
{\scriptsize{[T-VAR]}}, {\scriptsize{[T-CONST]}}, {\scriptsize{[T-FUN]}}, 
and {\scriptsize{[T-REF]}}  
are straightforward. The rules for function applications handle either 
known or unknown function types. Given a specific function type, an 
application only accepts an argument type that is consistent to the 
parameter type. Otherwise, the argument can have any type, and the return 
type of the application is unknown. Similarly, there are two assignment 
rules for whether the left-hand side is dynamically-typed. 
A dereference expression only has a specific type if its argument is a 
reference. \begin{figure}[ht]
    \caption{$\lambda ^? _{\rightarrow}$ Type System}
    \hrule
    \[
        \begin{bprooftree}
            \AxiomC{$(x\!:\!T) \in \Gamma$}
            \RightLabel{\scriptsize{\text{[T-VAR]}}}
            \UnaryInfC{$\Gamma \vdash x : T$}
        \end{bprooftree}
        \begin{bprooftree}
            \AxiomC{$\Delta k = T$}
            \RightLabel{\scriptsize{\text{[T-CONST]}}}
            \UnaryInfC{$\Gamma \vdash k : T$}
        \end{bprooftree}
        \begin{bprooftree}
            \AxiomC{$\Gamma,x\!:\!S \vdash e : T$}
            \RightLabel{\scriptsize{\text{[T-FUN]}}}
            \UnaryInfC{$\Gamma \vdash \lambda x\!:\!S. \: e : S \rightarrow T$}
        \end{bprooftree}
    \]
    \vspace{1mm}
    \[
        \begin{bprooftree}
                \AxiomC{$\Gamma \vdash e : S$}
                \AxiomC{$S \sim \texttt{Nat}$}
                \RightLabel{\scriptsize{\text{[T-SUCC]}}}
                \BinaryInfC{$\Gamma \vdash \texttt{succ} \: e : \texttt{Nat}$}
        \end{bprooftree}
        \begin{bprooftree}
            \AxiomC{$\Gamma \vdash e : S$}
            \AxiomC{$S \sim \texttt{Nat}$}
            \RightLabel{\scriptsize{\text{[T-PRED]}}}
            \BinaryInfC{$\Gamma \vdash \texttt{pred} \: e : \texttt{Nat}$}
        \end{bprooftree}
    \]
    \vspace{1mm}
    \[
        \begin{bprooftree}
            \AxiomC{$\Gamma \vdash e : S$}
            \AxiomC{$S \sim \texttt{Nat}$}
            \RightLabel{\scriptsize{\text{[T-ISZERO]}}}
            \BinaryInfC{$\Gamma \vdash \texttt{iszero} \: e : \texttt{Bool}$}
        \end{bprooftree}
    \]
    \vspace{1mm}
    \[
        \begin{bprooftree}
                \AxiomC{$\Gamma \vdash e_1 : B$}
                \AxiomC{$B \sim \texttt{Bool}$}
                \AxiomC{$\Gamma \vdash e_2 : S$}
                \AxiomC{$\Gamma \vdash e_3 : T$}
                \AxiomC{$S \sim T$}
                \RightLabel{\scriptsize{\text{[T-IF]}}}
                \QuinaryInfC{$\Gamma \vdash \texttt{if} \: e_1 \: \texttt{then} \: e_2 \: 
                \texttt{else} \: e_3 : \: S$}
        \end{bprooftree}
    \]
    \vspace{1mm}
    \[
        \begin{bprooftree}
            \AxiomC{$\Gamma \vdash e_1 : S \rightarrow T$}
            \AxiomC{$\Gamma \vdash e_2 : S'$}
            \AxiomC{$S' \sim S$}
            \RightLabel{\scriptsize{\text{[T-APP1]}}}
            \TrinaryInfC{$\Gamma \vdash e_1 \: e_2 : T$}
        \end{bprooftree}
    \]
    \vspace{1mm}
    \[
        \begin{bprooftree}
            \AxiomC{$\Gamma \vdash e_1 : \: ?$}
            \AxiomC{$\Gamma \vdash e_2 : S$}
            \RightLabel{\scriptsize{\text{[T-APP2]}}}
            \BinaryInfC{$\Gamma \vdash e_1 \: e_2 : \: ?$}
        \end{bprooftree}
        \begin{bprooftree}
            \AxiomC{$\Gamma \vdash e : T$}
            \RightLabel{\scriptsize{\text{[T-REF]}}}
            \UnaryInfC{$\Gamma \vdash \texttt{ref} \: e : \texttt{Ref} \: T$}
        \end{bprooftree}
    \]
    \vspace{1mm}
    \[
        \begin{bprooftree}
            \AxiomC{$\Gamma \vdash e : \texttt{Ref} \: T$}
            \RightLabel{\scriptsize{\text{[T-DEREF1]}}}
            \UnaryInfC{$\Gamma \vdash \: !e : T$}
        \end{bprooftree}
        \begin{bprooftree}
            \AxiomC{$\Gamma \vdash e : \: ?$}
            \RightLabel{\scriptsize{\text{[T-DEREF2]}}}
            \UnaryInfC{$\Gamma \vdash \: !e : \: ?$}
        \end{bprooftree}
    \]
    \vspace{1mm}
    \[
        \begin{bprooftree}
            \AxiomC{$\Gamma \vdash e_1 : \texttt{Ref} \: T$}
            \AxiomC{$\Gamma \vdash e_2 : S$}
            \AxiomC{$S \sim T$}
            \RightLabel{\scriptsize{\text{[T-ASSIGN1]}}}
            \TrinaryInfC{$\Gamma \vdash e_1 := e_2 : S$}
        \end{bprooftree}
    \]
    \vspace{1mm}
    \[
        \begin{bprooftree}
            \AxiomC{$\Gamma \vdash e_1 : \: ?$}
            \AxiomC{$\Gamma \vdash e_2 : T$}
            \RightLabel{\scriptsize{\text{[T-ASSIGN2]}}}
            \BinaryInfC{$\Gamma \vdash e_1 := e_2 : \: ?$}
        \end{bprooftree}
    \]
    \hrule
    \end{figure}   We treat arithmetic 
and boolean expressions ($\texttt{succ} \: e$, $\texttt{pred} \: e$, $\texttt{iszero} \: e$) 
as constant functions that accept any type consistent to \texttt{Nat} and 
return either a \texttt{Nat} or a \texttt{Bool}. Conditional expressions are more 
complicated because we have to typecheck both the condition and the two 
branches. The condition type must be consistent to 
\texttt{Bool}. The types of two branches need to be consistent. If so, the 
resulting type can derive from either branch type.
